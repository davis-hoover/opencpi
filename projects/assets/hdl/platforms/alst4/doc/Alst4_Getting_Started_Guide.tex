\iffalse
This file is protected by Copyright. Please refer to the COPYRIGHT file
distributed with this source distribution.

This file is part of OpenCPI <http://www.opencpi.org>

OpenCPI is free software: you can redistribute it and/or modify it under the
terms of the GNU Lesser General Public License as published by the Free Software
Foundation, either version 3 of the License, or (at your option) any later
version.

OpenCPI is distributed in the hope that it will be useful, but WITHOUT ANY
WARRANTY; without even the implied warranty of MERCHANTABILITY or FITNESS FOR A
PARTICULAR PURPOSE. See the GNU Lesser General Public License for more details.

You should have received a copy of the GNU Lesser General Public License along
with this program. If not, see <http://www.gnu.org/licenses/>.
\fi

%----------------------------------------------------------------------------------------
% Required document specific properties
%----------------------------------------------------------------------------------------
\def\comp{alst4}
\def\Comp{ALST4}
\def\docTitle{\Comp{} Getting Started Guide}
\def\snippetpath{../../../../../../doc/av/tex/snippets}
%----------------------------------------------------------------------------------------
% Global latex header (this must be after document specific properties)
%----------------------------------------------------------------------------------------
\iffalse
This file is protected by Copyright. Please refer to the COPYRIGHT file
distributed with this source distribution.

This file is part of OpenCPI <http://www.opencpi.org>

OpenCPI is free software: you can redistribute it and/or modify it under the
terms of the GNU Lesser General Public License as published by the Free Software
Foundation, either version 3 of the License, or (at your option) any later
version.

OpenCPI is distributed in the hope that it will be useful, but WITHOUT ANY
WARRANTY; without even the implied warranty of MERCHANTABILITY or FITNESS FOR A
PARTICULAR PURPOSE. See the GNU Lesser General Public License for more details.

You should have received a copy of the GNU Lesser General Public License along
with this program. If not, see <http://www.gnu.org/licenses/>.
\fi

% Sets OpenCPI Version used throughout all the docs. This is updated by
% scripts/update-release.sh when a release is being made and must not
% be changed manually.
\def\ocpiversion{v2.2.0}

\documentclass{article}
\author{}  % Force author to be blank
\date{OpenCPI Release:\ \ \ocpiversion}  % Force date to be blank and override date with version
\title{OpenCPI\\\docTitle}  % docTitle must be defined before including this file
%----------------------------------------------------------------------------------------
% Paper size, orientation and margins
%----------------------------------------------------------------------------------------
\usepackage{geometry}
\geometry{
  letterpaper,  % paper type
  portrait,     % text direction
  left=.75in,   % left margin
  top=.75in,    % top margin
  right=.75in,  % right margin
  bottom=.75in  % bottom margin
}
%----------------------------------------------------------------------------------------
% Header/Footer
%----------------------------------------------------------------------------------------
\usepackage{fancyhdr} \pagestyle{fancy}  % required for fancy headers
\renewcommand{\headrulewidth}{0.5pt}
\renewcommand{\footrulewidth}{0.5pt}
\lhead{\small{\docTitle}}
\rhead{\small{OpenCPI}}
%----------------------------------------------------------------------------------------
% Various packages
%----------------------------------------------------------------------------------------
\usepackage{amsmath}
\usepackage[page,toc]{appendix}  % for appendix stuff
\usepackage{enumitem}
\usepackage{graphicx}   % for including pictures by file
\usepackage{hyperref}   % for linking urls and lists
\usepackage{listings}   % for coding language styles
\usepackage{pdflscape}  % for landscape view
\usepackage{pifont}     % for sideways table
\usepackage{ragged2e}   % for justify
\usepackage{rotating}   % for sideways table
\usepackage{scrextend}
\usepackage{setspace}
\usepackage{subfig}
\usepackage{textcomp}
\usepackage[dvipsnames,usenames]{xcolor}  % for color names see https://en.wikibooks.org/wiki/LaTeX/Colors
\usepackage{xstring}
\uchyph=0  % Never hyphenate acronyms like RCC
\renewcommand\_{\textunderscore\allowbreak}  % Allow words to break/newline on underscores
%----------------------------------------------------------------------------------------
% Table packages
%----------------------------------------------------------------------------------------
\usepackage[tableposition=top]{caption}
\usepackage{float}
\floatstyle{plaintop}
\usepackage{longtable}  % for long possibly multi-page tables
\usepackage{multicol}   % for more advanced table layout
\usepackage{multirow}   % for more advanced table layout
\usepackage{tabularx}   % c=center,l=left,r=right,X=fill
% These define tabularx columns "C" and "R" to match "X" but center/right aligned
\newcolumntype{C}{>{\centering\arraybackslash}X}
\newcolumntype{M}[1]{>{\centering\arraybackslash}m{#1}}
\newcolumntype{P}[1]{>{\centering\arraybackslash}p{#1}}
\newcolumntype{R}{>{\raggedleft\arraybackslash}X}
%----------------------------------------------------------------------------------------
% Block Diagram / FSM Drawings
%----------------------------------------------------------------------------------------
\usepackage{tikz}
\usetikzlibrary{arrows,decorations.markings,fit,positioning,shapes}
\usetikzlibrary{automata}  % used for the fsm
\usetikzlibrary{calc}      % for duplicating clients
\usepgfmodule{oo}          % to define a client box
%----------------------------------------------------------------------------------------
% Colors Used
%----------------------------------------------------------------------------------------
\usepackage{colortbl}
\definecolor{blue}{rgb}{.7,.8,.9}
\definecolor{ceruleanblue}{rgb}{0.16, 0.32, 0.75}
\definecolor{cyan}{rgb}{0.0,0.6,0.6}
\definecolor{darkgreen}{rgb}{0,0.6,0}
\definecolor{deepmagenta}{rgb}{0.8, 0.0, 0.8}
\definecolor{maroon}{rgb}{0.5,0,0}
%----------------------------------------------------------------------------------------
% Define where to hyphenate
%----------------------------------------------------------------------------------------
\hyphenation{Cent-OS}
\hyphenation{install-ation}
%----------------------------------------------------------------------------------------
% Define Commands & Rename Commands
%----------------------------------------------------------------------------------------
\newcommand{\code}[1]{\texttt{#1}}  % For inline code snippet or command line
\newcommand{\sref}[1]{Section~\ref{#1}}  % To quickly reference a section
\newcommand{\todo}[1]{\textcolor{red}{TODO: #1}\PackageWarning{TODO:}{#1}}  % To do notes
\renewcommand{\contentsname}{Table of Contents}
\renewcommand{\listfigurename}{List of Figures}
\renewcommand{\listtablename}{List of Tables}

% This gives a link to gitlab.io document. By default, it outputs the filename.
% You can optionally change the link, e.g.
% \githubio{FPGA\_Vendor\_Tools\_Installation\_Guide.pdf} vs.
% \githubio[\textit{FPGA Vendor Tools Installation Guide}]{FPGA\_Vendor\_Tools\_Installation\_Guide.pdf}
% or if you want the raw ugly URL to come out, \githubioURL{FPGA_Vendor_Tools_Installation_Guide.pdf}
\newcommand{\githubio}[2][]{% The default is for FIRST param!
\href{http://opencpi.gitlab.io/releases/\ocpiversion/docs/#2}{\ifthenelse{\equal{#1}{}}{\texttt{#2}}{#1}}}
\newcommand{\gitlabcom}[2][]{% The default is for FIRST param!
\href{http://gitlab.com/opencpi/#2}{\ifthenelse{\equal{#1}{}}{\texttt{#2}}{#1}}}
\newcommand{\githubioURL}[1]{\url{http://opencpi.gitlab.io/releases/\ocpiversion/docs/#1}}
% Lastly, if you want a SINGLE leading path stripped, e.g. assets/X.pdf => X.pdf:
\newcommand{\githubioFlat}[1]{%
\StrBehind{#1}{/}[\den]%
\href{http://opencpi.gitlab.io/releases/\ocpiversion/docs/#1}{\texttt{\den}}%
}
%----------------------------------------------------------------------------------------
% VHDL Coding Language Style
% modified from: http://latex-community.org/forum/viewtopic.php?f=44&t=22076
%----------------------------------------------------------------------------------------
\lstdefinelanguage{VHDL}
{
  basicstyle=\ttfamily\footnotesize,
  columns=fullflexible,keepspaces,  % https://tex.stackexchange.com/a/46695/87531
  keywordstyle=\color{ceruleanblue},
  commentstyle=\color{darkgreen},
  morekeywords={
    library, use, all, entity, is, port, in, out, end, architecture, of,
    begin, and, signal, when, if, else, process, end,
  },
  morecomment=[l]--
}
%----------------------------------------------------------------------------------------
% XML Coding Language Style
% modified from http://tex.stackexchange.com/questions/10255/xml-syntax-highlighting
%----------------------------------------------------------------------------------------
\lstdefinelanguage{XML}
{
  basicstyle=\ttfamily\footnotesize,
  columns=fullflexible,keepspaces,
  morestring=[s]{"}{"},
  morecomment=[s]{!--}{--},
  commentstyle=\color{darkgreen},
  moredelim=[s][\color{black}]{>}{<},
  moredelim=[s][\color{cyan}]{\ }{=},
  stringstyle=\color{maroon},
  identifierstyle=\color{ceruleanblue}
}
%----------------------------------------------------------------------------------------
% DIFF Coding Language Style
% modified from http://tex.stackexchange.com/questions/50176/highlighting-a-diff-file
%----------------------------------------------------------------------------------------
\lstdefinelanguage{diff}
{
  basicstyle=\ttfamily\footnotesize,
  columns=fullflexible,keepspaces,
  breaklines=true,                            % wrap text
  morecomment=[f][\color{ceruleanblue}]{@@},  % group identifier
  morecomment=[f][\color{red}]-,              % deleted lines
  morecomment=[f][\color{darkgreen}]+,        % added lines
  morecomment=[f][\color{deepmagenta}]{---},  % Diff header lines (must appear after +,-)
  morecomment=[f][\color{deepmagenta}]{+++},
}
%----------------------------------------------------------------------------------------
% Python Coding Language Style
%----------------------------------------------------------------------------------------
\lstdefinelanguage{python}
{
  basicstyle=\ttfamily\footnotesize,
  columns=fullflexible,keepspaces,
  keywordstyle=\color{ceruleanblue},
  commentstyle=\color{darkgreen},
  stringstyle=\color{orange},
  morekeywords={
    print, if, sys, len, from, import, as, open,close, def, main, for, else,
    write, read, range,
  },
  comment=[l]{\#}
}
%----------------------------------------------------------------------------------------
% Fontsize Notes in order from smallest to largest
%----------------------------------------------------------------------------------------
%    \tiny
%    \scriptsize
%    \footnotesize
%    \small
%    \normalsize
%    \large
%    \Large
%    \LARGE
%    \huge
%    \Huge

%----------------------------------------------------------------------------------------

\begin{document}
\maketitle
\thispagestyle{empty}
\newpage

\tableofcontents
\newpage

\section*{Zipper Deprecation Notice:}
Beginning with OpenCPI Version 1.5, support for Lime Microsystems' Zipper card is now deprecated.

\section*{ALST4 Getting Started Guide}
\setcounter{section}{0}

\section{Hardware Prerequisites}
This section describes the hardware prerequisites required for an operational alst4 (Altera Stratix IV) platform using OpenCPI. The optional HSMC Debug Loopback and HSMC Debug Breakout Header cards are only intended for testing purposes. Also note that the slot configurations in Table \ref{table:supported_slots} are limited by what FPGA bitstreams are currently built by OpenCPI and not by what hardware configurations are theoretically possible using OpenCPI.\\ \\
Hardware prerequisites are as follows.
\begin{itemize}
\item A Stratix IV GX230 board, which has undergone an OpenCPI-specific initial one-time hardware setup \cite{alst4_hardware_setup} and is plugged into a PCIE slot of an x86 computer.
\item Optionally, one of the following HSMC card configurations in Table  \ref{table:supported_slots} may exist
\end{itemize}
\begin{center}
        \begin{table}[!htbp]
        \centering
        \caption{OpenCPI-supported Stratix IV hardware HSMC slot configurations}
        \label{table:supported_slots}
        \begin{tabular}{c|c|c|}
                \cline{2-3}
                & HSMC A slot & HSMC B slot \\ \hline
                \multicolumn{1}{|c|}{Test loopback A setup} & HSMC Debug Loopback Card & (empty)\\ \hline
                \multicolumn{1}{|c|}{Test loopback B setup} & (empty) & HSMC Debug Loopback Card \\ \hline
                \multicolumn{1}{|c|}{Test dual loopback setup} & HSMC Debug Loopback Card & HSMC Debug Loopback Card \\ \hline
                \multicolumn{1}{|c|}{Test breakout A setup} & HSMC Debug Breakout Header Card & (empty)\\ \hline
                \multicolumn{1}{|c|}{Test breakout B setup} & (empty) & HSMC Debug Breakout Header Card \\ \hline
                \multicolumn{1}{|c|}{Zipper A setup\ref{deprecation_zipper}} & Modified\cite{zipper_mods} Zipper/MyriadRF & (empty)\\
                \multicolumn{1}{|c|}{ } & transceiver card & \\ \hline
                \multicolumn{1}{|c|}{Zipper B setup\ref{deprecation_zipper}} & (empty) & Modified\cite{zipper_mods} Zipper/MyriadRF \\
                \multicolumn{1}{|c|}{ } & & transceiver card \\ \hline
        \end{tabular}
        \end{table}
        \footnotetext[1]{\label{deprecation_zipper}Deprecated Support as
        of OpenCPI 1.5}
\end{center}

\section{Software Prerequisites}
\begin{itemize}
\item A CentOS 6 or CentOS 7 operating system installed on the x86 computer.
\item Altera Quartus installed on the x86 computer. For more information refer to \cite{fpga_vendor_tool_guide}
\item OpenCPI framework and prerequisite RPMs installed on the x86 computer. For more information refer to \cite{rpm_installation_guide}
\item OpenCPI core project compiled for alst4.
\item OpenCPI assets project compiled for alst4.
\end{itemize}

\section{Reserve Memory for Driver}
\begin{flushleft}

When OpenCPI communicates to cards via PCI, it uses a loadable Linux kernel device driver
for discovery and DMA-based communication, which requires local (reserved) DMA memory
resources. DMA memory resources must be allocated or reserved on the CPU-side memory,
that is accessible to both the CPU (via the local mmap system call), as well as,
OpenCPI's PCI DMA engine with the board is issuing PCI READ or WRITE TLPs. By default,
Linux allocates 128 KB of memory for the OpenCPI driver. However, OpenCPI applications may have buffering requirements that necessitate additional memory resources. \\ \medskip

In the example provided below, special measures (memmap=) are used to allocate 128 MB of memory. The memmap parameter is used to reserved more block memory from the Linux kernel. While this variable supports many formats, the following usage has proven to be sufficient: \\ \bigskip
	memmap=SIZE\$START \\ \bigskip
Where SIZE is the number of bytes to reserve in either hexadecimal or decimal, and
START is the physical address in hexadecimal bytes. It is required that the pages for all addresses and sizes are on even boundaries (0x1000 or 4096 bytes). \\

\subsection{Calculate Values in Preparation for Memory Reservation}
At this time, the OpenCPI PCI DMA engine requires that the user-mode DMA memory pool be in a 32 or 64-bit memory range and due to the manner with which Linux manages memory, it is recommended that the address be higher than the first 24 bits. With these requirements, the first step is to find a “usable” contiguous memory range by examining the BIOS physical RAM map as reported by dmesg.\\ \medskip

Run dmesg and filter on BIOS to review the physical RAM map: \\
\lstset{language=bash, backgroundcolor=\color{lightgray}, columns=flexible, breaklines=true, prebreak=\textbackslash, basicstyle=\ttfamily, showstringspaces=false,upquote=true, aboveskip=\baselineskip, belowskip=\baselineskip}
\begin{lstlisting}
dmesg | grep BIOS
\end{lstlisting}
The output will look something like:
\begin{lstlisting}
BIOS-provided physical RAM map:
 BIOS-e820: 0000000000000000 - 000000000009f800 (usable)
 BIOS-e820: 000000000009f800 - 00000000000a0000 (reserved)
 BIOS-e820: 00000000000ca000 - 00000000000cc000 (reserved)
 BIOS-e820: 00000000000dc000 - 00000000000e4000 (reserved)
 BIOS-e820: 00000000000e8000 - 0000000000100000 (reserved)
 BIOS-e820: 0000000000100000 - 000000005fef0000 (usable)
 BIOS-e820: 000000005fef0000 - 000000005feff000 (ACPI data)
 BIOS-e820: 000000005feff000 - 000000005ff00000 (ACPI NVS)
 BIOS-e820: 000000005ff00000 - 0000000060000000 (usable)
 BIOS-e820: 00000000e0000000 - 00000000f0000000 (reserved)
 BIOS-e820: 00000000fec00000 - 00000000fec10000 (reserved)
 BIOS-e820: 00000000fee00000 - 00000000fee01000 (reserved)
 BIOS-e820: 00000000fffe0000 - 0000000100000000 (reserved)
\end{lstlisting}

Select a "(usable)" section of memory and reserve a subsection of that memory. Once the memory is reserved, the Linux kernel will ignore it. In this example, there are three usable sections:\\
\begin{lstlisting}
 BIOS-e820: 0000000000000000 - 000000000009f800 (usable)
 BIOS-e820: 0000000000100000 - 000000005fef0000 (usable)
 BIOS-e820: 000000005ff00000 - 0000000060000000 (usable)
\end{lstlisting}

Upon close review of the usable regions, the first range is too small and below the first 24 bits, while the third ranges is simply too small. Fortunately the second address space meets the address range requirement (between 24 and 32 bits) and it is large enough for to reserve several hundred megabytes of memory. \\ \medskip

The starting memory address for the user-mode DMA region is calculated by subtracting 0x08000000 (128 MB) from the largest memory region available, as long as it is greater than 0x08000000 (128MB) and inside the 32-bit address range (address is less than 4GB). In this example, the 2nd region is the largest: 0x5FEF0000 - 0x100000 = 0x5FDF0000 = 1,608,450,048 (~1.6GB) and it is inside of the 32-bit address space. The starting memory address (0x5FEF0000 - 0x08000000) is 0x57EF0000. And this is the value that used to construct the memmap parameter, as shown below:\\ \medskip

memmap=128M\$0x57EF0000 \\ \medskip

When calculating the starting address, the user must ensure that address occurs on an even page boundary of 4 KB. This may necessitate an additional adjustment to the starting address. \\ \medskip

In some cases, the \texttt{\$dmesg | grep BIOS} returns a value like 0x5FEFFFFF. It is recommended that the user simply change this address, such that, its low word is all zeros, ex. 0x5FEF0000, prior to calculating the starting address. \\ \medskip

\subsection{Configure Memory Reservation}
\textbf{Critical Note:
If other memmap parameters are implemented, e.g. for non-OpenCPI PCI cards, then grubby usage will be different. The OpenCPI driver will use the first memmap parameter on the command line OR the parameter ``opencpi\_memmap'' if it is explicitly given. If this parameter is given, the standard memmap command with the same parameters must ALSO be passed to the kernel.}\\ \bigskip

Once the memmap parameter as been calculated, it will need to be added to the kernel command line in the boot loader. \\
\bigskip
For CentOS, the utility ``grubby'' can be used to add the parameter to all kernels in the start-up menu. The single quotes are REQUIRED or the shell will interpret the \$0: \\
\bigskip
\textbf{\textit{CentOS 7}} uses \textit{grub2}, which \textbf{requires a DOUBLE} backslash:\\
\begin{lstlisting}
sudo grubby --update-kernel=ALL --args=memmap='128M\\$0x57EF0000'
\end{lstlisting}

To verify the current kernel has the argument set:\\
\begin{lstlisting}
sudo -v
sudo grubby --info $(sudo grubby --default-kernel)
\end{lstlisting}

\textbf{\textit{CentOS 7}} displays a \textbf{SINGLE} backslash before the \$, for example: \\
\begin{lstlisting}
args="ro rdblacklist=nouveau crashkernel=auto rd.lvm.lv=vg.0/root quiet audit=1 boot=UUID=96933cb5-f478-4933-a0d4-16953cf47f5c memmap=128M\$0x57EF0000 LANG=en_US.UTF-8"
\end{lstlisting}

If no longer desired, the parameter can also be removed:
\begin{lstlisting}
sudo grubby --update-kernel=ALL --remove-args=memmap
\end{lstlisting}

More information concerning grubby can be found at:\\
\url{https://access.redhat.com/documentation/en-US/Red_Hat_Enterprise_Linux/7/html/System_Administrators_Guide/sec-Making_Persistent_Changes_to_a_GRUB_2_Menu_Using_the_grubby_Tool.html}
\bigskip

For the memmap parameter:\\
\url{https://www.kernel.org/doc/html/latest/admin-guide/kernel-parameters.html}

\subsection{Apply Memory Reservation}
Reboot the system, making certain to boot from the new configuration.
\subsection{Verify Memory Reservation}
Once the system has finished booting, examine the state of the physical RAM map to confirm that the desired memory has been reserved:\\
\bigskip
\begin{lstlisting}
dmesg | more
Linux version 2.6.18-128.el5 (mockbuild@hs20-bc1-7.build.redhat.com) (gcc version 4.1.2 20080704 (Red Hat 4.1.2-44)) #1 SMP Wed Dec 17 11:41:38 EST 2008
Command line: ro root=/dev/VolGroup00/LogVol00 rhgb quiet memmap=128M$0x57EF0000
BIOS-provided physical RAM map:
 BIOS-e820: 0000000000000000 - 000000000009f800 (usable)
 BIOS-e820: 000000000009f800 - 00000000000a0000 (reserved)
 BIOS-e820: 00000000000ca000 - 00000000000cc000 (reserved)
 BIOS-e820: 00000000000dc000 - 00000000000e4000 (reserved)
 BIOS-e820: 00000000000e8000 - 0000000000100000 (reserved)
 BIOS-e820: 0000000000100000 - 000000005fef0000 (usable)
 BIOS-e820: 000000005fef0000 - 000000005feff000 (ACPI data)
 BIOS-e820: 000000005feff000 - 000000005ff00000 (ACPI NVS)
 BIOS-e820: 000000005ff00000 - 0000000060000000 (usable)
 BIOS-e820: 00000000e0000000 - 00000000f0000000 (reserved)
 BIOS-e820: 00000000fec00000 - 00000000fec10000 (reserved)
 BIOS-e820: 00000000fee00000 - 00000000fee01000 (reserved)
 BIOS-e820: 00000000fffe0000 - 0000000100000000 (reserved)
user-defined physical RAM map:
 user: 0000000000000000 - 000000000009f800 (usable)
 user: 000000000009f800 - 00000000000a0000 (reserved)
 user: 00000000000ca000 - 00000000000cc000 (reserved)
 user: 00000000000dc000 - 00000000000e4000 (reserved)
 user: 00000000000e8000 - 0000000000100000 (reserved)
 user: 0000000000100000 - 0000000057ef0000 (usable)
 user: 0000000057ef0000 - 000000005fef0000 (reserved)  <== New
 user: 000000005fef0000 - 000000005feff000 (ACPI data)
 user: 000000005feff000 - 000000005ff00000 (ACPI NVS)
 user: 000000005ff00000 - 0000000060000000 (usable)
 user: 00000000e0000000 - 00000000f0000000 (reserved)
 user: 00000000fec00000 - 00000000fec10000 (reserved)
 user: 00000000fee00000 - 00000000fee01000 (reserved)
 user: 00000000fffe0000 - 0000000100000000 (reserved)
DMI present.
\end{lstlisting}

A new "(reserved)" area is shown between the second "(useable)" section and the (ACPI data) section. Now, when the "ocpidriver load" is ran, it will detect the new reserved area, and pass that data to the OpenCPI kernel module. \\
\end{flushleft}


\section{Driver Notes}
\input{\snippetpath/Driver_Snippet}

\section{Loading the OpenCPI driver}
When OpenCPI is installed via RPMs, the OpenCPI driver should have been installed. However, when developing with source OpenCPI, the user is required to manage the loading of the OpenCPI driver. \\
The following terminal outputs are intended to provide the user with expected behavior of when the driver is not and is loaded. The user should note that only when the driver is installed can the alst4 be discovered as a valid OpenCPI container.

\begin{lstlisting}
ocpidriver unload
The driver module was successfully unloaded.

ocpidriver load
Found generic reserved DMA memory on the linux boot command line and assuming it is for OpenCPI: [memmap=128M$0x1000000]
Driver loaded successfully.

ocpidriver unload
The driver module was successfully unloaded.

ocpirun -C
OCPI( 2:816.0497): When searching for PCI device '0000:03:00.0': Can't open /dev/mem, forgot to load the driver? sudo?
OCPI( 2:816.0499): When searching for PCI device '0000:08:00.0': Can't open /dev/mem, forgot to load the driver? sudo?
OCPI( 2:816.0544): In HDL Container driver, got PCI search error: Can't open /dev/mem, forgot to load the driver? sudo?
Available containers:
 #  Model Platform       OS     OS-Version  Arch     Name
 0  rcc   centos7        linux  c7          x86_64   rcc0

ocpidriver load
Found generic reserved DMA memory on the linux boot command line and assuming it is for OpenCPI: [memmap=128M$0x1000000]
Driver loaded successfully.

ocpirun -C
Available containers:
 #  Model Platform       OS     OS-Version  Arch     Name
 0  hdl   ml605                                      PCI:0000:08:00.0
 1  hdl   alst4                                      PCI:0000:03:00.0
 2  rcc   centos7        linux  c7          x86_64   rcc0
\end{lstlisting}

\section{Proof of Operation}
The following commands may be run in order to verify correct OpenCPI operation on the x86/Stratix IV system.\\ \\
Existence of alst4 RCC/HDL containers may be verified by running the following command and verifying that similar output is produced.\\
\lstset{language=bash, backgroundcolor=\color{lightgray}, columns=flexible, breaklines=true, prebreak=\textbackslash, basicstyle=\ttfamily, showstringspaces=false,upquote=true, aboveskip=\baselineskip, belowskip=\baselineskip}
\begin{lstlisting}
ocpirun -C
Available containers:
 #  Model Platform       OS     OS-Version  Arch     Name
 0  rcc   centos7        linux  c7          x86_64   rcc0
 1  hdl   alst4                                      PCI:0000:02:00.0
\end{lstlisting}
Operation of the RCC container can be verified by running the hello application via the following command and verifying that identical output is produced. Note that the OCPI\_LIBRARY\_PATH environment variable must be setup to include the hello\_world.rcc built shared object file prior to running this command.
\begin{lstlisting}
ocpirun -t 1 assets/applications/hello.xml
Hello, world
\end{lstlisting}
Simultaneous RCC/HDL container operation can be verified by running the testbias application via the following command and verifying that identical output is produced. Note that the OCPI\_LIBRARY\_PATH environment variable must be setup correctly for your system prior to running this command.\\
\begin{lstlisting}
ocpirun -d -m bias=hdl assets/applications/testbias.xml
Property  0: file_read.fileName = "test.input" (cached)
Property  1: file_read.messagesInFile = "false" (cached)
Property  2: file_read.opcode = "0" (cached)
Property  3: file_read.messageSize = "16"
Property  4: file_read.granularity = "4" (cached)
Property  5: file_read.repeat = "<unreadable>"
Property  6: file_read.bytesRead = "0"
Property  7: file_read.messagesWritten = "0"
Property  8: file_read.suppressEOF = "false"
Property  9: file_read.badMessage = "false"
Property 10: file_read.ocpi_debug = "false" (parameter)
Property 11: file_read.ocpi_endian = "little" (parameter)
Property 12: bias.biasValue = "16909060" (cached)
Property 13: bias.ocpi_debug = "false" (parameter)
Property 14: bias.ocpi_endian = "little" (parameter)
Property 15: bias.test64 = "0"
Property 16: file_write.fileName = "test.output" (cached)
Property 17: file_write.messagesInFile = "false" (cached)
Property 18: file_write.bytesWritten = "0"
Property 19: file_write.messagesWritten = "0"
Property 20: file_write.stopOnEOF = "true" (cached)
Property 21: file_write.ocpi_debug = "false" (parameter)
Property 22: file_write.ocpi_endian = "little" (parameter)
Property  3: file_read.messageSize = "16"
Property  5: file_read.repeat = "<unreadable>"
Property  6: file_read.bytesRead = "4000"
Property  7: file_read.messagesWritten = "251"
Property  8: file_read.suppressEOF = "false"
Property  9: file_read.badMessage = "false"
Property 15: bias.test64 = "0"
Property 18: file_write.bytesWritten = "4000"
Property 19: file_write.messagesWritten = "250"
\end{lstlisting}

\subsection*{Known Issues}
\subsubsection*{JTAG Daemon}
When loading FPGA bitstreams onto the alst4 FPGA (which can occur when running either \code{ocpihdl load} or \code{ocpirun}), multiple issues exists with the Altera jtag daemon which may cause the FPGA loading to fail. The following is an example of the terminal output when this failure occurs:
\begin{lstlisting}
Checking existing loaded bitstream on OpenCPI HDL device "PCI:0000:0b:00.0"...
OpenCPI FPGA at PCI 0000:0b:00.0: bitstream date Wed Oct 19 16:04:45 2016, platf
orm "alst4", part "ep4sgx230k", UUID 482195b4-9637-11e6-8002-d76b7b3cbb11
Existing loaded bitstream looks ok, proceeding to snapshot the PCI configuration
(into /tmp/ocpibitstream15980.1).
Scanning for JTAG cables...
Found cable "USB-Blaster [3-11]" to use for device "PCI:0000:0b:00.0" (no serial
number specified).
Error: did not find part ep4sgx230k in the jtag chain for cable USB-Blaster [3-1
1].
Look at /tmp/ocpibitstream15980.log for details.
Error: Could not find jtag position for part ep4sgx230k on JTAG cable "USB-Blast
er [3-11]".
OpenCPI FPGA at PCI 0000:0b:00.0: bitstream date Wed Oct 19 16:04:45 2016, platf
Exception thrown: Bitstream loading error (exit code: 1) loading "../../hdl/ass
emblies/dc offset iq imbalance mixer cic dec timestamper/container-dc offset iq imba
lance mixer cic dec timestamper alst4 base alst4 adc hsmc port b/target-stratix4/dc
offset iq imbalance mixer cic dec timestamper alst4 base alst4 adc hsmc port b.sof.gz
" on HDL device "PCI:0000:0b:00.0" with command: /opt/opencpi/cdk//scripts/loadBi
tStream "../../hdl/assemblies/dc offset iq imbalance mixer cic dec timestamper/conta
iner-dc offset iq imbalance mixer cic dec timestamper alst4 base alst4 adc hsmc port b
/target-stratix4/dc offset iq imbalance mixer cic dec timestamper alst4 base alst4 adc
hsmc port b.sof.gz" "PCI:0000:0b:00.0" "alst4" "ep4sgx230k" "" ""
\end{lstlisting}
The failure may also manifest as a permissions issue:
\begin{lstlisting}
Scanning for JTAG cables...
JTAG cable setup for platform "alst4" failed.
Dump of /tmp/ocpibitstream5904.cables:
********************************************************************************
Cable "USB-Blaster variant [3-7]": cannot get serial number.
********************************************************************************
Dump of /tmp/ocpibitstream5904.log:
********************************************************************************
Error when locking chain - Insufficient port permissions
********************************************************************************
\end{lstlisting}

\noindent The follow commands implement a known remedy for each of the aforementioned errors:
\begin{lstlisting}
sudo killall jtagd
sudo chmod 755 /sys/kernel/debug/usb/devices
sudo chmod 755 /sys/kernel/debug/usb
sudo chmod 755 /sys/kernel/debug
sudo mount --bind /dev/bus /proc/bus
sudo ln -s /sys/kernel/debug/usb/devices /proc/bus/usb/devices
sudo <quartus_directory>/bin/jtagd
sudo <quartus_directory>/bin/jtagconfig
\end{lstlisting}

\subsubsection*{Single Port of Data from CPU to FPGA} % AV-3783
\label{bug:3783}
The current implementations of the PCI-e specification on this platform correctly implements data flow from the CPU to the FPGA, only under certain configurations (assembly/container) and is limited to only a single port of data from CPU to FPGA. Fundamentally, OpenCPI only supports a single port connection from the CPU to the FPGA. \\ \\
To ensure the proper configurations are met, assembly and container XML files must be designed based on the following rules:
\begin{enumerate}
\item When a single worker exists in an assembly and it ports are connected to the assembly (Externals='true'), then
the container must be built for the "base" container. (i.e. the assembly's Makefile must contain "DefaultContainer=").
\item When an assembly's external connections are explicitly defined (i.e. not using Externals='true'), then the first external assembly connection that is defined in the assembly XML must be that of the CPU to FPGA, and the "base" container used (i.e. the assembly's Makefile must contain "DefaultContainer="), or
\item When defining external connections in a container XML, then the first interconnect container connection defined must be that of the CPU to FPGA and the "base" container used (i.e. the assembly's Makefile must contain "DefaultContainer=").
\end{enumerate}
Note that this applies to the TX/DAC data path connections for bitstreams with transceiver transmit data flow from a CPU (e.g. RCC worker to FPGA TX/DAC data path). See \path{projects/assets/hdl/assemblies/empty/cnt_1rx_1tx_bypassasm_fmcomms_2_3_lpc_LVDS_ml605.xml} as an example.

\pagebreak
  \begin{thebibliography}{1}


  \bibitem{alst4_hardware_setup}
	 \githubio[ALST4 Hardware Setup]{assets/Alst4\_Hardware\_Setup.pdf}
  \bibitem{fpga_vendor_tool_guide}
	 \githubio[FPGA Vendor Tools Guide]{FPGA\_Vendor\_Tools\_Installation\_Guide.pdf}
	   \bibitem{rpm_installation_guide}
	 \githubio[OpenCPI Installation Guide]{OpenCPI\_Installation\_Guide.pdf}
	   \bibitem{zipper_mods}
	 \githubio[Required Modifications for Myriad-RF 1 and Zipper Daughtercards]{assets/Required\_Modifications\_for\_Myriad-RF\_1\_Zipper\_Daughtercards.pdf}

  \end{thebibliography}

\end{document}
