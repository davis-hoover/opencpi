\iffalse
This file is protected by Copyright. Please refer to the COPYRIGHT file
distributed with this source distribution.

This file is part of OpenCPI <http://www.opencpi.org>

OpenCPI is free software: you can redistribute it and/or modify it under the
terms of the GNU Lesser General Public License as published by the Free Software
Foundation, either version 3 of the License, or (at your option) any later
version.

OpenCPI is distributed in the hope that it will be useful, but WITHOUT ANY
WARRANTY; without even the implied warranty of MERCHANTABILITY or FITNESS FOR A
PARTICULAR PURPOSE. See the GNU Lesser General Public License for more details.

You should have received a copy of the GNU Lesser General Public License along
with this program. If not, see <http://www.gnu.org/licenses/>.
\fi

%----------------------------------------------------------------------------------------
% Required document specific properties
%----------------------------------------------------------------------------------------
\def\comp{matchstiq\_{}z1\_{}i2c}
\edef\ecomp{matchstiq_z1_i2c}
\def\Comp{Matchstiq-Z1 I2C}
\def\docTitle{\Comp{} Component Data Sheet}
\def\snippetpath{../../../../../../../../doc/av/tex/snippets}
%----------------------------------------------------------------------------------------
% Global latex header (this must be after document specific properties)
%----------------------------------------------------------------------------------------
\iffalse
This file is protected by Copyright. Please refer to the COPYRIGHT file
distributed with this source distribution.

This file is part of OpenCPI <http://www.opencpi.org>

OpenCPI is free software: you can redistribute it and/or modify it under the
terms of the GNU Lesser General Public License as published by the Free Software
Foundation, either version 3 of the License, or (at your option) any later
version.

OpenCPI is distributed in the hope that it will be useful, but WITHOUT ANY
WARRANTY; without even the implied warranty of MERCHANTABILITY or FITNESS FOR A
PARTICULAR PURPOSE. See the GNU Lesser General Public License for more details.

You should have received a copy of the GNU Lesser General Public License along
with this program. If not, see <http://www.gnu.org/licenses/>.
\fi

% Sets OpenCPI Version used throughout all the docs. This is updated by
% scripts/update-release.sh when a release is being made and must not
% be changed manually.
\def\ocpiversion{v2.2.0}

\documentclass{article}
\author{}  % Force author to be blank
\date{OpenCPI Release:\ \ \ocpiversion}  % Force date to be blank and override date with version
\title{OpenCPI\\\docTitle}  % docTitle must be defined before including this file
%----------------------------------------------------------------------------------------
% Paper size, orientation and margins
%----------------------------------------------------------------------------------------
\usepackage{geometry}
\geometry{
  letterpaper,  % paper type
  portrait,     % text direction
  left=.75in,   % left margin
  top=.75in,    % top margin
  right=.75in,  % right margin
  bottom=.75in  % bottom margin
}
%----------------------------------------------------------------------------------------
% Header/Footer
%----------------------------------------------------------------------------------------
\usepackage{fancyhdr} \pagestyle{fancy}  % required for fancy headers
\renewcommand{\headrulewidth}{0.5pt}
\renewcommand{\footrulewidth}{0.5pt}
\lhead{\small{\docTitle}}
\rhead{\small{OpenCPI}}
%----------------------------------------------------------------------------------------
% Various packages
%----------------------------------------------------------------------------------------
\usepackage{amsmath}
\usepackage[page,toc]{appendix}  % for appendix stuff
\usepackage{enumitem}
\usepackage{graphicx}   % for including pictures by file
\usepackage{hyperref}   % for linking urls and lists
\usepackage{listings}   % for coding language styles
\usepackage{pdflscape}  % for landscape view
\usepackage{pifont}     % for sideways table
\usepackage{ragged2e}   % for justify
\usepackage{rotating}   % for sideways table
\usepackage{scrextend}
\usepackage{setspace}
\usepackage{subfig}
\usepackage{textcomp}
\usepackage[dvipsnames,usenames]{xcolor}  % for color names see https://en.wikibooks.org/wiki/LaTeX/Colors
\usepackage{xstring}
\uchyph=0  % Never hyphenate acronyms like RCC
\renewcommand\_{\textunderscore\allowbreak}  % Allow words to break/newline on underscores
%----------------------------------------------------------------------------------------
% Table packages
%----------------------------------------------------------------------------------------
\usepackage[tableposition=top]{caption}
\usepackage{float}
\floatstyle{plaintop}
\usepackage{longtable}  % for long possibly multi-page tables
\usepackage{multicol}   % for more advanced table layout
\usepackage{multirow}   % for more advanced table layout
\usepackage{tabularx}   % c=center,l=left,r=right,X=fill
% These define tabularx columns "C" and "R" to match "X" but center/right aligned
\newcolumntype{C}{>{\centering\arraybackslash}X}
\newcolumntype{M}[1]{>{\centering\arraybackslash}m{#1}}
\newcolumntype{P}[1]{>{\centering\arraybackslash}p{#1}}
\newcolumntype{R}{>{\raggedleft\arraybackslash}X}
%----------------------------------------------------------------------------------------
% Block Diagram / FSM Drawings
%----------------------------------------------------------------------------------------
\usepackage{tikz}
\usetikzlibrary{arrows,decorations.markings,fit,positioning,shapes}
\usetikzlibrary{automata}  % used for the fsm
\usetikzlibrary{calc}      % for duplicating clients
\usepgfmodule{oo}          % to define a client box
%----------------------------------------------------------------------------------------
% Colors Used
%----------------------------------------------------------------------------------------
\usepackage{colortbl}
\definecolor{blue}{rgb}{.7,.8,.9}
\definecolor{ceruleanblue}{rgb}{0.16, 0.32, 0.75}
\definecolor{cyan}{rgb}{0.0,0.6,0.6}
\definecolor{darkgreen}{rgb}{0,0.6,0}
\definecolor{deepmagenta}{rgb}{0.8, 0.0, 0.8}
\definecolor{maroon}{rgb}{0.5,0,0}
%----------------------------------------------------------------------------------------
% Define where to hyphenate
%----------------------------------------------------------------------------------------
\hyphenation{Cent-OS}
\hyphenation{install-ation}
%----------------------------------------------------------------------------------------
% Define Commands & Rename Commands
%----------------------------------------------------------------------------------------
\newcommand{\code}[1]{\texttt{#1}}  % For inline code snippet or command line
\newcommand{\sref}[1]{Section~\ref{#1}}  % To quickly reference a section
\newcommand{\todo}[1]{\textcolor{red}{TODO: #1}\PackageWarning{TODO:}{#1}}  % To do notes
\renewcommand{\contentsname}{Table of Contents}
\renewcommand{\listfigurename}{List of Figures}
\renewcommand{\listtablename}{List of Tables}

% This gives a link to gitlab.io document. By default, it outputs the filename.
% You can optionally change the link, e.g.
% \githubio{FPGA\_Vendor\_Tools\_Installation\_Guide.pdf} vs.
% \githubio[\textit{FPGA Vendor Tools Installation Guide}]{FPGA\_Vendor\_Tools\_Installation\_Guide.pdf}
% or if you want the raw ugly URL to come out, \githubioURL{FPGA_Vendor_Tools_Installation_Guide.pdf}
\newcommand{\githubio}[2][]{% The default is for FIRST param!
\href{http://opencpi.gitlab.io/releases/\ocpiversion/docs/#2}{\ifthenelse{\equal{#1}{}}{\texttt{#2}}{#1}}}
\newcommand{\gitlabcom}[2][]{% The default is for FIRST param!
\href{http://gitlab.com/opencpi/#2}{\ifthenelse{\equal{#1}{}}{\texttt{#2}}{#1}}}
\newcommand{\githubioURL}[1]{\url{http://opencpi.gitlab.io/releases/\ocpiversion/docs/#1}}
% Lastly, if you want a SINGLE leading path stripped, e.g. assets/X.pdf => X.pdf:
\newcommand{\githubioFlat}[1]{%
\StrBehind{#1}{/}[\den]%
\href{http://opencpi.gitlab.io/releases/\ocpiversion/docs/#1}{\texttt{\den}}%
}
%----------------------------------------------------------------------------------------
% VHDL Coding Language Style
% modified from: http://latex-community.org/forum/viewtopic.php?f=44&t=22076
%----------------------------------------------------------------------------------------
\lstdefinelanguage{VHDL}
{
  basicstyle=\ttfamily\footnotesize,
  columns=fullflexible,keepspaces,  % https://tex.stackexchange.com/a/46695/87531
  keywordstyle=\color{ceruleanblue},
  commentstyle=\color{darkgreen},
  morekeywords={
    library, use, all, entity, is, port, in, out, end, architecture, of,
    begin, and, signal, when, if, else, process, end,
  },
  morecomment=[l]--
}
%----------------------------------------------------------------------------------------
% XML Coding Language Style
% modified from http://tex.stackexchange.com/questions/10255/xml-syntax-highlighting
%----------------------------------------------------------------------------------------
\lstdefinelanguage{XML}
{
  basicstyle=\ttfamily\footnotesize,
  columns=fullflexible,keepspaces,
  morestring=[s]{"}{"},
  morecomment=[s]{!--}{--},
  commentstyle=\color{darkgreen},
  moredelim=[s][\color{black}]{>}{<},
  moredelim=[s][\color{cyan}]{\ }{=},
  stringstyle=\color{maroon},
  identifierstyle=\color{ceruleanblue}
}
%----------------------------------------------------------------------------------------
% DIFF Coding Language Style
% modified from http://tex.stackexchange.com/questions/50176/highlighting-a-diff-file
%----------------------------------------------------------------------------------------
\lstdefinelanguage{diff}
{
  basicstyle=\ttfamily\footnotesize,
  columns=fullflexible,keepspaces,
  breaklines=true,                            % wrap text
  morecomment=[f][\color{ceruleanblue}]{@@},  % group identifier
  morecomment=[f][\color{red}]-,              % deleted lines
  morecomment=[f][\color{darkgreen}]+,        % added lines
  morecomment=[f][\color{deepmagenta}]{---},  % Diff header lines (must appear after +,-)
  morecomment=[f][\color{deepmagenta}]{+++},
}
%----------------------------------------------------------------------------------------
% Python Coding Language Style
%----------------------------------------------------------------------------------------
\lstdefinelanguage{python}
{
  basicstyle=\ttfamily\footnotesize,
  columns=fullflexible,keepspaces,
  keywordstyle=\color{ceruleanblue},
  commentstyle=\color{darkgreen},
  stringstyle=\color{orange},
  morekeywords={
    print, if, sys, len, from, import, as, open,close, def, main, for, else,
    write, read, range,
  },
  comment=[l]{\#}
}
%----------------------------------------------------------------------------------------
% Fontsize Notes in order from smallest to largest
%----------------------------------------------------------------------------------------
%    \tiny
%    \scriptsize
%    \footnotesize
%    \small
%    \normalsize
%    \large
%    \Large
%    \LARGE
%    \huge
%    \Huge

\graphicspath{{figures/}}
%----------------------------------------------------------------------------------------

\begin{document}
\maketitle
\thispagestyle{empty}
\newpage

\def\name{\comp}
\def\workertype{Device}
\def\version{\ocpiversion}
\def\releasedate{4/2019}
\def\componentlibrary{ocpi.assets.platforms.matchstiq\_{}z1.devices}
\def\workers{\comp{}.hdl}
\def\testedplatforms{Matchstiq-Z1(PL)}
\section*{Summary - \Comp}
\begin{tabular}{|c|M{13.5cm}|}
  \hline
  \rowcolor{blue}
   & \\
  \hline
  Name              & \comp             \\
  \hline
  Worker Type       & \workertype       \\
  \hline
  OpenCPI Release   & \ocpiversion      \\
  \hline
  Last Update       & \releasedate      \\
  \hline
  Component Library & \componentlibrary \\
  \hline
  Workers           & \workers          \\
  \hline
  Tested Platforms  & \testedplatforms  \\
  \hline
\end{tabular}


\section*{Worker Implementation Details}
The Matchstiq-Z1 I2C device worker uses the subdevice construct to implement the I2C bus for the Matchstiq-Z1 platform. Matchstiq-Z1 I2C supports 5 device workers:
\begin{enumerate}
	\item Si5338
	\item Matchstiq-Z1 AVR
	\item Pca9534
	\item Pca9535
	\item Tmp100
\end{enumerate}
Matchstiq-Z1 I2C uses the i2c primitive library which is based upon the OpenCores I2C controller. This revision of the device worker supports 8 bit and 16 bit I2C accesses.

\section*{Block Diagrams}
\subsection*{Top level}
\begin{figure}[ht]
	\centerline{\includegraphics[scale=0.4]{block_diagram}}
	\caption{I2C Connection Block Diagram}
	\label{fig:tb}
\end{figure}
\subsection*{State Machine}
\begin{figure}[ht]
	\centerline{\includegraphics[scale=0.6]{state_machine_diagram}}
	\caption{I2C OpenCores Controller State Machince}
	\label{fig:tb}
\end{figure}

\section*{Source Dependencies}
\subsection*{\comp.hdl}
\begin{itemize}
	\item assets/hdl/platforms/matchstiq\_z1/devices/\comp.hdl/\comp.vhd
	\item assets/hdl/primitives/i2c/i2c\_pkg.vhd
	      \begin{itemize}
	      	\item assets/hdl/primitives/i2c/i2c\_opencores\_ctrl.vhd
	      	\item assets/hdl/primitives/i2c/i2c\_master\_byte\_ctrl.v
	      	\item assets/hdl/primitives/i2c/i2c\_master\_bit\_ctrl.v
	      	\item assets/hdl/primitives/i2c/timescale.v
	      	\item assets/hdl/primitives/i2c/i2c\_master\_defines.v
	      \end{itemize}
	\item core/hdl/primitives/ocpi/raw\_arb.vhd
\end{itemize}

\begin{landscape}
\section*{Component Spec Properties}
	\begin{scriptsize}
		\begin{tabular}{|p{3cm}|p{1.5cm}|c|c|c|c|c|p{7cm}|}
			\hline
			\rowcolor{blue}
			Name                   & Type   & SequenceLength & ArrayDimensions & Accessibility       & Valid Range & Default & Usage                        \\
			\hline
			\verb+NUSERS_p+        & -      & -              & -               & Readable, Parameter & -           & 5       & Number of supported devices     \\
			\hline
			\verb+SLAVE_ADDRESS_p+ & UChar  & -              & \verb+NUSERS_p+ & Readable, Parameter & -           & -       & Array of I2C Slave Addresses \\
			\hline
			\verb+CLK_FREQ_p+       & Float & -              & -               & Readable, Parameter & -           & 100e6 & Input clock rate which is divided down to create I2C clock  \\
			\hline
		\end{tabular}
	\end{scriptsize}

	\section*{Worker Interfaces}
	\subsection*{\comp.hdl}
	\begin{scriptsize}
		\begin{tabular}{|M{2cm}|M{1.5cm}|c|c|M{6cm}|}
			\hline
			\rowcolor{blue}
			Type & Name & DataWidth & Advanced & Usage \\
			\hline
			RawProp
			& rprops
			& -
			& Count=\verb+NUSERS_p+ Optional=true
			& \begin{flushleft}Raw properties connections for master devices \newline Index 0: matchstiq\_z1\_avr \newline Index 1: si5338 \newline Index 2: tmp100 \newline Index 3: pca9534 \newline Index 4: pca9535 \end{flushleft}\\
			\hline
		\end{tabular}
	\end{scriptsize}

	\section*{Signals}
	\begin{scriptsize}
	\begin{tabular}{|c|c|c|c|p{2.6cm}|c|c|c|}
		\hline
		\rowcolor{blue}
		Name & Type  & Width & Description \\
		\hline
		SDA  & Inout & 1     & I2C Data    \\
		\hline
		SCL  & Inout & 1     & I2C Clock   \\
		\hline
	\end{tabular}
	\end{scriptsize}
\end{landscape}

\section*{Control Timing and Signals}
The Matchstiq-Z1 I2C HDL device worker uses the clock from the Control Plane and standard Control Plane signals.

\begin{landscape}
\section*{Worker Configuration Parameters}
\subsubsection*{\comp.hdl}
%\input{../../\ecomp.hdl/configurations.inc}
\section*{Performance and Resource Utilization}
\subsubsection*{\comp.hdl}
%\input{../../\ecomp.hdl/utilization.inc}
\end{landscape}
\section*{Test and Verification}
Testing of the Matchstiq-Z1 I2C device worker consists of a C++ test bench that use the Application Control Interface API to command the UUT.
% simulation section belongs in a differnt doc, leaving commented out so we dont lose content
% As the simulation (i2c_sim.test/) is generic, it was moved to hdl/devices/.
%\subsubsection*{Simulation}
%The simulation test bench is based on a simulation model of an I2C slave.
%Workers analagous to the Matchstiq-Z1 I2C component and the workers it supports were developed based on the simulation model.
%The slave has 4 8-bit registers and a user configurable slave address. The testbench performs a write and readback test for each type of worker it supports. The following shows the console output of a successful run of the testbench.\par\bigskip
%\noindent \texttt{App initialized.\\
%App started.\\
%App stopped.\\
%I2C Readback Test for 8 bit master: Passed.\\
%I2C Readback Test for 16 bit master: Passed.\\
%I2C Readback Test for 8 bit mixed master: Passed.\\
%I2C Readback Test for 16 bit mixed master: Passed.\\
%I2C Readback Test for 16 bit extended write master: Passed.\\
%I2C Sim Testbench: Passed.}\par\bigskip
%\noindent Should all tests pass a file i2c\_sim\_testbench.results is produced for use with automated testing.

\subsubsection*{Hardware}
\begin{flushleft}
The testbench for this worker checks the functionality of the I2C devices and generates an output file with the received input data. \\ \medskip

Building the test assembly requires that the \textit{matchstiq\_z1} platform has been built. Details on how to build the \textit{matchstiq\_z1} platform can be found in the Matchstiq-Z1 platform document. To build the testbench's assembly and ACI, follow the instructions that are provided by running \textbf{\textit{make show}} within this test's directory.\\ \medskip

Connect a signal generator to the input "RX" channel. Configured the signal generator to produce tone at a frequency of 2.140001 GHz and amplitude -55 dBm.\\ \medskip

Execute and validate the output of the test by continuing to follow the instructions provided by running \textbf{\textit{make show}} within this test's directory.\\ \medskip

An example of the terminal output is provide below:

\begin{lstlisting}
% ./target-xilinx13_3/testbench                 
Application XML used for testbench: ./hw_testbench_app_file.xml
Start of Testbench
Set Sampling Clock to 200 kHz (100 kSps): 
PCA9535: Starting Test 
PCA9535: Testing filter bandwidth: 
PCA9535: Set unfiltered
PCA9535: Set filter bandwidth to 300 to 700 MHz
PCA9535: Set filter bandwidth to 625 to 1080 MHz
PCA9535: Set filter bandwidth to 1000 to 2100 MHz
PCA9535: Set filter bandwidth to 1700 to 2500 MHz
PCA9535: Set filter bandwidth to 2200 to 3800 MHz
PCA9535: Set filter bandwidth to unfiltered
PCA9535: Testing Lime RX input:
PCA9535: Set Lime RX input to 2
PCA9535: Set Lime RX input to 3
PCA9535: Testing Pre-lime LNA:
PCA9535: Setting Pre-lime LNA off
PCA9535: Setting Pre-lime LNA on 
PCA9535: End of Test 
Matchstiq-Z1 AVR: Starting Test 
Matchstiq-Z1 AVR: Testing attenuator:
Matchstiq-Z1 AVR: Reset attenuator to 0:
Matchstiq-Z1 AVR: Testing LED:
Matchstiq-Z1 AVR: Set LED off
Matchstiq-Z1 AVR: Set LED green
Matchstiq-Z1 AVR: Set LED red
Matchstiq-Z1 AVR: Set LED orange
Matchstiq-Z1 AVR: Testing Serial Number:
Matchstiq-Z1 AVR: Serial number is: 6188
Matchstiq-Z1 AVR: Testing WARP voltage register: 
Matchstiq-Z1 AVR: Set WARP voltage to 2048
Matchstiq-Z1 AVR: End of Test 
TMP100: Starting Test 
TMP100: Testing temperature:
TMP100: Temperature is: 42 degrees C
TMP100: End of Test 
\end{lstlisting}

Additionally, an output file is produced odata/testbench\_rx.out which can be plotted. Figure 1 shows the expected result for the received data. These results should be inspected manually as the testbench does not verify these trends.\par\bigskip
	\begin{figure}[ht]
		\centerline{\includegraphics[scale=0.5]{testbench_rx}}
		\caption{Expected Results}
		\label{fig:tb}
	\end{figure}
\end{flushleft}

\section*{References}
\begin{flushleft}
	\begin{itemize}
		\item[1)] The Matchstiq-Z1 Software Development Manual (provided by Epiq with the Platform Development Kit)
	\end{itemize}
\end{flushleft}
\end{document}
