%TODO: Use LaTeX_Header
\documentclass{article}
\iffalse
This file is protected by Copyright. Please refer to the COPYRIGHT file
distributed with this source distribution.

This file is part of OpenCPI <http://www.opencpi.org>

OpenCPI is free software: you can redistribute it and/or modify it under the
terms of the GNU Lesser General Public License as published by the Free Software
Foundation, either version 3 of the License, or (at your option) any later
version.

OpenCPI is distributed in the hope that it will be useful, but WITHOUT ANY
WARRANTY; without even the implied warranty of MERCHANTABILITY or FITNESS FOR A
PARTICULAR PURPOSE. See the GNU Lesser General Public License for more details.

You should have received a copy of the GNU Lesser General Public License along
with this program. If not, see <http://www.gnu.org/licenses/>.
\fi

\author{} % Force author to be blank
%----------------------------------------------------------------------------------------
% Paper size, orientation and margins
%----------------------------------------------------------------------------------------
\usepackage{geometry}
\geometry{
	letterpaper,			% paper type
	portrait,				% text direction
	left=.75in,				% left margin
	top=.75in,				% top margin
	right=.75in,			% right margin
	bottom=.75in			% bottom margin
 }
%----------------------------------------------------------------------------------------
% Header/Footer
%----------------------------------------------------------------------------------------
\usepackage{fancyhdr} \pagestyle{fancy} % required for fancy headers
\renewcommand{\headrulewidth}{0.5pt}
\renewcommand{\footrulewidth}{0.5pt}
\newcommand{\terminaloutput}[1]{\texttt{#1}}
\rhead{\small{ANGRYVIPER Team}}
%----------------------------------------------------------------------------------------
% Appendix packages
%----------------------------------------------------------------------------------------
\usepackage[toc,page]{appendix}
%----------------------------------------------------------------------------------------
% Defined Commands & Renamed Commands
%----------------------------------------------------------------------------------------
\renewcommand{\contentsname}{Table of Contents}
\renewcommand{\listfigurename}{List of Figures}
\renewcommand{\listtablename}{List of Tables}
%----------------------------------------------------------------------------------------
% Various packages
%----------------------------------------------------------------------------------------
\usepackage{hyperref} % for linking urls and lists
\usepackage{graphicx} % for including pictures by file
\usepackage{listings} % for coding language styles
\usepackage{rotating} % for sideways table
\usepackage{pifont}   % for sideways table
\usepackage{pdflscape} % for landscape view
%----------------------------------------------------------------------------------------
% Table packages
%----------------------------------------------------------------------------------------
\usepackage{longtable} % for long possibly multi-page tables
\usepackage{tabularx} % c=center,l=left,r=right,X=fill
\usepackage{float}
\floatstyle{plaintop}
\usepackage[tableposition=top]{caption}
\newcolumntype{P}[1]{>{\centering\arraybackslash}p{#1}}
\newcolumntype{M}[1]{>{\centering\arraybackslash}m{#1}}
%----------------------------------------------------------------------------------------
% Block Diagram / FSM Drawings
%----------------------------------------------------------------------------------------
\usepackage{tikz}
\usetikzlibrary{shapes,arrows,fit,positioning}
\usetikzlibrary{automata} % used for the fsm
\usetikzlibrary{calc} % For duplicating clients
\usepgfmodule{oo} % To define a client box
%----------------------------------------------------------------------------------------
% Colors Used
%----------------------------------------------------------------------------------------
\usepackage{colortbl}
\definecolor{blue}{rgb}{.7,.8,.9}
\definecolor{ceruleanblue}{rgb}{0.16, 0.32, 0.75}
\definecolor{drkgreen}{rgb}{0,0.6,0}
\definecolor{deepmagenta}{rgb}{0.8, 0.0, 0.8}
\definecolor{cyan}{rgb}{0.0,0.6,0.6}
\definecolor{maroon}{rgb}{0.5,0,0}
\usepackage{multirow}
%----------------------------------------------------------------------------------------
% Update the docTitle and docVersion per document
%----------------------------------------------------------------------------------------
\def\docTitle{Component Data Sheet}
\def\docVersion{1.5}
%----------------------------------------------------------------------------------------
\date{Version \docVersion} % Force date to be blank and override date with version
\title{\docTitle}
\lhead{\small{\docTitle}}

% find and replace: dev signal, devsignal -> \devsignal{}
\def\devsignal{devsignal}
% find and replace: Dev Signal, Dev signal, dev Signal, DevSignal -> \DevSignal{}
\def\DevSignal{DevSignal}

\def\snippetpath{../../../../../../doc/av/tex/snippets}
% Usage:
% \def\snippetpath{../../../../../doc/av/tex/snippets/}
% % Usage:
% \def\snippetpath{../../../../../doc/av/tex/snippets/}
% % Usage:
% \def\snippetpath{../../../../../doc/av/tex/snippets/}
% \input{\snippetpath/includes}
% From then on, you can use "input" With no paths to get to "snippets"
% You also get all "major" snippets not part of the global LaTeX_Header
% NOTE: If not using the global LaTeX_Header, you need to
% \usepackage{ifthen} to use the \githubio macro

\hyphenation{ANGRY-VIPER} % Tell it where to hyphenate
\hyphenation{Cent-OS} % Tell it where to hyphenate
\hyphenation{install-ation} % Tell it where to hyphenate

\newcommand{\todo}[1]{\textcolor{red}{TODO: #1}\PackageWarning{TODO:}{#1}} % To do notes
\newcommand{\code}[1]{\texttt{#1}} % For inline code snippet or command line
\newcommand{\sref}[1]{Section~\ref{#1}} % To quickly reference a section

% To quickly reference a versioned PDF on gitlab.io
\def\ocpiversion{develop}

% This gives a link to gitlab.io document. By default, it puts the filename.
% You can optionally change the link, e.g.
% \githubio{FPGA\_Vendor\_Tools\_Installation\_Guide.pdf} vs.
% \githubio[\textit{FPGA Vendor Tools Installation Guide}]{FPGA\_Vendor\_Tools\_Installation\_Guide.pdf}
% or if you want the raw ugly URL to come out, \githubioURL{FPGA_Vendor_Tools_Installation_Guide.pdf}
\newcommand{\githubio}[2][]{% The default is for FIRST param!
\href{http://opencpi.gitlab.io/releases/\ocpiversion/docs/#2}{\ifthenelse{\equal{#1}{}}{\texttt{#2}}{#1}}}
\newcommand{\gitlabcom}[2][]{% The default is for FIRST param!
\href{http://gitlab.com/opencpi/#2}{\ifthenelse{\equal{#1}{}}{\texttt{#2}}{#1}}}
\newcommand{\githubioURL}[1]{\url{http://opencpi.gitlab.io/releases/\ocpiversion/docs/#1}}
% Lastly, if you want a SINGLE leading path stripped, e.g. assets/X.pdf => X.pdf:
\newcommand{\githubioFlat}[1]{%
\StrBehind{#1}{/}[\den]%
\href{http://opencpi.gitlab.io/releases/\ocpiversion/docs/#1}{\texttt{\den}}%
}

% Fix import paths
\makeatletter
\def\input@path{{\snippetpath/}}
\makeatother

% From then on, you can use "input" With no paths to get to "snippets"
% You also get all "major" snippets not part of the global LaTeX_Header
% NOTE: If not using the global LaTeX_Header, you need to
% \usepackage{ifthen} to use the \githubio macro

\hyphenation{ANGRY-VIPER} % Tell it where to hyphenate
\hyphenation{Cent-OS} % Tell it where to hyphenate
\hyphenation{install-ation} % Tell it where to hyphenate

\newcommand{\todo}[1]{\textcolor{red}{TODO: #1}\PackageWarning{TODO:}{#1}} % To do notes
\newcommand{\code}[1]{\texttt{#1}} % For inline code snippet or command line
\newcommand{\sref}[1]{Section~\ref{#1}} % To quickly reference a section

% To quickly reference a versioned PDF on gitlab.io
\def\ocpiversion{develop}

% This gives a link to gitlab.io document. By default, it puts the filename.
% You can optionally change the link, e.g.
% \githubio{FPGA\_Vendor\_Tools\_Installation\_Guide.pdf} vs.
% \githubio[\textit{FPGA Vendor Tools Installation Guide}]{FPGA\_Vendor\_Tools\_Installation\_Guide.pdf}
% or if you want the raw ugly URL to come out, \githubioURL{FPGA_Vendor_Tools_Installation_Guide.pdf}
\newcommand{\githubio}[2][]{% The default is for FIRST param!
\href{http://opencpi.gitlab.io/releases/\ocpiversion/docs/#2}{\ifthenelse{\equal{#1}{}}{\texttt{#2}}{#1}}}
\newcommand{\gitlabcom}[2][]{% The default is for FIRST param!
\href{http://gitlab.com/opencpi/#2}{\ifthenelse{\equal{#1}{}}{\texttt{#2}}{#1}}}
\newcommand{\githubioURL}[1]{\url{http://opencpi.gitlab.io/releases/\ocpiversion/docs/#1}}
% Lastly, if you want a SINGLE leading path stripped, e.g. assets/X.pdf => X.pdf:
\newcommand{\githubioFlat}[1]{%
\StrBehind{#1}{/}[\den]%
\href{http://opencpi.gitlab.io/releases/\ocpiversion/docs/#1}{\texttt{\den}}%
}

% Fix import paths
\makeatletter
\def\input@path{{\snippetpath/}}
\makeatother

% From then on, you can use "input" With no paths to get to "snippets"
% You also get all "major" snippets not part of the global LaTeX_Header
% NOTE: If not using the global LaTeX_Header, you need to
% \usepackage{ifthen} to use the \githubio macro

\hyphenation{ANGRY-VIPER} % Tell it where to hyphenate
\hyphenation{Cent-OS} % Tell it where to hyphenate
\hyphenation{install-ation} % Tell it where to hyphenate

\newcommand{\todo}[1]{\textcolor{red}{TODO: #1}\PackageWarning{TODO:}{#1}} % To do notes
\newcommand{\code}[1]{\texttt{#1}} % For inline code snippet or command line
\newcommand{\sref}[1]{Section~\ref{#1}} % To quickly reference a section

% To quickly reference a versioned PDF on gitlab.io
\def\ocpiversion{develop}

% This gives a link to gitlab.io document. By default, it puts the filename.
% You can optionally change the link, e.g.
% \githubio{FPGA\_Vendor\_Tools\_Installation\_Guide.pdf} vs.
% \githubio[\textit{FPGA Vendor Tools Installation Guide}]{FPGA\_Vendor\_Tools\_Installation\_Guide.pdf}
% or if you want the raw ugly URL to come out, \githubioURL{FPGA_Vendor_Tools_Installation_Guide.pdf}
\newcommand{\githubio}[2][]{% The default is for FIRST param!
\href{http://opencpi.gitlab.io/releases/\ocpiversion/docs/#2}{\ifthenelse{\equal{#1}{}}{\texttt{#2}}{#1}}}
\newcommand{\gitlabcom}[2][]{% The default is for FIRST param!
\href{http://gitlab.com/opencpi/#2}{\ifthenelse{\equal{#1}{}}{\texttt{#2}}{#1}}}
\newcommand{\githubioURL}[1]{\url{http://opencpi.gitlab.io/releases/\ocpiversion/docs/#1}}
% Lastly, if you want a SINGLE leading path stripped, e.g. assets/X.pdf => X.pdf:
\newcommand{\githubioFlat}[1]{%
\StrBehind{#1}{/}[\den]%
\href{http://opencpi.gitlab.io/releases/\ocpiversion/docs/#1}{\texttt{\den}}%
}

% Fix import paths
\makeatletter
\def\input@path{{\snippetpath/}}
\makeatother


\def\comp{ad9361\_dac}
\edef\ecomp{ad9361_adc}
\def\Comp{AD9361 DAC}
\graphicspath{ {figures/} }

\begin{document}

\section*{Summary - \Comp}
\begin{longtable}{|p{\dimexpr0.25\textwidth-2\tabcolsep\relax}
                  |p{\dimexpr0.75\textwidth-2\tabcolsep\relax}|}
	\hline
	\rowcolor{blue}
	                  &                  \\
	\hline
	Package Prefix    & ocpi.assets.devices     \\
	\hline
	Name              & \comp            \\
	\hline
	OpenCPI Release & v\docVersion ~ (released 4/2019) \\
	\hline
	Workers           & \comp.hdl        \\
	\hline
	Tested Platforms  &
\begin{itemize}
  \item Agilent Zedboard/Analog Devices FMCOMMS2 (Vivado only)
  \item Agilent Zedboard/Analog Devices FMCOMMS3 (Vivado only)
  \item x86/Xilinx ML605/Analog Devices FMCOMMS2 (FMC-LPC slot only)
  \item x86/Xilinx ML605/Analog Devices FMCOMMS3 (FMC-LPC slot only)
  \item Ettus E310 (Vivado only, application for testing exists in e310 project)
\end{itemize} \\
	\hline
\end{longtable}

	\begin{center}
	\textit{\textbf{Revision History}}
		\begin{table}[H]
		\label{table:revisions} % Add "[H]" to force placement of table
      \begin{longtable}{|p{\dimexpr0.15\textwidth-2\tabcolsep\relax}
                        |p{\dimexpr0.75\textwidth-2\tabcolsep\relax}
                        |p{\dimexpr0.1\textwidth-2\tabcolsep\relax}|}
			\hline
			\rowcolor{blue}
			\textbf{Revision} & \textbf{Description of Change} & \textbf{Date} \\
		    \hline
		    v1.3 & Initial release & 3/2018 \\
		    \hline
		    v1.3.1 & Version bump only & 4/2018 \\
		    \hline
        v1.4 &
          \begin{itemize}
            \item TX powerdown functionality added (added optional \verb+event_in+ port, \verb+dev_tx_event+ devsignal port, and \verb+min_num_cp_clks_per_txen_event+ property)
            \item \verb+IDATA_WIDTH_p+ parameter property added
            \item source dependency list updated (new sources added, paths now specified by project)
          \end{itemize} & 10/2018 \\
		    \hline
		    v1.5 & Version bump only & 4/2019 \\
		    \hline
		    v1.7 & Table of Worker Configurations removed & 5/2020 \\
			\hline
			\end{longtable}
		\end{table}
	\end{center}

\section*{Functionality}
	The \Comp{} device worker ingests a single TX channel's data to be sent to the AD9361 IC \cite{ad9361}. Up to two instances of this worker can be used to send multichannel TX data to an AD9361 in an independent, non-phase-coherent fashion. This worker also has a port which controls AD9361 transmitter power on/off.
\section*{Block Diagrams}
\subsection*{Top level}
\makeatletter
\newcommand{\gettikzxy}[3]{%
  \tikz@scan@one@point\pgfutil@firstofone#1\relax
  \edef#2{\the\pgf@x}%
  \edef#3{\the\pgf@y}%
}
\makeatother
\pgfooclass{clientbox}{ % This is the class clientbox
    \method clientbox() { % The clientbox
    }
    \method apply(#1,#2,#3,#4) { % Causes the clientbox to be shown at coordinate (#1,#2) and named #3
        \node[rectangle,draw=white,fill=white] at (#1,#2) (#3) {#4};
    }
}
\pgfoonew \myclient=new clientbox()
\begin{center}
  \begin{tikzpicture}[% List of styles applied to all, to override specify on a case-by-case
      every node/.style={
        align=center,      % use this so that the "\\" for line break works
				minimum size=1.5cm	% creates space above and below text in rectangle
      },
      every edge/.style={draw,thick}
    ]
		\node[rectangle,ultra thick,draw=black,fill=blue,minimum width=5 cm](R1){Parameter Properties: \\ \verb+fifo_depth+ \\ \verb+IDATA_WIDTH_p+ \\ \\ \\ \Comp \\};
		%\node[rectangle,draw=white,fill=white](R3)[below= of R1]{``dev\_dac'' devsignal port\\ DAC data bus (see AD9361\_DAC\_Sub.pdf)};
		%\node[rectangle,draw=white,fill=white](R4)[left= of R1]{``in'' StreamInterface \\ Complex signed samples (Q0.15 I, Q0.15 Q)};
		\node[rectangle,draw=white,fill=white](R5)[above= of R1]{Non-parameter Properties:\\\verb+underrun+ \\ \verb+min_num_cp_clks_per_txen_event+};
		\path[->]
		(R1)edge []	node [] {} (R5)
		(R5)edge []	node [] {} (R1)
    ;
    \gettikzxy{(R1)}{\rx}{\ry}
    \myclient.apply(\rx - 220,\ry + 20,C1, ``in'' StreamInterface \\ Complex signed samples \texttt{(}Q0.15 I, Q0.15 Q\texttt{)} );
    \path[<-]($(R1.west) + (-0 pt,20 pt)$) edge [] node [] {} (C1);
    \myclient.apply(\rx - 247,\ry - 20,C1, ``event\_in'' StreamInterface \\ \texttt{(}connection optional\texttt{)} txOn/txOff Zero-Length Messages );
    \path[<-]($(R1.west) + (-0 pt,-20 pt)$) edge [] node [] {} (C1);
    \myclient.apply(\rx - 55,\ry-80,C1, ``dev\_dac'' \\ dev signal port \texttt{(}see \\ AD9361\_DAC\_Sub.pdf\texttt{)} );
    \path[<->]($(R1.south) + (-55 pt,0)$) edge [] node [] {} (C1);
    \myclient.apply(\rx + 55,\ry-80,C1, ``dev\_tx\_event'' \\ dev signal port \texttt{(}see \\ AD9361\_DAC\_Sub.pdf\texttt{)} );
    \path[->]($(R1.south) + (55 pt,0)$) edge [] node [] {} (C1);


  \end{tikzpicture}
\end{center}
\pagebreak

\section*{Worker Implementation Details}
\subsection*{\comp.hdl}
\subsubsection*{DAC Data Flow (in port)}
The \comp{}.hdl worker ingests signed Q0.15 I/Q samples from its \verb+in+ port, rounds them to signed Q0.11 I/Q samples, then passes them through an asynchronous First-In-First-Out (FIFO) buffer on to the \verb+dev_dac+ \devsignal{} bus. The rounding is done in order to map to the AD9361's 12-bit I/Q DAC bus\cite{adi_ug570}. For more information on how ad9361\_dac\_sub.hdl handles the data from this worker's \verb+dev_dac+ port, see \cite{dac_sub_comp_datasheet}. The asynchronous FIFO is necessary in order to cross clock domains from control clock to \verb+dev_dac+'s dac\_clk clock. Note that the control clock rate is considered static but platform-specific and that the clock rate of dac\_clk is potentially runtime variable. The FIFO's depth in number of samples is determined at build-time by the \verb+fifo_depth+ parameter property. The \verb+in+ port's data width is also configurable via the \verb+IDATA_WIDTH_p+ parameter property, whose default value of 32 allows for a signed Q0.15 I/ Q0.15 Q input sample to be processed in a single control clock cycle. An \verb+underrun+ property indicates when invalid samples have been clocked in by the DAC due to the FIFO being empty.
\subsubsection*{AD9361 Transmitter Power Control (event\_in port)}
\noindent The \verb+event_in+ port provides a port message-based mechanism (in the control plane clock domain) for turning on/off the AD9361 transmitter. Connection of this port is optional. If the port is disconnected, the transmitter will be on for the duration of an application. \\ \\
\noindent The transmitter is powered \textit{on} when:
\begin{itemize}
  \item the AD9361 is being initialized (typical duration is 200 ms), or
  \item one or more \comp.hdl workers exist in the bitstream and
    \begin{itemize}
      \item all \comp.hdl workers have their \verb+event_in+ ports disconnected and any are in the operating state, or
      \item any \comp.hdl workers have their \verb+event_in+ ports connected and receive a txOn message (while in the operating state)
    \end{itemize}
\end{itemize}
The transmitter is powered \textit{off} when:
\begin{itemize}
  \item the AD9361 is not being initialized and
  \begin{itemize}
    \item no \comp.hdl workers exist in the bitstream, or
    \item one \comp.hdl worker exists in the bitstream and
      \begin{itemize}
        \item is not in the operating state, or
        \item is in operating state and has its \verb+event_in+ port connected but has not yet received a message on its \verb+event_in+ port,
        \item is in operating state and has its \verb+event_in+ port connected and receives a txOff message on its \verb+event_in+ port
      \end{itemize}
    \item two \comp.hdl workers exist in the bitstream and
      \begin{itemize}
        \item both workers are not in the operating state, or
        \item no workers which are in the operating state and have their \verb+event_in+ port connected have yet received a message on their \verb+event_in+ port
        \item both workers have received a txOff message on their \verb+event_in+ ports in succession (with no txOn messages in-between) while in the operating state
      \end{itemize}
  \end{itemize}
\end{itemize}
Note that the normal use case for utilizing both DAC channels (and thus using two \comp.hdl workers) is for MIMO applications. As such, the normal use case is to either have no \verb+even_in+ ports connected, causing the transmitter to default to on for the duration of an application, or to have all connected and to send the same txOn/txOff message to all \verb+event_in+ ports simultaneously. \\

\pagebreak
\noindent Note that \verb+event_in+ messages exist in the control plane clock domain but the AD9361 registers them in the AD9361 FB\_CLK domain, which may be slower than the control plane clock . The \verb+min_num_cp_clks_per_txen_event+ property enforces that tx events are properly synchronized to the AD9361 FB\_CLK without losing any events by ensuring \verb+event_in+ messages are property spaced out. This is done by applying backpressure to the \verb+event_in+ port after each message is received. Backpressure is applied for \verb+min_num_cp_clks_per_txen_event+ - 1 number of control plane clocks after each \verb+event_in+ message is received. See property description for more info on calculation of the \verb+min_num_cp_clks_per_txen_event+ value. This property's default value of 180 was calculated using the worst-case (lowest) AD9361 sampling rate, worst-case (highest) AD9361 TX FIR interpolation factor, and highest known control plane clock rate of 125 MHz. The AD9361 LVDS or CMOS single port full duplex DDR mode was used for the default calculation. This property's value can be lowered for specific sampling rates / control plane clock rates if desired. Note that if using with a control plane clock rate of greater than 125 MHz, the default value should be overridden with a higher value.

\pagebreak


\section*{Source Dependencies}
\subsection*{\comp.hdl}
\begin{itemize}
	\item core/hdl/primitives/util/util\_pkg.vhd
	\item core/hdl/primitives/util/zlm\_detector.vhd
	\item assets/hdl/devices/ad9361\_dac.hdl/ad9361\_dac.vhd
	\item assets/hdl/devices/ad9361\_dac.hdl/trunc\_round\_16\_to\_12\_signed.vhd
	\item assets/hdl/primitives/misc\_prims/event\_in\_to\_txen/src/event\_in\_to\_txen.vhd
	\item assets/hdl/primitives/misc\_prims/misc\_prims\_pkg.vhd
	\item assets/hdl/primitives/util/dac\_fifo.vhd
	\item assets/hdl/primitives/util/util\_pkg.vhd
	\item assets/hdl/primitives/util/sync\_status.vhd
	\item assets/hdl/primitives/bsv/imports/SyncFIFO.v
	\item assets/hdl/primitives/bsv/imports/SyncResetA.v
	\item assets/hdl/primitives/bsv/imports/SyncHandshake.v
	\item assets/hdl/primitives/bsv/bsv\_pkg.vhd
\end{itemize}
\begin{landscape}

	\section*{Component Spec Properties}
	\begin{scriptsize}
		\begin{tabular}{|p{3.75cm}|p{1.25cm}|p{2cm}|p{2.75cm}|p{1.5cm}|p{1.5cm}|p{1cm}|p{6.62cm}|}
			\hline
			\rowcolor{blue}
			Name               & Type & SequenceLength & ArrayDimensions & Accessibility      & Valid Range & Default & Usage                                                                               \\
			\hline
			\verb+underrun+    & Bool & -              & -               & Volatile, Writable    & Standard    & -       & Flag set when DAC tries to send a sample and the DAC FIFO is empty. Once high, this flag is not cleared (i.e. set low) until the property is written to again (the flag clears regardless of write value, i.e. writing true or false both result in a value of false).\\
			\hline
		\end{tabular}
	\end{scriptsize}

	\section*{Worker Properties}
	\subsection*{\comp.hdl}
	\begin{scriptsize}
		\begin{tabular}{|p{2cm}|p{4cm}|p{1cm}|p{2cm}|p{2cm}|p{2cm}|p{2cm}|p{1cm}|p{3.95cm}|}
			\hline
			\rowcolor{blue}
			Scope        & Name                 & Type & SequenceLength & ArrayDimensions & Accessibility & Valid Range        & Default & Usage                                                                                                                  \\
			\hline
			Property     & \verb+fifo_depth+    & ULong& -              & -               & Parameter     & Standard           & 64      & Depth in number of samples of the control-to-DAC clock domain crossing FIFO. \\
			\hline
			Property     & \verb+IDATA_WIDTH_p+ & UShort&-              & -               & Parameter     & Standard           & 32      & \\
			\hline
			Property     & \verb+min_num_cp_clks_per_txen_events+ & UShort & - & -        & Initial, Readable & Standard       & 180     &
After every ZLM received on the \verb+event_in+ port, backpressure will be held on that port for one less than the number of control plane clock cycles specified by this property. This is done in order to ensure tx events are properly synchronized to the AD9361 FB\_CLK without losing any events. Minimum required value is ceil(1.5 * control plane clock rate / AD9361 FB\_CLK rate [use lowest expected FB\_CLK rate for your scenario]).
\\
			\hline
		\end{tabular}
	\end{scriptsize}

	\section*{Component Ports}
	\begin{scriptsize}
		\begin{tabular}{|p{2cm}|p{1.5cm}|p{4cm}|p{1.5cm}|p{1.5cm}|p{10.75cm}|}
			\hline
			\rowcolor{blue}
			Name & Producer & Protocol           & Optional & Advanced & Usage                  \\
			\hline
			in   & False    & iqstream\_protocol & False     & -        & Complex signed samples (Q0.15 I, Q0.15 Q). \\
			\hline
			event\_in & False & tx\_event-prot   & False     & -        & TX on/off events. \\
			\hline
		\end{tabular}
	\end{scriptsize}

	\section*{Worker Interfaces}
	\subsection*{\comp.hdl}
	\begin{scriptsize}
		\begin{tabular}{|p{2cm}|p{1.5cm}|p{1.5cm}|p{1.5cm}|p{1.5cm}|p{13.23cm}|}
			\hline
			\rowcolor{blue}
			Type            & Name & DataWidth & Optional & Advanced & Usage                  \\
			\hline
			StreamInterface & in   & 32        & False    & -        & Complex signed samples (Q0.15 I, Q0.15 Q). This port ingests data and forces backpressure. Because both a ``pulling'' pressure from the DAC clock and potentially limited ``pushing pressure'' from this port exists, it is possible for a value to be clocked to the DAC while no new value was yet seen at the in port. This event is monitored via the \verb+underrun+ property.  \\
			\hline
			StreamInterface & event\_in  & -   & True     & -        & TX on/off events. \\
			\hline
		\end{tabular}
	\end{scriptsize} \\ \\
	\begin{scriptsize}
		\begin{tabular}{|p{1.75cm}|p{2.25cm}|p{1.25cm}|p{1.25cm}|p{1.25cm}|p{3cm}|p{1.4cm}|p{0.9cm}|p{6.88cm}|}
			\hline
			\rowcolor{blue}
			Type                       & Name                            & Count & Optional & Master                & Signal                & Direction                  & Width                    & Description                                                                                                                  \\
			\hline
			\multirow{15}{*}{\DevSignal{}} & \multirow{15}{*}{dev\_dac} & \multirow{15}{*}{1} & \multirow{15}{*}{False} & \multirow{15}{*}{True}  & present & Output&1& Value is hardcoded to logic 1 inside this worker. \\
			\cline{6-9}
			&             &        &     &      & dac\_clk     & Input     & 1      & Clock for dac\_ready, dac\_take, dac\_data\_I, and dac\_data\_Q. \\
			\cline{6-9}
			&             &        &     &      & dac\_ready   & Output    & 1      & Indicates that the dac\_data\_I and dac\_data\_Q are valid/ready to be latched on the next rising edge of dac\_clk. \\
			\cline{6-9}
			&             &        &     &      & dac\_take    & Input     & 1      & Indicates that dac\_data\_I and dac\_data\_Q were latched on the previous rising edge of dac\_clk. If in the previous clock cycle dac\_ready was 1, the values of dac\_data\_I and dac\_data\_Q should not be allowed to update with a new sample until dac\_take is 1. \\
			\cline{6-9}
			&             &        &     &      & dac\_data\_I & Output    & 12     & Signed Q0.11 I value of DAC sample corresponding to RX channel 1. \\
			\cline{6-9}
			&             &        &     &      & dac\_data\_Q & Output    & 12     & Signed Q0.11 Q value of DAC sample corresponding to RX channel 1. \\
			\hline
		\end{tabular}
	\end{scriptsize}
\end{landscape}

\section*{Control Timing and Signals}
\subsection*{Clock Domains}
The \Comp{} device worker contains two clock domains: the clock from the control plane, and the dac\_clk clock from the \devsignal{}. It is expected that the control plane clock is faster than the dac\_clk clock in order to prevent a FIFO underrun (monitored via the \verb+underrun+ property).
\subsection*{Latency}
The latency from the input port to the \devsignal{} data bus is both both non-deterministic and dynamic. Non-determinism exists as a result of the data flowing through an asynchronous FIFO with each side in a different clock domain. Runtime dynamism exists as a result of the AD9361 DATA\_CLK\_P clock, and therefore the dac\_clk clock rates, being runtime dynamic. The use of any FIFO, synchronous or asynchronous, between the input port and the \devsignal{} also creates runtime dynamism in latency.
\subsection*{Backpressure}
Backpressure is transferred from the \devsignal{}'s dac\_clk clock to the input port. The input port is expected to frequently experience backpressure in order to prevent a FIFO underrun. Backpressure is applied to the \verb+event_in+ port according to the \verb+min_num_cp_clks_per_txen_event+ property value.

\section*{Worker Configuration Parameters}
\subsubsection*{\comp.hdl}
%\input{../../\ecomp.hdl/configurations.inc}
\section*{Performance and Resource Utilization}
\subsubsection*{\comp.hdl}
Fmax refers to the maximum allowable clock rate for any registered signal paths within a given clock domain for an FPGA design. Fmax in the table below is specific only to this worker and represents the maximum possible Fmax for any OpenCPI bitstream built with this worker included. Note that the Fmax value for a given clock domain for the final bitstream is often worse than the Fmax specific to this worker, even if this worker is the only one included in the bitstream. \\ \\

%\input{../../\ecomp.hdl/utilization.inc}
\input{utilization_custom.inc}

\footnotetext[1]{\label{abc}These measurements were the result of a Vivado timing analysis which was different from the Vivado analysis performed by default for OpenCPI worker builds. For more info see Appendix \ref{appendix}}
\footnotetext[2]{\label{quartustiming}Quartus does not perform timing analysis at the OpenCPI worker build (i.e. synthesis) stage.}

\pagebreak
\begin{thebibliography}{1}

\bibitem{ad9361} AD9361 Datasheet and Product Info \\
\url{https://www.analog.com/en/products/ad9361.html}
\bibitem{adi_ug570} AD9361 Reference Manual UG-570\\
AD9361\_Reference\_Manual\_UG-570.pdf
\bibitem{vendor_tools_install} FPGA Vendor Tools Installation Guide \\
\githubioURL{FPGA_Vendor_Tools_Installation_Guide.pdf}
\bibitem{dac_sub_comp_datasheet} AD9361 DAC Sub Component Data Sheet \\
\githubioURL{assets/AD9361_DAC_Sub.pdf}

\end{thebibliography}
\pagebreak
\section{Appendix - Vivado Timing Analysis} \label{appendix}

The Vivado timing report that OpenCPI runs for device workers may erroneously report a max delay for a clocking path which should have been ignored. Custom Vivado tcl commands had to be run for this device worker to extract pertinent information from Vivado timing analysis. After building the worker, the following commands were run from the assets project directory (after the Vivado settings64.sh was sourced):
\lstset{language=bash, backgroundcolor=\color{lightgray}, columns=flexible, breaklines=true, prebreak=\textbackslash, basicstyle=\ttfamily, showstringspaces=false,upquote=true, aboveskip=\baselineskip, belowskip=\baselineskip}
\begin{lstlisting}
cd hdl/devices/
vivado -mode tcl
\end{lstlisting}
Then the following commands were run inside the Vivado tcl terminal:
\begin{lstlisting}
open_project ad9361_dac.hdl/target-zynq/ad9361_dac_rv.xpr
synth_design -part xc7z020clg484-1 -top ad9361_dac_rv -mode out_of_context
create_clock -name clk1 -period 0.001 [get_nets {ctl_in[Clk]}]
create_clock -name clk2 -period 0.001 [get_nets {dev_dac_in[dac_clk]}]
set_clock_groups -asynchronous -group [get_clocks clk1] -group [get_clocks clk2]
\end{lstlisting}
The Fmax for the control plane clock for this worker is computed as the maximum magnitude slack with a control plane clock of 1 ps plus 2 times the assumed 1 ps control plane clock period (5.372 ns + 0.002 ns = 5.374 ns, 1/5.374 ns = 186.08 MHz). The Fmax for the dac\_clk clock from the devsignal is computed as the maximum magnitude slack with dac\_clk of 1 ps plus 2 times the assumed 1 ps dac\_clk period (4.306 ns + 0.002 ns = 4.308 ns, 1/4.287 ns = 232.12 MHz). \\ \\
The following command is run to get control plane clock timing:
\begin{lstlisting}
report_timing -delay_type min_max -sort_by slack -input_pins -group clk1
\end{lstlisting}
The expected output of the command is as follows:
\fontsize{6}{12}\selectfont
\begin{lstlisting}
INFO: [Timing 38-35] Done setting XDC timing constraints.
INFO: [Timing 38-91] UpdateTimingParams: Speed grade: -1, Delay Type: min_max.
INFO: [Timing 38-191] Multithreading enabled for timing update using a maximum of 8 CPUs
WARNING: [Timing 38-242] The property HD.CLK_SRC of clock port "ctl_in[Clk]" is not set. In out-of-context mode, this prevents timing estimation for clock delay/skew
Resolution: Set the HD.CLK_SRC property of the out-of-context port to the location of the clock buffer instance in the top-level design
WARNING: [Timing 38-242] The property HD.CLK_SRC of clock port "dev_dac_in[dac_clk]" is not set. In out-of-context mode, this prevents timing estimation for clock delay/skew
Resolution: Set the HD.CLK_SRC property of the out-of-context port to the location of the clock buffer instance in the top-level design
INFO: [Timing 38-78] ReportTimingParams: -max_paths 1 -nworst 1 -delay_type min_max -sort_by slack.
Copyright 1986-2017 Xilinx, Inc. All Rights Reserved.
-------------------------------------------------------------------------------------------------
| Tool Version : Vivado v.2017.1 (lin64) Build 1846317 Fri Apr 14 18:54:47 MDT 2017
| Date         : Wed Oct  3 16:41:06 2018
| Host         : <removed> running 64-bit CentOS Linux release 7.5.1804 (Core)
| Command      : report_timing -delay_type min_max -sort_by slack -input_pins -group clk1
| Design       : ad9361_dac_rv
| Device       : 7z020-clg484
| Speed File   : -1  PRODUCTION 1.11 2014-09-11
-------------------------------------------------------------------------------------------------

Timing Report

Slack (VIOLATED) :        -5.372ns  (required time - arrival time)
  Source:                 IN_port/fifo/data0_reg_reg[13]/C
                            (rising edge-triggered cell FDRE clocked by clk1  {rise@0.000ns fall@0.001ns period=0.001ns})
  Destination:            worker/fifo/fifo/fifoMem_reg/DIADI[9]
                            (rising edge-triggered cell RAMB18E1 clocked by clk1  {rise@0.000ns fall@0.001ns period=0.001ns})
  Path Group:             clk1
  Path Type:              Setup (Max at Slow Process Corner)
  Requirement:            0.002ns  (clk1 rise@0.002ns - clk1 rise@0.000ns)
  Data Path Delay:        4.374ns  (logic 1.904ns (43.533%)  route 2.470ns (56.467%))
  Logic Levels:           5  (CARRY4=3 LUT4=1 LUT5=1)
  Clock Path Skew:        -0.049ns (DCD - SCD + CPR)
    Destination Clock Delay (DCD):    0.924ns = ( 0.926 - 0.002 )
    Source Clock Delay      (SCD):    0.973ns
    Clock Pessimism Removal (CPR):    0.000ns
  Clock Uncertainty:      0.035ns  ((TSJ^2 + TIJ^2)^1/2 + DJ) / 2 + PE
    Total System Jitter     (TSJ):    0.071ns
    Total Input Jitter      (TIJ):    0.000ns
    Discrete Jitter          (DJ):    0.000ns
    Phase Error              (PE):    0.000ns

    Location             Delay type                Incr(ns)  Path(ns)    Netlist Resource(s)
  -------------------------------------------------------------------    -------------------
                         (clock clk1 rise edge)       0.000     0.000 r
                                                      0.000     0.000 r  ctl_in[Clk] (IN)
                         net (fo=198, unset)          0.973     0.973    IN_port/fifo/ctl_in[Clk]
                         FDRE                                         r  IN_port/fifo/data0_reg_reg[13]/C
  -------------------------------------------------------------------    -------------------
                         FDRE (Prop_fdre_C_Q)         0.518     1.491 r  IN_port/fifo/data0_reg_reg[13]/Q
                         net (fo=3, unplaced)         0.759     2.250    IN_port/fifo/IN_data[5]
                                                                      r  IN_port/fifo/fifoMem_reg_i_36/I1
                         LUT4 (Prop_lut4_I1_O)        0.295     2.545 r  IN_port/fifo/fifoMem_reg_i_36/O
                         net (fo=1, unplaced)         0.902     3.447    IN_port/fifo/fifoMem_reg_i_36_n_0
                                                                      r  IN_port/fifo/fifoMem_reg_i_24/I1
                         LUT5 (Prop_lut5_I1_O)        0.124     3.571 r  IN_port/fifo/fifoMem_reg_i_24/O
                         net (fo=1, unplaced)         0.000     3.571    IN_port/fifo/fifoMem_reg_i_24_n_0
                                                                      r  IN_port/fifo/fifoMem_reg_i_5/S[0]
                         CARRY4 (Prop_carry4_S[0]_CO[3])
                                                      0.513     4.084 r  IN_port/fifo/fifoMem_reg_i_5/CO[3]
                         net (fo=1, unplaced)         0.009     4.093    IN_port/fifo/fifoMem_reg_i_5_n_0
                                                                      r  IN_port/fifo/fifoMem_reg_i_4/CI
                         CARRY4 (Prop_carry4_CI_CO[3])
                                                      0.117     4.210 r  IN_port/fifo/fifoMem_reg_i_4/CO[3]
                         net (fo=1, unplaced)         0.000     4.210    IN_port/fifo/fifoMem_reg_i_4_n_0
                                                                      r  IN_port/fifo/fifoMem_reg_i_3/CI
                         CARRY4 (Prop_carry4_CI_O[1])
                                                      0.337     4.547 r  IN_port/fifo/fifoMem_reg_i_3/O[1]
                         net (fo=1, unplaced)         0.800     5.347    worker/fifo/fifo/sD_IN[9]
                         RAMB18E1                                     r  worker/fifo/fifo/fifoMem_reg/DIADI[9]
  -------------------------------------------------------------------    -------------------

                         (clock clk1 rise edge)       0.002     0.002 r
                                                      0.000     0.002 r  ctl_in[Clk] (IN)
                         net (fo=198, unset)          0.924     0.926    worker/fifo/fifo/ctl_in[Clk]
                         RAMB18E1                                     r  worker/fifo/fifo/fifoMem_reg/CLKBWRCLK
                         clock pessimism              0.000     0.926
                         clock uncertainty           -0.035     0.891
                         RAMB18E1 (Setup_ramb18e1_CLKBWRCLK_DIADI[9])
                                                     -0.916    -0.025    worker/fifo/fifo/fifoMem_reg
  -------------------------------------------------------------------
                         required time                         -0.025
                         arrival time                          -5.347
  -------------------------------------------------------------------
                         slack                                 -5.372




report_timing: Time (s): cpu = 00:00:08 ; elapsed = 00:00:09 . Memory (MB): peak = 2095.184 ; gain = 497.547 ; free physical = 7704 ; free virtual = 54670
\end{lstlisting}
\fontsize{10}{12}\selectfont
The following command is run to get dev\_dac.dac\_clk timing:
\begin{lstlisting}
report_timing -delay_type min_max -sort_by slack -input_pins -group clk2
\end{lstlisting}
The expected output of the command is as follows:
\fontsize{6}{12}\selectfont
\begin{lstlisting}
INFO: [Timing 38-91] UpdateTimingParams: Speed grade: -1, Delay Type: min_max.
INFO: [Timing 38-191] Multithreading enabled for timing update using a maximum of 8 CPUs
INFO: [Timing 38-78] ReportTimingParams: -max_paths 1 -nworst 1 -delay_type min_max -sort_by slack.
Copyright 1986-2017 Xilinx, Inc. All Rights Reserved.
-------------------------------------------------------------------------------------------------
| Tool Version : Vivado v.2017.1 (lin64) Build 1846317 Fri Apr 14 18:54:47 MDT 2017
| Date         : Thu Oct  4 10:56:37 2018
| Host         : <removed> running 64-bit CentOS Linux release 7.5.1804 (Core)
| Command      : report_timing -delay_type min_max -sort_by slack -input_pins -group clk2
| Design       : ad9361_dac_rv
| Device       : 7z020-clg484
| Speed File   : -1  PRODUCTION 1.11 2014-09-11
-------------------------------------------------------------------------------------------------

Timing Report

Slack (VIOLATED) :        -4.306ns  (required time - arrival time)
  Source:                 worker/fifo/fifo/dEnqPtr_reg[0]/C
                            (rising edge-triggered cell FDCE clocked by clk2  {rise@0.000ns fall@0.001ns period=0.001ns})
  Destination:            worker/fifo/fifo/fifoMem_reg/ENARDEN
                            (rising edge-triggered cell RAMB18E1 clocked by clk2  {rise@0.000ns fall@0.001ns period=0.001ns})
  Path Group:             clk2
  Path Type:              Setup (Max at Slow Process Corner)
  Requirement:            0.002ns  (clk2 rise@0.002ns - clk2 rise@0.000ns)
  Data Path Delay:        3.781ns  (logic 1.061ns (28.063%)  route 2.720ns (71.937%))
  Logic Levels:           3  (LUT2=1 LUT6=2)
  Clock Path Skew:        -0.049ns (DCD - SCD + CPR)
    Destination Clock Delay (DCD):    0.924ns = ( 0.926 - 0.002 )
    Source Clock Delay      (SCD):    0.973ns
    Clock Pessimism Removal (CPR):    0.000ns
  Clock Uncertainty:      0.035ns  ((TSJ^2 + TIJ^2)^1/2 + DJ) / 2 + PE
    Total System Jitter     (TSJ):    0.071ns
    Total Input Jitter      (TIJ):    0.000ns
    Discrete Jitter          (DJ):    0.000ns
    Phase Error              (PE):    0.000ns

    Location             Delay type                Incr(ns)  Path(ns)    Netlist Resource(s)
  -------------------------------------------------------------------    -------------------
                         (clock clk2 rise edge)       0.000     0.000 r
                                                      0.000     0.000 r  dev_dac_in[dac_clk] (IN)
                         net (fo=35, unset)           0.973     0.973    worker/fifo/fifo/dev_dac_in[dac_clk]
                         FDCE                                         r  worker/fifo/fifo/dEnqPtr_reg[0]/C
  -------------------------------------------------------------------    -------------------
                         FDCE (Prop_fdce_C_Q)         0.518     1.491 r  worker/fifo/fifo/dEnqPtr_reg[0]/Q
                         net (fo=1, unplaced)         0.965     2.456    worker/fifo/fifo/dEnqPtr[0]
                                                                      r  worker/fifo/fifo/dGDeqPtr_rep[0]_i_3/I0
                         LUT6 (Prop_lut6_I0_O)        0.295     2.751 r  worker/fifo/fifo/dGDeqPtr_rep[0]_i_3/O
                         net (fo=1, unplaced)         0.449     3.200    worker/fifo/fifo/dGDeqPtr_rep[0]_i_3_n_0
                                                                      r  worker/fifo/fifo/dGDeqPtr_rep[0]_i_1/I1
                         LUT6 (Prop_lut6_I1_O)        0.124     3.324 r  worker/fifo/fifo/dGDeqPtr_rep[0]_i_1/O
                         net (fo=18, unplaced)        0.506     3.830    worker/fifo/fifo/dGDeqPtr0
                                                                      r  worker/fifo/fifo/fifoMem_reg_i_1/I0
                         LUT2 (Prop_lut2_I0_O)        0.124     3.954 r  worker/fifo/fifo/fifoMem_reg_i_1/O
                         net (fo=1, unplaced)         0.800     4.754    worker/fifo/fifo/fifoMem_reg_i_1_n_0
                         RAMB18E1                                     r  worker/fifo/fifo/fifoMem_reg/ENARDEN
  -------------------------------------------------------------------    -------------------

                         (clock clk2 rise edge)       0.002     0.002 r
                                                      0.000     0.002 r  dev_dac_in[dac_clk] (IN)
                         net (fo=35, unset)           0.924     0.926    worker/fifo/fifo/dev_dac_in[dac_clk]
                         RAMB18E1                                     r  worker/fifo/fifo/fifoMem_reg/CLKARDCLK
                         clock pessimism              0.000     0.926
                         clock uncertainty           -0.035     0.891
                         RAMB18E1 (Setup_ramb18e1_CLKARDCLK_ENARDEN)
                                                     -0.443     0.448    worker/fifo/fifo/fifoMem_reg
  -------------------------------------------------------------------
                         required time                          0.448
                         arrival time                          -4.754
  -------------------------------------------------------------------
                         slack                                 -4.306




\end{lstlisting}

\end{document}
