\documentclass{article}
\iffalse
This file is protected by Copyright. Please refer to the COPYRIGHT file
distributed with this source distribution.

This file is part of OpenCPI <http://www.opencpi.org>

OpenCPI is free software: you can redistribute it and/or modify it under the
terms of the GNU Lesser General Public License as published by the Free Software
Foundation, either version 3 of the License, or (at your option) any later
version.

OpenCPI is distributed in the hope that it will be useful, but WITHOUT ANY
WARRANTY; without even the implied warranty of MERCHANTABILITY or FITNESS FOR A
PARTICULAR PURPOSE. See the GNU Lesser General Public License for more details.

You should have received a copy of the GNU Lesser General Public License along
with this program. If not, see <http://www.gnu.org/licenses/>.
\fi

\author{} % Force author to be blank
%----------------------------------------------------------------------------------------
% Paper size, orientation and margins
%----------------------------------------------------------------------------------------
\usepackage{geometry}
\geometry{
	letterpaper,			% paper type
	portrait,				% text direction
	left=.75in,				% left margin
	top=.75in,				% top margin
	right=.75in,			% right margin
	bottom=.75in			% bottom margin
 }
%----------------------------------------------------------------------------------------
% Header/Footer
%----------------------------------------------------------------------------------------
\usepackage{fancyhdr} \pagestyle{fancy} % required for fancy headers
\renewcommand{\headrulewidth}{0.5pt}
\renewcommand{\footrulewidth}{0.5pt}
\rhead{\small{ANGRYVIPER Team}}
%----------------------------------------------------------------------------------------
% Appendix packages
%----------------------------------------------------------------------------------------
\usepackage[toc,page]{appendix}
%----------------------------------------------------------------------------------------
% Defined Commands & Renamed Commands
%----------------------------------------------------------------------------------------
\renewcommand{\contentsname}{Table of Contents}
\renewcommand{\listfigurename}{List of Figures}
\renewcommand{\listtablename}{List of Tables}
\newcommand{\todo}[1]{\textcolor{red}{TODO: #1}\PackageWarning{TODO:}{#1}} % To do notes
\newcommand{\code}[1]{\texttt{#1}} % For inline code snippet or command line
%----------------------------------------------------------------------------------------
% Various packages
%----------------------------------------------------------------------------------------
\usepackage{hyperref} % for linking urls and lists
\usepackage{graphicx} % for including pictures by file
\usepackage{listings} % for coding language styles
\usepackage{rotating} % for sideways table
\usepackage{pifont}   % for sideways table
\usepackage{pdflscape} % for landscape view
%----------------------------------------------------------------------------------------
% Table packages
%----------------------------------------------------------------------------------------
\usepackage{longtable} % for long possibly multi-page tables
\usepackage{tabularx} % c=center,l=left,r=right,X=fill
\usepackage{float}
\floatstyle{plaintop}
\usepackage[tableposition=top]{caption}
\newcolumntype{P}[1]{>{\centering\arraybackslash}p{#1}}
\newcolumntype{M}[1]{>{\centering\arraybackslash}m{#1}}
%----------------------------------------------------------------------------------------
% Block Diagram / FSM Drawings
%----------------------------------------------------------------------------------------
\usepackage{tikz}
\usetikzlibrary{shapes,arrows,fit,positioning}
\usetikzlibrary{automata} % used for the fsm
\usetikzlibrary{calc} % For duplicating clients
\usepgfmodule{oo} % To define a client box
%----------------------------------------------------------------------------------------
% Colors Used
%----------------------------------------------------------------------------------------
\usepackage{colortbl}
\definecolor{blue}{rgb}{.7,.8,.9}
\definecolor{ceruleanblue}{rgb}{0.16, 0.32, 0.75}
\definecolor{drkgreen}{rgb}{0,0.6,0}
\definecolor{deepmagenta}{rgb}{0.8, 0.0, 0.8}
\definecolor{cyan}{rgb}{0.0,0.6,0.6}
\definecolor{maroon}{rgb}{0.5,0,0}
\usepackage{multirow}
%----------------------------------------------------------------------------------------
% Update the docTitle and docVersion per document
%----------------------------------------------------------------------------------------
\def\docTitle{Component Data Sheet}
\def\docVersion{1.6}
%----------------------------------------------------------------------------------------
\date{Version \docVersion} % Force date to be blank and override date with version
\title{\docTitle}
\lhead{\small{\docTitle}}

\graphicspath{ {figures/} }

\def\comp{GPI}
\edef\ecomp{gpi}


\begin{document}

\section*{Summary - GPI}
	\begin{tabular}{|c|M{13.5cm}|}
		\hline
		\rowcolor{blue}
		• & • \\
		\hline
		Name & gpi \\
		\hline
		Version & v\docVersion \\
		\hline
		Release Date & 11/2019 \\
		\hline
		Component Library & ocpi.assets.devices \\
		\hline
		Workers & gpi.hdl \\
		\hline
		Tested Platforms & xsim, matchstiq\_z1 \\
		\hline
	\end{tabular}
	
\begin{center}
	\textit{\textbf{Revision History}}
		\begin{table}[H]
		\label{table:revisions} % Add "[H]" to force placement of table
			\begin{tabularx}{\textwidth}{|c|X|l|}
			\hline
			\rowcolor{blue}
			\textbf{Revision} & \textbf{Description of Change} & \textbf{Date} \\
		    \hline
		    v1.6 & Initial Release & 11/2019 \\
		    \hline
			\end{tabularx}
		\end{table}
	\end{center}	
	
\section*{Functionality}
\begin{flushleft}
The GPI device worker reads the GPIO pins present on an embedded platform. It is configured for general purpose input.

\end{flushleft}

\section*{Worker Implementation Details}
\begin{flushleft}

The GPI device worker reads GPIO pins and stores the values in the property \texttt{mask\_data} and send the values to its (optional) output port. \newline

It's (optional) output port uses a protocol that has one opcode and contains data and a mask. It sends data regardless of whether or not the worker downstream is ready, so it is something that cannot be piped directly to software. See \texttt{gpio\_protocol.xml} in \path{ocpi.core/specs/gpio_protocol.xml}  for further details on the protocol. \newline

For the \texttt{mask\_data} property and the output port data, the MSW is a mask and the LSW are the values read from the GPIO pins. \newline

The mask allows knowledge of which GPIO pins changed since the previous read cycle. The mask is the current data XOR the previous data. \newline

There are parameter properties that enable/disable build-time inclusion of 3 different circuits.  \newline

The \texttt{USE\_DEBOUNCE} property enables/disables build-time inclusion of a debounce circuit which debounces input from mechanical switches. 
There are the \texttt{CLK\_RATE\_HZ} and \texttt{DEBOUNCE\_TIME\_SEC} properties used by the GPI device worker to calculate the width of the debounce circuit's counter. The counter width is calculated by using the formula: $\mathrm{ceil(log2(CLK\_RATE\_HZ * DEBOUNCE\_TIME\_SEC))}$. 

It takes $2^{\mathrm{counter width}}$ + 3 clock cycles until the input is ready at the debounce circuit's output. If the debounce circuit is not enabled, metastability flip flops are used instead. These metastability flip flops have a latency of two clock cycles. \newline

The \texttt{EDGE\_MODE} property enables/disables build-time inclusion of an edge detector circuit. The edge detector detects rising and falling edge of an input
on the rising edge of the clock. There is a \texttt{RISING} property that is used to select between using the rising edge output of the edge detector or the falling edge. If \texttt{RISING} is true the rising edge is used otherwise the falling edge is used. The outputs of the edge detector have a latency of one clock cycle. The output of either the debounce circuit or metastability flip flops is fed to the edge detector. If the edge detector is not used, the output of debounce circuit or metastability flip flops is used instead. \newline

The \texttt{USE\_TOGGLE} property enables/disables build-time inclusion of a toggle circuit. The toggle circuit has a register that stores the result of XOR'ing the output of the edge detector, debounce circuit, or metastability flip flops with the current value of the register. If toggle circuit is not used the output of the edge detector, debounce circuit, or metastability flops is used instead.   \newline

The output of these chain of circuits is fed to device worker's output port. \newline

There is a parameter property called \texttt{EVENT\_MODE}. If \texttt{EVENT\_MODE} is set to true the output port's valid signal is driven high only when there is a change in the data to be sent to the output port. If it is set to false, the output port valid signal is always high and the data is valid every clock cycle. 

\end{flushleft}


\section*{Block Diagrams}
	\subsection*{Top level}
	\makeatletter
\newcommand{\gettikzxy}[3]{%
  \tikz@scan@one@point\pgfutil@firstofone#1\relax
  \edef#2{\the\pgf@x}%
  \edef#3{\the\pgf@y}%
}
\makeatother
\pgfooclass{clientbox}{ % This is the class clientbox
    \method clientbox() { % The clientbox
    }
    \method apply(#1,#2,#3,#4) { % Causes the clientbox to be shown at coordinate (#1,#2) and named #3
        \node[rectangle,draw=white,fill=white] at (#1,#2) (#3) {#4};
    }
}
\pgfoonew \myclient=new clientbox()
\begin{center}
\begin{tikzpicture}[% List of styles applied to all, to override specify on a case-by-case
					every node/.style={
						align=center,  		% use this so that the "\\" for line break works
						minimum size=1.5cm	% creates space above and below text in rectangle
						},
					every edge/.style={draw,thick}
					]
\node[rectangle,ultra thick,draw=black,fill=blue,minimum width=10 cm](R1){ Parameter Properties: \\ See property table \\ \\ gpi \\};
		\node[rectangle,draw=white,fill=white](R2)[above= of R1]{Non-parameter Properties:\\ mask\_data};
		\path[->]
		(R2)edge []	node [] {} (R1)
		(R1)edge []	node [] {} (R2)
		;					
		
    \gettikzxy{(R1)}{\rx}{\ry}
    \myclient.apply(\rx + 270,\ry + 20,C1, ``out'' StreamInterface \\ \texttt{(}connection optional\texttt{)} \\ mask and data \texttt{(}MSW: mask, LSW: data\texttt{)});
    \path[->]($(R1.east) + (0 pt,+20 pt)$) edge [] node [] {} (C1);
\myclient.apply(\rx + 0,\ry-80,C1, Signals \\ see signals table);
    \path[->]($(R1.south) + (0 pt,0 pt)$) edge [] node [] {} (C1);

\end{tikzpicture}
\end{center}

\section*{Source Dependencies}
\subsection*{gpi.hdl}
	\begin{itemize}
		\item assets/hdl/devices/gpi.hdl/gpi.vhd
		\item assets/primitives/misc\_prims/edge\_detector/src/edge\_detector.vhd
		\item assets/primitives/misc\_prims/debounce/src/debounce.vhd
	\end{itemize}

\subsection*{gpi\_em.hdl}
	\begin{itemize}
		\item assets/hdl/devices/gpi\_em.hdl/gpi\_em.vhd
		\item assets/primitives/util\_prims/set\_clr/src/set\_clr.vhd
	\end{itemize}
	
\begin{landscape}
\section*{Worker Properties}

\begin{flushleft}

        \begin{scriptsize}
        \begin{tabular}{|p{2.2cm}|p{1cm}|p{1cm}|p{2cm}|p{1.5cm}|p{2cm}|p{1.5cm}|p{7cm}|}
                \hline
                \rowcolor{blue}
                Name & Type & Default & SequenceLength & ArrayLength & ArrayDimensions & Accessibility & Usage \\
                \hline
                \verb+NUM_INPUTS+ & uChar & 1 & - & - & - & Parameter & Number of GP Inputs. The max number of GP Inputs supported is 16. \\
                \hline
                \verb+USE_DEBOUNCE+ & bool & false & - & - & - & Parameter & Enable/Disable build-time inclusion of debounce circuit. \\
                \hline
              \verb+CLK_RATE_HZ+ & double & 100e6 & - & - & - & Parameter & The clock rate of the clock feeding the debounce circuit in Hz. \\
                \hline
                \verb+DEBOUNCE_TIME_PSEC+ & double & 1e10 & - & - & - & Parameter & The desired debounce time for the debounce circuit in picoseconds. \\
                \hline
                \verb+EDGE_MODE+ & bool & false & - & - & - & Parameter & Enable/Disable build-time inclusion of edge detector circuit. \\
                \hline
                \verb+RISING+ & bool & true & - & - & - & Parameter & 
                True - Selects the edge detector's rising edge output. \newline
                False - Selects the edge detector's falling edge output. \\
                \hline
                \verb+USE_TOGGLE+ & bool & false & - & - & - & Parameter & Enable/Disable build-time inclusion of toggle circuit. \\
                \hline
                \verb+EVENT_MODE+ & bool & false & - & - & - & Parameter & 
                True - Output port data valid only when there is change in the data to be sent the output port. \newline
                False - Output port data valid every clock cycle. \\
                \hline
                \verb+mask_data+ & uLong & 0 & - & - & - & Volatile & Bitfield
    containing the data read from the GPIO pins and the mask. The mask allows
    knowledge of which GPIO pins changed since the previous read cycle. The MSW
    must be the mask and LSW must be the data. The mask is the current data XOR the previous data. \\
                \hline
        \end{tabular}
        \end{scriptsize}
 

\end{flushleft}

 
\section*{Component Ports}

        \begin{scriptsize}
                \begin{tabular}{|M{2.5cm}|M{2cm}|M{2cm}|M{3cm}|M{11cm}|}
                        \hline
                        \rowcolor{blue}
                        Name & Protocol & Producer & Optional & Usage\\
                        \hline
                        out
                        & gpio\_protocol
                        & true
                        & true
                        & Mask and the data read from GPIO pins \\
                        \hline
                \end{tabular}
			\end{scriptsize}
			
\section*{Worker Interfaces}
\subsection*{gpi.hdl}
\begin{scriptsize}
\begin{tabular}{|M{5cm}|M{4cm}|M{4cm}|c|M{6.5cm}|M{6cm}|}
            \hline
            \rowcolor{blue}
            Type    & Name & DataWidth (b) & Advanced  & Usage     \\
            \hline
            StreamInterface & out   & 32  & - & Mask and the data read from GPIO pins \\
           \hline

\end{tabular}
\end{scriptsize} \\

	
\section*{Signals}
	\begin{scriptsize}
		\begin{tabular}{|M{2.5cm}|M{3cm}|M{6.5cm}|M{9.2cm}|}
			\hline
			\rowcolor{blue}
			Name         & Type   & Width (b) & Description                       \\
			\hline
			gpi\_pin    & out & NUM\_INPUTS & Input from GPIO pins            \\
			\hline
		\end{tabular}
	\end{scriptsize}
\end{landscape}

\section*{Control Timing and Signals}
\begin{flushleft}

The GPI device worker uses the clock from the Control Plane and standard Control Plane signals.

\end{flushleft}

\begin{landscape}
\section*{Worker Configuration Parameters}
\subsubsection*{\comp.hdl}

\section*{Performance and Resource Utilization}
\subsubsection*{\comp.hdl}

\end{landscape}



\section*{Test and Verification}
\normalsize

\begin{flushleft}

The gpi device worker unit test has five test cases and utilizes the gpi\_em device emulator. The first case tests the gpi device worker by toggling GPIO pins on and then off with the debounce circuit, edge detector circuit, and toggle circuit disabled. Test case two toggles the pins on and off with only the debounce circuit enabled. The third test case toggles the pins on and off with only the edge detector circuit enabled. Test case four toggles the pins on, off, on, and then off with only the toggle circuit enabled. The fifth test case tests toggling the GPIO pins on, off, and then on with the debounce circuit, edge detector circuit, and toggle circuit enabled. \newline

The gpi\_em device emulator uses the data received from a data source to write the GPIO pins. \newline

The \path{generate.py} script generates the input data for the gpi\_em device emulator and the \path{verify.py} script checks the output file for expected length and
data contents.

\end{flushleft}
\end{document}
