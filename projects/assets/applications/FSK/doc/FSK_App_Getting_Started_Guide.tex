\iffalse
This file is protected by Copyright. Please refer to the COPYRIGHT file
distributed with this source distribution.

This file is part of OpenCPI <http://www.opencpi.org>

OpenCPI is free software: you can redistribute it and/or modify it under the
terms of the GNU Lesser General Public License as published by the Free Software
Foundation, either version 3 of the License, or (at your option) any later
version.

OpenCPI is distributed in the hope that it will be useful, but WITHOUT ANY
WARRANTY; without even the implied warranty of MERCHANTABILITY or FITNESS FOR A
PARTICULAR PURPOSE. See the GNU Lesser General Public License for more details.

You should have received a copy of the GNU Lesser General Public License along
with this program. If not, see <http://www.gnu.org/licenses/>.
\fi

%----------------------------------------------------------------------------------------
% Required document specific properties
%----------------------------------------------------------------------------------------
\def\docTitle{FSK App Getting Started Guide}
\def\snippetpath{../../../../../doc/av/tex/snippets}
%----------------------------------------------------------------------------------------
% Global latex header (this must be after document specific properties)
%----------------------------------------------------------------------------------------
\iffalse
This file is protected by Copyright. Please refer to the COPYRIGHT file
distributed with this source distribution.

This file is part of OpenCPI <http://www.opencpi.org>

OpenCPI is free software: you can redistribute it and/or modify it under the
terms of the GNU Lesser General Public License as published by the Free Software
Foundation, either version 3 of the License, or (at your option) any later
version.

OpenCPI is distributed in the hope that it will be useful, but WITHOUT ANY
WARRANTY; without even the implied warranty of MERCHANTABILITY or FITNESS FOR A
PARTICULAR PURPOSE. See the GNU Lesser General Public License for more details.

You should have received a copy of the GNU Lesser General Public License along
with this program. If not, see <http://www.gnu.org/licenses/>.
\fi

% Sets OpenCPI Version used throughout all the docs. This is updated by
% scripts/update-release.sh when a release is being made and must not
% be changed manually.
\def\ocpiversion{v2.2.0}

\documentclass{article}
\author{}  % Force author to be blank
\date{OpenCPI Release:\ \ \ocpiversion}  % Force date to be blank and override date with version
\title{OpenCPI\\\docTitle}  % docTitle must be defined before including this file
%----------------------------------------------------------------------------------------
% Paper size, orientation and margins
%----------------------------------------------------------------------------------------
\usepackage{geometry}
\geometry{
  letterpaper,  % paper type
  portrait,     % text direction
  left=.75in,   % left margin
  top=.75in,    % top margin
  right=.75in,  % right margin
  bottom=.75in  % bottom margin
}
%----------------------------------------------------------------------------------------
% Header/Footer
%----------------------------------------------------------------------------------------
\usepackage{fancyhdr} \pagestyle{fancy}  % required for fancy headers
\renewcommand{\headrulewidth}{0.5pt}
\renewcommand{\footrulewidth}{0.5pt}
\lhead{\small{\docTitle}}
\rhead{\small{OpenCPI}}
%----------------------------------------------------------------------------------------
% Various packages
%----------------------------------------------------------------------------------------
\usepackage{amsmath}
\usepackage[page,toc]{appendix}  % for appendix stuff
\usepackage{enumitem}
\usepackage{graphicx}   % for including pictures by file
\usepackage{hyperref}   % for linking urls and lists
\usepackage{listings}   % for coding language styles
\usepackage{pdflscape}  % for landscape view
\usepackage{pifont}     % for sideways table
\usepackage{ragged2e}   % for justify
\usepackage{rotating}   % for sideways table
\usepackage{scrextend}
\usepackage{setspace}
\usepackage{subfig}
\usepackage{textcomp}
\usepackage[dvipsnames,usenames]{xcolor}  % for color names see https://en.wikibooks.org/wiki/LaTeX/Colors
\usepackage{xstring}
\uchyph=0  % Never hyphenate acronyms like RCC
\renewcommand\_{\textunderscore\allowbreak}  % Allow words to break/newline on underscores
%----------------------------------------------------------------------------------------
% Table packages
%----------------------------------------------------------------------------------------
\usepackage[tableposition=top]{caption}
\usepackage{float}
\floatstyle{plaintop}
\usepackage{longtable}  % for long possibly multi-page tables
\usepackage{multicol}   % for more advanced table layout
\usepackage{multirow}   % for more advanced table layout
\usepackage{tabularx}   % c=center,l=left,r=right,X=fill
% These define tabularx columns "C" and "R" to match "X" but center/right aligned
\newcolumntype{C}{>{\centering\arraybackslash}X}
\newcolumntype{M}[1]{>{\centering\arraybackslash}m{#1}}
\newcolumntype{P}[1]{>{\centering\arraybackslash}p{#1}}
\newcolumntype{R}{>{\raggedleft\arraybackslash}X}
%----------------------------------------------------------------------------------------
% Block Diagram / FSM Drawings
%----------------------------------------------------------------------------------------
\usepackage{tikz}
\usetikzlibrary{arrows,decorations.markings,fit,positioning,shapes}
\usetikzlibrary{automata}  % used for the fsm
\usetikzlibrary{calc}      % for duplicating clients
\usepgfmodule{oo}          % to define a client box
%----------------------------------------------------------------------------------------
% Colors Used
%----------------------------------------------------------------------------------------
\usepackage{colortbl}
\definecolor{blue}{rgb}{.7,.8,.9}
\definecolor{ceruleanblue}{rgb}{0.16, 0.32, 0.75}
\definecolor{cyan}{rgb}{0.0,0.6,0.6}
\definecolor{darkgreen}{rgb}{0,0.6,0}
\definecolor{deepmagenta}{rgb}{0.8, 0.0, 0.8}
\definecolor{maroon}{rgb}{0.5,0,0}
%----------------------------------------------------------------------------------------
% Define where to hyphenate
%----------------------------------------------------------------------------------------
\hyphenation{Cent-OS}
\hyphenation{install-ation}
%----------------------------------------------------------------------------------------
% Define Commands & Rename Commands
%----------------------------------------------------------------------------------------
\newcommand{\code}[1]{\texttt{#1}}  % For inline code snippet or command line
\newcommand{\sref}[1]{Section~\ref{#1}}  % To quickly reference a section
\newcommand{\todo}[1]{\textcolor{red}{TODO: #1}\PackageWarning{TODO:}{#1}}  % To do notes
\renewcommand{\contentsname}{Table of Contents}
\renewcommand{\listfigurename}{List of Figures}
\renewcommand{\listtablename}{List of Tables}

% This gives a link to gitlab.io document. By default, it outputs the filename.
% You can optionally change the link, e.g.
% \githubio{FPGA\_Vendor\_Tools\_Installation\_Guide.pdf} vs.
% \githubio[\textit{FPGA Vendor Tools Installation Guide}]{FPGA\_Vendor\_Tools\_Installation\_Guide.pdf}
% or if you want the raw ugly URL to come out, \githubioURL{FPGA_Vendor_Tools_Installation_Guide.pdf}
\newcommand{\githubio}[2][]{% The default is for FIRST param!
\href{http://opencpi.gitlab.io/releases/\ocpiversion/docs/#2}{\ifthenelse{\equal{#1}{}}{\texttt{#2}}{#1}}}
\newcommand{\gitlabcom}[2][]{% The default is for FIRST param!
\href{http://gitlab.com/opencpi/#2}{\ifthenelse{\equal{#1}{}}{\texttt{#2}}{#1}}}
\newcommand{\githubioURL}[1]{\url{http://opencpi.gitlab.io/releases/\ocpiversion/docs/#1}}
% Lastly, if you want a SINGLE leading path stripped, e.g. assets/X.pdf => X.pdf:
\newcommand{\githubioFlat}[1]{%
\StrBehind{#1}{/}[\den]%
\href{http://opencpi.gitlab.io/releases/\ocpiversion/docs/#1}{\texttt{\den}}%
}
%----------------------------------------------------------------------------------------
% VHDL Coding Language Style
% modified from: http://latex-community.org/forum/viewtopic.php?f=44&t=22076
%----------------------------------------------------------------------------------------
\lstdefinelanguage{VHDL}
{
  basicstyle=\ttfamily\footnotesize,
  columns=fullflexible,keepspaces,  % https://tex.stackexchange.com/a/46695/87531
  keywordstyle=\color{ceruleanblue},
  commentstyle=\color{darkgreen},
  morekeywords={
    library, use, all, entity, is, port, in, out, end, architecture, of,
    begin, and, signal, when, if, else, process, end,
  },
  morecomment=[l]--
}
%----------------------------------------------------------------------------------------
% XML Coding Language Style
% modified from http://tex.stackexchange.com/questions/10255/xml-syntax-highlighting
%----------------------------------------------------------------------------------------
\lstdefinelanguage{XML}
{
  basicstyle=\ttfamily\footnotesize,
  columns=fullflexible,keepspaces,
  morestring=[s]{"}{"},
  morecomment=[s]{!--}{--},
  commentstyle=\color{darkgreen},
  moredelim=[s][\color{black}]{>}{<},
  moredelim=[s][\color{cyan}]{\ }{=},
  stringstyle=\color{maroon},
  identifierstyle=\color{ceruleanblue}
}
%----------------------------------------------------------------------------------------
% DIFF Coding Language Style
% modified from http://tex.stackexchange.com/questions/50176/highlighting-a-diff-file
%----------------------------------------------------------------------------------------
\lstdefinelanguage{diff}
{
  basicstyle=\ttfamily\footnotesize,
  columns=fullflexible,keepspaces,
  breaklines=true,                            % wrap text
  morecomment=[f][\color{ceruleanblue}]{@@},  % group identifier
  morecomment=[f][\color{red}]-,              % deleted lines
  morecomment=[f][\color{darkgreen}]+,        % added lines
  morecomment=[f][\color{deepmagenta}]{---},  % Diff header lines (must appear after +,-)
  morecomment=[f][\color{deepmagenta}]{+++},
}
%----------------------------------------------------------------------------------------
% Python Coding Language Style
%----------------------------------------------------------------------------------------
\lstdefinelanguage{python}
{
  basicstyle=\ttfamily\footnotesize,
  columns=fullflexible,keepspaces,
  keywordstyle=\color{ceruleanblue},
  commentstyle=\color{darkgreen},
  stringstyle=\color{orange},
  morekeywords={
    print, if, sys, len, from, import, as, open,close, def, main, for, else,
    write, read, range,
  },
  comment=[l]{\#}
}
%----------------------------------------------------------------------------------------
% Fontsize Notes in order from smallest to largest
%----------------------------------------------------------------------------------------
%    \tiny
%    \scriptsize
%    \footnotesize
%    \small
%    \normalsize
%    \large
%    \Large
%    \LARGE
%    \huge
%    \Huge

%----------------------------------------------------------------------------------------

\begin{document}
\maketitle
\thispagestyle{empty}
\newpage

	\begin{center}
	\textit{\textbf{Revision History}}
		\begin{table}[H]
		\label{table:revisions} % Add "[H]" to force placement of table
			\begin{tabularx}{\textwidth}{|c|X|l|}
			\hline
			\rowcolor{blue}
			\textbf{Revision} & \textbf{Description of Change} & \textbf{Date} \\
		    \hline
		    v1.1 & Initial Release & 3/2017 \\
		    \hline
		    v1.2 & Updated for OpenCPI Release 1.2 & 8/2017 \\
			\hline
			v1.3 & Updated for OpenCPI Release 1.3 & 1/2018 \\
			\hline
			v1.4 & Updated OCPI\_LIBRARY\_PATH paths & 9/2018 \\
			\hline
			v1.5 & Updated for OpenCPI Release 1.5 & 4/2019 \\
			\hline
			\end{tabularx}
		\end{table}
	\end{center}

\newpage

\tableofcontents

\newpage

\section{References}

	This document assumes a basic understanding of the Linux command line (or ``shell'') environment.  The reference(s) in Table 1 can be used as an overview of OpenCPI and may prove useful.

\def\myreferences{
\hline
FSK App\footnote{Provides details of the ``FSK App'' reference application} & \path{FSK_app.pdf} \\
}
\iffalse
This file is protected by Copyright. Please refer to the COPYRIGHT file
distributed with this source distribution.

This file is part of OpenCPI <http://www.opencpi.org>

OpenCPI is free software: you can redistribute it and/or modify it under the
terms of the GNU Lesser General Public License as published by the Free Software
Foundation, either version 3 of the License, or (at your option) any later
version.

OpenCPI is distributed in the hope that it will be useful, but WITHOUT ANY
WARRANTY; without even the implied warranty of MERCHANTABILITY or FITNESS FOR A
PARTICULAR PURPOSE. See the GNU Lesser General Public License for more details.

You should have received a copy of the GNU Lesser General Public License along
with this program. If not, see <http://www.gnu.org/licenses/>.
\fi

% This snippet creates the "References" table labeled "table:references"
% It creates three columns: Name, Publisher, Link and then inserts default documents
%
% To skip these defaults, define macros named
% refskipgs to skip "Getting Started"
% refskipig to skip "Installation Guide"
% refskipac to skip "Acronyms and Definitions"
% refskipocpiov to skip "OpenCPI Overview"
%
% See RPM_Installation_Guide.tex for examples
%
% After the defaults, it optionally inserts the "myreferences" macro that
% you defined elsewhere (you put hlines above all lines)
%
% If you want the \caption on the bottom, define "refcapbottom"
\begin{center}
\renewcommand*\footnoterule{} % Remove separator line from footnote
\renewcommand{\thempfootnote}{\arabic{mpfootnote}} % Use Arabic numbers (or can't reuse)
\begin{minipage}{0.9\textwidth}
  \begin{table}[H]
\ifx\refcapbottom\undefined
  \caption {References}
  \label{table:references}
\fi
  \begin{tabularx}{\textwidth}{|C|C|}
    \hline
    \rowcolor{blue}
    \textbf{Title} & \textbf{Link} \\
\ifx\refskipocpiov\undefined
    \hline
    OpenCPI User Guide & \githubio{OpenCPI\_User.pdf} \\
\fi
\ifx\refskipac\undefined
    \hline
    Acronyms and Definitions & \githubio{Acronyms\_and\_Definitions.pdf} \\
\fi
\ifx\refskipig\undefined
    \hline
    OpenCPI Installation Guide & \githubio{OpenCPI\_Installation.pdf} \\
\fi
\ifx\myreferences\undefined
\else
    \myreferences
\fi
    \hline
  \end{tabularx}
\ifx\refcapbottom\undefined
\else
  \caption {References}
  \label{table:references}
\fi
  \end{table}
\end{minipage}
\end{center}


\newpage
\begin{flushleft}
\section{Overview}
This purpose of this document is to provide a compact set of instructions to build, run, and verify the OpenCPI FSK App reference application. While the FSK reference application supports several execution modes (filerw, txrx, bbloopback, rx, tx) and is supported on various systems, this guide will only discuss execution of the \textit{txrx} mode on the Epiq Solutions Matchstiq-Z1.

\section{Prerequisites}
This document assumes that the OpenCPI framework and the appropriate  Xilinx Vivado tools have been installed. The application is supported on all of the OpenCPI platforms, but all of the examples shown here are for the Matchstiq-Z1 platform.

\section{Build the \textit{core} project}
If the \textit{core} project has not been created yet, follow the instructions in the Getting Started Guide. Once the \textit{core} project has been created, the following ocpidev command can be used to build the primitives and workers required by the FSK app. Navigate to the \textit{core} project directory and run the command:
\begin{verbatim}
ocpidev build --rcc --rcc-platform xilinx13_3 --hdl --hdl-platform matchstiq_z1 --no-assemblies
\end{verbatim}
Note: The -\/-no-assemblies argument excludes the creation of executable bitstreams.\\
This step takes approximately 20 minutes to complete.\\

\section{Build the \textit{assets} project}
If the \textit{assets} project has not been created yet, follow the instructions in the Getting Started Guide. Once the \textit{assets} project has been created, the following ocpidev command can be used to build the primitives and workers required by the FSK app. Navigate to the \textit{assets} project directory and run the command:
\begin{verbatim}
ocpidev build --rcc --rcc-platform xilinx13_3 --hdl --hdl-platform matchstiq_z1 --hdl-assembly fsk_modem
\end{verbatim}
This step takes approximately 60 minutes to complete.

\section{Build the FSK Application Executable}
Next, ensure the executable for the FSK Application has been built. Navigate to the \texttt{applications/FSK} of the \textit{assets} project and run the command:
\begin{verbatim}
ocpidev build --rcc-platform xilinx13_3
\end{verbatim}
If successful, a directory name \textit{target-xilinx13\_3} will contain the FSK executable.

\section{Running the Application}
Prior to executing the application, ensure that the RX and TX ports of the platform are connected together via a coax cable, RF attenuation pads are not necessary for this test. The following steps describe setup and execution of the FSK in \textit{txrx mode} on the Matchstiq-Z1. The \textit{applications/FSK/Makefile} contains steps to execute the FSK in its other modes. To view these steps, open the Makefile in an editor or execute ``\textit{make show}'' in the FSK application's directory.
\subsection{Setting Up the Execution Platform}
For embedded platforms (Matchstiq-Z1), connect to the platform via a serial port. Example syntax for establishing a connection can be seen below:
\begin{verbatim}
sudo screen /dev/ttyUSB0 115200
\end{verbatim}
\subsubsection{Networked mode}
Networked mode is one of two modes typically used for running the application. It involves creating NFS mounts between the development platform and the execution platform to enable application deployment. A setup script is used to automate the required steps for this mode. Example syntax for running this script can be seen below:
\begin{verbatim}
source /mnt/card/opencpi/mynetsetup.sh <IP Address of Development Host>
\end{verbatim}
\subsubsection{Standalone mode}
Standalone mode is the other mode used for running the application. It involves deploying the application using files only on local storage and not over a network connection. Similar to Networked mode, the setup for Standalone mode also involves running a setup script. Example syntax for running this script can be seen below:
\begin{verbatim}
source /mnt/card/opencpi/mysetup.sh
\end{verbatim}
\subsection{Setting OCPI\_LIBRARY\_PATH Environment Variable}
The OCPI\_LIBRARY\_PATH environment variable is a colon-separated list of files/directories which is searched for executable artifacts during deployment of the application. For proper execution of the FSK App, all of the FSK App artifacts (detailed in the FSK App document) must be included in the \texttt{OCPI\_LIBRARY\_PATH} environment variable. Example syntax for setting the library path is provide in the below sections.
\subsubsection{Networked Mode}
In networked mode, artifact files can be accessed via NFS mounts. In the below export string, <RCC platform> must be replace with the appropriate OpenCPI's name for target RCC platform, i.e. xilinx13\_3, xilinx13\_4. This name can be obtained by examining the output of ``ocpirun -C'', when performed on the target platform.
\begin{verbatim}
$ export OCPI_LIBRARY_PATH=/mnt/net/cdk/<RCC platform>/artifacts:\
/mnt/ocpi_assets/applications/FSK/../../hdl/assemblies/fsk_modem:\
/mnt/ocpi_assets/applications/FSK/../../artifacts
\end{verbatim}

At the time of this release, the <RCC platform> for the Matchstiq-Z1 is limited to xilinx13\_3, as shown in the export below. Used this export string to configured the environment on the Matchstiq-Z1 for execution of the FSK in \textit{txrx} and \textit{bbloopback} modes.
\begin{verbatim}
$ export OCPI_LIBRARY_PATH=/mnt/net/cdk/xilinx13_3/artifacts:\
/mnt/ocpi_assets/applications/FSK/../../hdl/assemblies/fsk_modem:\
/mnt/ocpi_assets/applications/FSK/../../artifacts
\end{verbatim}


\subsubsection{Standalone Mode}
In standalone mode, artifact files can be copied to the SD card.
\begin{verbatim}
$ export OCPI_LIBRARY_PATH=/mnt/card/opencpi/artifacts
\end{verbatim}
\subsection{Running the Application}
To run the application, navigate to the \texttt{applications/FSK} directory and run the executable in the target-* directory of your Execution Platform. Example syntax can be seen below.
\begin{verbatim}
$ ./target-xilinx13_3/FSK txrx
\end{verbatim}
Reference the \textit{FSK\_app.pdf} for executing the various modes (filerw, tx, rx, txrx, bbloopback) of the FSK application. Ensure that OCPI\_LIBRARY\_PATH is configured, as provided, for the respective mode and platform. \\ \medskip

\subsection{View the Results}
After the application completes, the output can be found in the \texttt{applications/FSK/odata} directory. To view the results on the Development Host, navigate to the \texttt{applications/FSK} directory and execute the command:
\begin{verbatim}
$ eog odata/out_app_fsk_txrx.bin
\end{verbatim}
\end{flushleft}
\end{document}
