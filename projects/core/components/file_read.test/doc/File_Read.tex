\iffalse
This file is protected by Copyright. Please refer to the COPYRIGHT file
distributed with this source distribution.

This file is part of OpenCPI <http://www.opencpi.org>

OpenCPI is free software: you can redistribute it and/or modify it under the
terms of the GNU Lesser General Public License as published by the Free Software
Foundation, either version 3 of the License, or (at your option) any later
version.

OpenCPI is distributed in the hope that it will be useful, but WITHOUT ANY
WARRANTY; without even the implied warranty of MERCHANTABILITY or FITNESS FOR A
PARTICULAR PURPOSE. See the GNU Lesser General Public License for more details.

You should have received a copy of the GNU Lesser General Public License along
with this program. If not, see <http://www.gnu.org/licenses/>.
\fi

%----------------------------------------------------------------------------------------
% Required document specific properties
%----------------------------------------------------------------------------------------
\def\comp{file\_{}read}
\def\Comp{File\ Read}
\def\docTitle{\Comp{} Component Data Sheet}
\def\snippetpath{../../../../../doc/av/tex/snippets/}
%----------------------------------------------------------------------------------------
% Global latex header (this must be after document specific properties)
%----------------------------------------------------------------------------------------
\iffalse
This file is protected by Copyright. Please refer to the COPYRIGHT file
distributed with this source distribution.

This file is part of OpenCPI <http://www.opencpi.org>

OpenCPI is free software: you can redistribute it and/or modify it under the
terms of the GNU Lesser General Public License as published by the Free Software
Foundation, either version 3 of the License, or (at your option) any later
version.

OpenCPI is distributed in the hope that it will be useful, but WITHOUT ANY
WARRANTY; without even the implied warranty of MERCHANTABILITY or FITNESS FOR A
PARTICULAR PURPOSE. See the GNU Lesser General Public License for more details.

You should have received a copy of the GNU Lesser General Public License along
with this program. If not, see <http://www.gnu.org/licenses/>.
\fi

% Sets OpenCPI Version used throughout all the docs. This is updated by
% scripts/update-release.sh when a release is being made and must not
% be changed manually.
\def\ocpiversion{v2.2.0}

\documentclass{article}
\author{}  % Force author to be blank
\date{OpenCPI Release:\ \ \ocpiversion}  % Force date to be blank and override date with version
\title{OpenCPI\\\docTitle}  % docTitle must be defined before including this file
%----------------------------------------------------------------------------------------
% Paper size, orientation and margins
%----------------------------------------------------------------------------------------
\usepackage{geometry}
\geometry{
  letterpaper,  % paper type
  portrait,     % text direction
  left=.75in,   % left margin
  top=.75in,    % top margin
  right=.75in,  % right margin
  bottom=.75in  % bottom margin
}
%----------------------------------------------------------------------------------------
% Header/Footer
%----------------------------------------------------------------------------------------
\usepackage{fancyhdr} \pagestyle{fancy}  % required for fancy headers
\renewcommand{\headrulewidth}{0.5pt}
\renewcommand{\footrulewidth}{0.5pt}
\lhead{\small{\docTitle}}
\rhead{\small{OpenCPI}}
%----------------------------------------------------------------------------------------
% Various packages
%----------------------------------------------------------------------------------------
\usepackage{amsmath}
\usepackage[page,toc]{appendix}  % for appendix stuff
\usepackage{enumitem}
\usepackage{graphicx}   % for including pictures by file
\usepackage{hyperref}   % for linking urls and lists
\usepackage{listings}   % for coding language styles
\usepackage{pdflscape}  % for landscape view
\usepackage{pifont}     % for sideways table
\usepackage{ragged2e}   % for justify
\usepackage{rotating}   % for sideways table
\usepackage{scrextend}
\usepackage{setspace}
\usepackage{subfig}
\usepackage{textcomp}
\usepackage[dvipsnames,usenames]{xcolor}  % for color names see https://en.wikibooks.org/wiki/LaTeX/Colors
\usepackage{xstring}
\uchyph=0  % Never hyphenate acronyms like RCC
\renewcommand\_{\textunderscore\allowbreak}  % Allow words to break/newline on underscores
%----------------------------------------------------------------------------------------
% Table packages
%----------------------------------------------------------------------------------------
\usepackage[tableposition=top]{caption}
\usepackage{float}
\floatstyle{plaintop}
\usepackage{longtable}  % for long possibly multi-page tables
\usepackage{multicol}   % for more advanced table layout
\usepackage{multirow}   % for more advanced table layout
\usepackage{tabularx}   % c=center,l=left,r=right,X=fill
% These define tabularx columns "C" and "R" to match "X" but center/right aligned
\newcolumntype{C}{>{\centering\arraybackslash}X}
\newcolumntype{M}[1]{>{\centering\arraybackslash}m{#1}}
\newcolumntype{P}[1]{>{\centering\arraybackslash}p{#1}}
\newcolumntype{R}{>{\raggedleft\arraybackslash}X}
%----------------------------------------------------------------------------------------
% Block Diagram / FSM Drawings
%----------------------------------------------------------------------------------------
\usepackage{tikz}
\usetikzlibrary{arrows,decorations.markings,fit,positioning,shapes}
\usetikzlibrary{automata}  % used for the fsm
\usetikzlibrary{calc}      % for duplicating clients
\usepgfmodule{oo}          % to define a client box
%----------------------------------------------------------------------------------------
% Colors Used
%----------------------------------------------------------------------------------------
\usepackage{colortbl}
\definecolor{blue}{rgb}{.7,.8,.9}
\definecolor{ceruleanblue}{rgb}{0.16, 0.32, 0.75}
\definecolor{cyan}{rgb}{0.0,0.6,0.6}
\definecolor{darkgreen}{rgb}{0,0.6,0}
\definecolor{deepmagenta}{rgb}{0.8, 0.0, 0.8}
\definecolor{maroon}{rgb}{0.5,0,0}
%----------------------------------------------------------------------------------------
% Define where to hyphenate
%----------------------------------------------------------------------------------------
\hyphenation{Cent-OS}
\hyphenation{install-ation}
%----------------------------------------------------------------------------------------
% Define Commands & Rename Commands
%----------------------------------------------------------------------------------------
\newcommand{\code}[1]{\texttt{#1}}  % For inline code snippet or command line
\newcommand{\sref}[1]{Section~\ref{#1}}  % To quickly reference a section
\newcommand{\todo}[1]{\textcolor{red}{TODO: #1}\PackageWarning{TODO:}{#1}}  % To do notes
\renewcommand{\contentsname}{Table of Contents}
\renewcommand{\listfigurename}{List of Figures}
\renewcommand{\listtablename}{List of Tables}

% This gives a link to gitlab.io document. By default, it outputs the filename.
% You can optionally change the link, e.g.
% \githubio{FPGA\_Vendor\_Tools\_Installation\_Guide.pdf} vs.
% \githubio[\textit{FPGA Vendor Tools Installation Guide}]{FPGA\_Vendor\_Tools\_Installation\_Guide.pdf}
% or if you want the raw ugly URL to come out, \githubioURL{FPGA_Vendor_Tools_Installation_Guide.pdf}
\newcommand{\githubio}[2][]{% The default is for FIRST param!
\href{http://opencpi.gitlab.io/releases/\ocpiversion/docs/#2}{\ifthenelse{\equal{#1}{}}{\texttt{#2}}{#1}}}
\newcommand{\gitlabcom}[2][]{% The default is for FIRST param!
\href{http://gitlab.com/opencpi/#2}{\ifthenelse{\equal{#1}{}}{\texttt{#2}}{#1}}}
\newcommand{\githubioURL}[1]{\url{http://opencpi.gitlab.io/releases/\ocpiversion/docs/#1}}
% Lastly, if you want a SINGLE leading path stripped, e.g. assets/X.pdf => X.pdf:
\newcommand{\githubioFlat}[1]{%
\StrBehind{#1}{/}[\den]%
\href{http://opencpi.gitlab.io/releases/\ocpiversion/docs/#1}{\texttt{\den}}%
}
%----------------------------------------------------------------------------------------
% VHDL Coding Language Style
% modified from: http://latex-community.org/forum/viewtopic.php?f=44&t=22076
%----------------------------------------------------------------------------------------
\lstdefinelanguage{VHDL}
{
  basicstyle=\ttfamily\footnotesize,
  columns=fullflexible,keepspaces,  % https://tex.stackexchange.com/a/46695/87531
  keywordstyle=\color{ceruleanblue},
  commentstyle=\color{darkgreen},
  morekeywords={
    library, use, all, entity, is, port, in, out, end, architecture, of,
    begin, and, signal, when, if, else, process, end,
  },
  morecomment=[l]--
}
%----------------------------------------------------------------------------------------
% XML Coding Language Style
% modified from http://tex.stackexchange.com/questions/10255/xml-syntax-highlighting
%----------------------------------------------------------------------------------------
\lstdefinelanguage{XML}
{
  basicstyle=\ttfamily\footnotesize,
  columns=fullflexible,keepspaces,
  morestring=[s]{"}{"},
  morecomment=[s]{!--}{--},
  commentstyle=\color{darkgreen},
  moredelim=[s][\color{black}]{>}{<},
  moredelim=[s][\color{cyan}]{\ }{=},
  stringstyle=\color{maroon},
  identifierstyle=\color{ceruleanblue}
}
%----------------------------------------------------------------------------------------
% DIFF Coding Language Style
% modified from http://tex.stackexchange.com/questions/50176/highlighting-a-diff-file
%----------------------------------------------------------------------------------------
\lstdefinelanguage{diff}
{
  basicstyle=\ttfamily\footnotesize,
  columns=fullflexible,keepspaces,
  breaklines=true,                            % wrap text
  morecomment=[f][\color{ceruleanblue}]{@@},  % group identifier
  morecomment=[f][\color{red}]-,              % deleted lines
  morecomment=[f][\color{darkgreen}]+,        % added lines
  morecomment=[f][\color{deepmagenta}]{---},  % Diff header lines (must appear after +,-)
  morecomment=[f][\color{deepmagenta}]{+++},
}
%----------------------------------------------------------------------------------------
% Python Coding Language Style
%----------------------------------------------------------------------------------------
\lstdefinelanguage{python}
{
  basicstyle=\ttfamily\footnotesize,
  columns=fullflexible,keepspaces,
  keywordstyle=\color{ceruleanblue},
  commentstyle=\color{darkgreen},
  stringstyle=\color{orange},
  morekeywords={
    print, if, sys, len, from, import, as, open,close, def, main, for, else,
    write, read, range,
  },
  comment=[l]{\#}
}
%----------------------------------------------------------------------------------------
% Fontsize Notes in order from smallest to largest
%----------------------------------------------------------------------------------------
%    \tiny
%    \scriptsize
%    \footnotesize
%    \small
%    \normalsize
%    \large
%    \Large
%    \LARGE
%    \huge
%    \Huge

\graphicspath{{figures/}}
%----------------------------------------------------------------------------------------

\begin{document}
\maketitle
\thispagestyle{empty}
\newpage

\begin{center}
  \textit{\textbf{Revision History}}
\end{center}
\begin{longtable}{|p{\dimexpr0.15\textwidth-2\tabcolsep\relax}
                  |p{\dimexpr0.65\textwidth-2\tabcolsep\relax}
                  |p{\dimexpr0.2\textwidth-2\tabcolsep\relax}|}
  \hline
  \rowcolor{blue}
  \textbf{Revision} & \textbf{Description of Change} & \textbf{Date} \\
  \hline
  v1.4 & Initial release. & 09/2018 \\
  \hline
  v1.5 & Changed all readable properties to readback and changed default value of MessageSize from 4096 to 0. Also updated the end of file descriptions with smart wrapper updates.  & 04/2019 \\
  \hline
\end{longtable}
\newpage

\def\name{\comp}
\def\workertype{Application}
\def\version{\ocpiversion}
\def\releasedate{04/2019}
\def\componentlibrary{ocpi.core}
\def\workers{\comp{}.hdl, \comp{}.rcc}
\def\testedplatforms{centos7, isim, modelsim, xilinx13\_{}3, xilinx13\_{}4, xsim}
\section*{Summary - \Comp}
\begin{tabular}{|c|M{13.5cm}|}
  \hline
  \rowcolor{blue}
   & \\
  \hline
  Name              & \comp             \\
  \hline
  Worker Type       & \workertype       \\
  \hline
  OpenCPI Release   & \ocpiversion      \\
  \hline
  Last Update       & \releasedate      \\
  \hline
  Component Library & \componentlibrary \\
  \hline
  Workers           & \workers          \\
  \hline
  Tested Platforms  & \testedplatforms  \\
  \hline
\end{tabular}


\section*{Functionality}
\begin{flushleft}
The File\_Read component injects file-based data into an application. It is used
by specifying an instance of the File\_Read component, and connecting its output port to
an input port of the component which will process the data first. The name of the file to
be read is specified in a property.
\subsection*{Operating Modes}
\subsubsection*{Data Streaming Mode}
In data streaming mode, the contents of the file becomes the payloads of a stream of
messages, each carrying a fixed number of bytes of file data and all with
the same opcode. The length and opcode of all output messages are specified as
properties. This means that this mode only lends itself to protocols that have a single opcode or the intent is to only send data on one of the opcodes of a multi-opcode protocol.\\ \medskip \medskip
If the number of bytes in the file is not an even multiple of the message size the
remaining bytes are sent in a final shorter message. The granularity of messages can
also be specified. This forces the message size to be a multiple of this value, and
forces truncation of the final message to be a multiple of this value.
\newpage
\subsubsection*{Messaging Mode}
In messaging mode, the contents of the file are interpreted as a sequence of defined
messages, with an 8-byte header in the file itself preceding the payload for each message.
This header contains the length and opcode of the message, with the data contents of
the message following the header. The length can be zero, meaning that a message
will be sent with the indicated opcode, and the length of the message will be zero.\\
\medskip \medskip
\input{snippets/messaging_snippet}
\subsection*{No Protocol}
The port on the component has no protocol specified.  This means that the data file must be formatted to match the protocol of the input port of the connected worker.  For message mode this means only using opcodes and payloads in the file that correspond to the protocol of the connected component.  In data streaming mode the file structure needs to correspond to the opcode that is set by the \textit{opcode} property.
\subsection*{Message Size/Buffers}
The system normally determines buffer sizes based on protocol.  Since File\_Read's output port has no protocol, the protocol used is that of the port that this output port is connected to.  If the connected port also has no protocol, then the system's default of 2KB is chosen.  This chosen buffer size can always be overriden in the OAS or with ACI PValues for \textit{any} connection in the application.\\ \medskip

The \textbf{messageSize} property for this component has a default value of zero, which means the system's (or application's) chosen buffer size is used.  If set to non-zero, it overrides the system/application value, which risks trying to write messages larger than the system is prepared for, resulting in an error.\\ \medskip

If the \textbf{messagesInFile} property is true (i.e. the component is operating in messaging mode), the buffer size must be large enough to accommodate the largest message found in the input file.  It is always better to set the buffer size on connections in the OAS (or using PValues in the ACI) than to use the \textbf{messageSize} property since the former method is universal for specifying buffer sizes for all connections in the application.\\ \medskip

For example the iqstream protocol uses a sequence of 2048 16-bit I/Q pairs which means any message over 8192 (2048 * 4 bytes per pair) is invalid.
\subsection*{End of File Handling}
When the File\_Read component reaches the end of its input file, it will do one of three
things:
\begin{itemize}
  \item send a end of file notification
  \item enter the ``done'' state with no further action, when the \textit{suppressEOF} property
is true
  \item restart reading at the beginning of the file, when the \textit{repeat} property is true
\end{itemize}
\end{flushleft}

\section*{Worker Implementation Details}
\subsection*{\comp.hdl}
\begin{flushleft}
This worker will only run on simulator platforms.  This includes isim, xsim, and modelsim and will not run on or be built for any other hardware platforms.  This is because it conatins code that cannot be realized into RTL.
\end{flushleft}
\subsection*{\comp.rcc}
\begin{flushleft}
This worker in implemented in the C language version of the RCC model.  Most new workers are implemented in the C++ language version of the RCC model.
\end{flushleft}

\section*{Source Dependencies}
\subsection*{\comp.rcc}
\begin{itemize}
\item file\_read.c
\item file\_read.h
\end{itemize}
\subsection*{\comp.hdl}
\begin{itemize}
\item file\_read.vhd
\end{itemize}


\begin{landscape}
	\section*{Component Spec Properties}
	\begin{scriptsize}
\begin{tabular}{|p{2cm}|p{2cm}|c|c|c|p{1.5cm}|p{1cm}|p{7cm}|}
\hline
\rowcolor{blue}
Name                 & Type   & SequenceLength & ArrayDimensions & Accessibility       & Valid Range & Default & Usage
\\
\hline
fileName & string length:1024 & - & - & Initial & -  &- & The name of the file whose contents are sent out as raw data to the output port.
\\
\hline
messagesInFile & bool  & - & - & Initial & -  &false & The flag that is used to turn on and off message mode. See section on Message Mode.
\\
\hline
opcode & uchar  & - & - & Initial & -  &0 & The Opcode that all the data in the datafile is sent in streaming mode. The opcode of the ZLM at the end of the file in messaging mode.
\\
\hline
messageSize & ulong  & - & - & Initial & -  &0 & The size of the messages in bytes that are created on the output port.  The connected worker's buffer needs to be big enough to take the data buffer that is being passed to the worker.  See Output Port section.
\\
\hline
granularity & ulong  & - & - & Initial & -  &1 & The final message at the end of a file will truncated to be a multiple of this property's value in bytes.
\\
\hline
repeat & bool  & - & - & Writeable & -  &- & The flag used to repeat the data file over and over.
\\
\hline
bytesRead & uLongLong  & - & - & Volatile & -  &- & The number of bytes read from file.  Useful for debugging data flow issues.
\\
\hline
messagesWritten & uLongLong  & - & - & Volatile & -  &- & The number of messages read from file.  Good property for debugging data flow issues.
\\
\hline
suppressEOF & bool  & - & - & Initial & -  &- & The flag used to enable or disable the Zero Length message that is propagated at the end of the file.
\\
\hline
badMessage & bool  & - & - & Volatile & -  &- & This flag is set by the worker if the worker has a problem getting data from file. An example of this happening is a bad filename.
\\
\hline
\end{tabular}
	\end{scriptsize}

	\section*{Worker Properties}
	\subsection*{\comp.hdl}
	\begin{scriptsize}
\begin{tabular}{|p{2cm}|p{2.75cm}|p{3.5cm}|p{2cm}|p{2cm}|p{2.25cm}|p{1.5cm}|p{1cm}|p{4cm}|}
			\hline
			\rowcolor{blue}
			Type     & Name                      & Type  & SequenceLength & ArrayDimensions & Accessibility       & Valid Range & Default & Usage                                      \\
			\hline
			Spec Property & fileName & string  & - & - & Readback & -  &- & added Readback
            \\
            \hline
            Spec Property & suppressEOF & bool  & - & - & Readback & -  &- & added Readback
            \\
            \hline
            Property & CWD\_MAX\_LENGTH & ulong  & - & - & Paramater & -  & 512 & parameter for max string length for the cwd property
            \\
            \hline
            Property & cwd & string length:CWD\_MAX\_LENGTH & - & - & Volatile & -  &- & the current working directory of the application (this is required for the hdl version of this worker and cannot be determined automatically)
            \\
            \hline
    \end{tabular}
	\end{scriptsize}

	\subsection*{\comp.rcc}
    \begin{scriptsize}
    \begin{tabular}{|p{2cm}|p{2.75cm}|p{1cm}|p{2.75cm}|p{2cm}|p{2.25cm}|p{2cm}|p{1cm}|p{5cm}|}
			\hline
			\rowcolor{blue}
			Type     & Name                      & Type  & SequenceLength & ArrayDimensions & Accessibility       & Valid Range & Default & Usage                                      \\
			\hline
            Spec Property & messageSize & ulong  & - & - &  Volatile & -  &4096 & added Volatile
            \\
            \hline
    \end{tabular}
	\end{scriptsize}

	\section*{Component Ports}
	\begin{scriptsize}
\begin{tabular}{|M{2cm}|M{1.5cm}|M{4cm}|c|c|M{9cm}|}
\hline
\rowcolor{blue}
Name & Producer & Protocol & Optional & Advanced & Usage
\\
\hline
out & true & None& False & - & Data streamed from file\\
\hline
\end{tabular}
	\end{scriptsize}

	\section*{Worker Interfaces}
	\subsection*{\comp.hdl}
	\begin{scriptsize}
\begin{tabular}{|M{2cm}|M{1.5cm}|M{4cm}|c|M{12cm}|}
\hline
\rowcolor{blue}
Type & Name & DataWidth & Advanced & Usage
\\
\hline
StreamInterface & out & 32 & - & Data streamed from file\\
\hline
\end{tabular}
	\end{scriptsize}
\end{landscape}

\section*{Test and Verification}
\begin{flushleft}
All test benches use this worker as part of the verification process. A unit-test does not exist for this component.
\end{flushleft}
\end{document}
