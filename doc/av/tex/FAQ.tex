\iffalse
This file is protected by Copyright. Please refer to the COPYRIGHT file
distributed with this source distribution.

This file is part of OpenCPI <http://www.opencpi.org>

OpenCPI is free software: you can redistribute it and/or modify it under the
terms of the GNU Lesser General Public License as published by the Free Software
Foundation, either version 3 of the License, or (at your option) any later
version.

OpenCPI is distributed in the hope that it will be useful, but WITHOUT ANY
WARRANTY; without even the implied warranty of MERCHANTABILITY or FITNESS FOR A
PARTICULAR PURPOSE. See the GNU Lesser General Public License for more details.

You should have received a copy of the GNU Lesser General Public License along
with this program. If not, see <http://www.gnu.org/licenses/>.
\fi

%----------------------------------------------------------------------------------------
% Required document specific properties
%----------------------------------------------------------------------------------------
\def\docTitle{Frequently Asked Questions}
\def\snippetpath{snippets}
%----------------------------------------------------------------------------------------
% Global latex header (this must be after document specific properties)
%----------------------------------------------------------------------------------------
\iffalse
This file is protected by Copyright. Please refer to the COPYRIGHT file
distributed with this source distribution.

This file is part of OpenCPI <http://www.opencpi.org>

OpenCPI is free software: you can redistribute it and/or modify it under the
terms of the GNU Lesser General Public License as published by the Free Software
Foundation, either version 3 of the License, or (at your option) any later
version.

OpenCPI is distributed in the hope that it will be useful, but WITHOUT ANY
WARRANTY; without even the implied warranty of MERCHANTABILITY or FITNESS FOR A
PARTICULAR PURPOSE. See the GNU Lesser General Public License for more details.

You should have received a copy of the GNU Lesser General Public License along
with this program. If not, see <http://www.gnu.org/licenses/>.
\fi

% Sets OpenCPI Version used throughout all the docs. This is updated by
% scripts/update-release.sh when a release is being made and must not
% be changed manually.
\def\ocpiversion{v2.2.0}

\documentclass{article}
\author{}  % Force author to be blank
\date{OpenCPI Release:\ \ \ocpiversion}  % Force date to be blank and override date with version
\title{OpenCPI\\\docTitle}  % docTitle must be defined before including this file
%----------------------------------------------------------------------------------------
% Paper size, orientation and margins
%----------------------------------------------------------------------------------------
\usepackage{geometry}
\geometry{
  letterpaper,  % paper type
  portrait,     % text direction
  left=.75in,   % left margin
  top=.75in,    % top margin
  right=.75in,  % right margin
  bottom=.75in  % bottom margin
}
%----------------------------------------------------------------------------------------
% Header/Footer
%----------------------------------------------------------------------------------------
\usepackage{fancyhdr} \pagestyle{fancy}  % required for fancy headers
\renewcommand{\headrulewidth}{0.5pt}
\renewcommand{\footrulewidth}{0.5pt}
\lhead{\small{\docTitle}}
\rhead{\small{OpenCPI}}
%----------------------------------------------------------------------------------------
% Various packages
%----------------------------------------------------------------------------------------
\usepackage{amsmath}
\usepackage[page,toc]{appendix}  % for appendix stuff
\usepackage{enumitem}
\usepackage{graphicx}   % for including pictures by file
\usepackage{hyperref}   % for linking urls and lists
\usepackage{listings}   % for coding language styles
\usepackage{pdflscape}  % for landscape view
\usepackage{pifont}     % for sideways table
\usepackage{ragged2e}   % for justify
\usepackage{rotating}   % for sideways table
\usepackage{scrextend}
\usepackage{setspace}
\usepackage{subfig}
\usepackage{textcomp}
\usepackage[dvipsnames,usenames]{xcolor}  % for color names see https://en.wikibooks.org/wiki/LaTeX/Colors
\usepackage{xstring}
\uchyph=0  % Never hyphenate acronyms like RCC
\renewcommand\_{\textunderscore\allowbreak}  % Allow words to break/newline on underscores
%----------------------------------------------------------------------------------------
% Table packages
%----------------------------------------------------------------------------------------
\usepackage[tableposition=top]{caption}
\usepackage{float}
\floatstyle{plaintop}
\usepackage{longtable}  % for long possibly multi-page tables
\usepackage{multicol}   % for more advanced table layout
\usepackage{multirow}   % for more advanced table layout
\usepackage{tabularx}   % c=center,l=left,r=right,X=fill
% These define tabularx columns "C" and "R" to match "X" but center/right aligned
\newcolumntype{C}{>{\centering\arraybackslash}X}
\newcolumntype{M}[1]{>{\centering\arraybackslash}m{#1}}
\newcolumntype{P}[1]{>{\centering\arraybackslash}p{#1}}
\newcolumntype{R}{>{\raggedleft\arraybackslash}X}
%----------------------------------------------------------------------------------------
% Block Diagram / FSM Drawings
%----------------------------------------------------------------------------------------
\usepackage{tikz}
\usetikzlibrary{arrows,decorations.markings,fit,positioning,shapes}
\usetikzlibrary{automata}  % used for the fsm
\usetikzlibrary{calc}      % for duplicating clients
\usepgfmodule{oo}          % to define a client box
%----------------------------------------------------------------------------------------
% Colors Used
%----------------------------------------------------------------------------------------
\usepackage{colortbl}
\definecolor{blue}{rgb}{.7,.8,.9}
\definecolor{ceruleanblue}{rgb}{0.16, 0.32, 0.75}
\definecolor{cyan}{rgb}{0.0,0.6,0.6}
\definecolor{darkgreen}{rgb}{0,0.6,0}
\definecolor{deepmagenta}{rgb}{0.8, 0.0, 0.8}
\definecolor{maroon}{rgb}{0.5,0,0}
%----------------------------------------------------------------------------------------
% Define where to hyphenate
%----------------------------------------------------------------------------------------
\hyphenation{Cent-OS}
\hyphenation{install-ation}
%----------------------------------------------------------------------------------------
% Define Commands & Rename Commands
%----------------------------------------------------------------------------------------
\newcommand{\code}[1]{\texttt{#1}}  % For inline code snippet or command line
\newcommand{\sref}[1]{Section~\ref{#1}}  % To quickly reference a section
\newcommand{\todo}[1]{\textcolor{red}{TODO: #1}\PackageWarning{TODO:}{#1}}  % To do notes
\renewcommand{\contentsname}{Table of Contents}
\renewcommand{\listfigurename}{List of Figures}
\renewcommand{\listtablename}{List of Tables}

% This gives a link to gitlab.io document. By default, it outputs the filename.
% You can optionally change the link, e.g.
% \githubio{FPGA\_Vendor\_Tools\_Installation\_Guide.pdf} vs.
% \githubio[\textit{FPGA Vendor Tools Installation Guide}]{FPGA\_Vendor\_Tools\_Installation\_Guide.pdf}
% or if you want the raw ugly URL to come out, \githubioURL{FPGA_Vendor_Tools_Installation_Guide.pdf}
\newcommand{\githubio}[2][]{% The default is for FIRST param!
\href{http://opencpi.gitlab.io/releases/\ocpiversion/docs/#2}{\ifthenelse{\equal{#1}{}}{\texttt{#2}}{#1}}}
\newcommand{\gitlabcom}[2][]{% The default is for FIRST param!
\href{http://gitlab.com/opencpi/#2}{\ifthenelse{\equal{#1}{}}{\texttt{#2}}{#1}}}
\newcommand{\githubioURL}[1]{\url{http://opencpi.gitlab.io/releases/\ocpiversion/docs/#1}}
% Lastly, if you want a SINGLE leading path stripped, e.g. assets/X.pdf => X.pdf:
\newcommand{\githubioFlat}[1]{%
\StrBehind{#1}{/}[\den]%
\href{http://opencpi.gitlab.io/releases/\ocpiversion/docs/#1}{\texttt{\den}}%
}
%----------------------------------------------------------------------------------------
% VHDL Coding Language Style
% modified from: http://latex-community.org/forum/viewtopic.php?f=44&t=22076
%----------------------------------------------------------------------------------------
\lstdefinelanguage{VHDL}
{
  basicstyle=\ttfamily\footnotesize,
  columns=fullflexible,keepspaces,  % https://tex.stackexchange.com/a/46695/87531
  keywordstyle=\color{ceruleanblue},
  commentstyle=\color{darkgreen},
  morekeywords={
    library, use, all, entity, is, port, in, out, end, architecture, of,
    begin, and, signal, when, if, else, process, end,
  },
  morecomment=[l]--
}
%----------------------------------------------------------------------------------------
% XML Coding Language Style
% modified from http://tex.stackexchange.com/questions/10255/xml-syntax-highlighting
%----------------------------------------------------------------------------------------
\lstdefinelanguage{XML}
{
  basicstyle=\ttfamily\footnotesize,
  columns=fullflexible,keepspaces,
  morestring=[s]{"}{"},
  morecomment=[s]{!--}{--},
  commentstyle=\color{darkgreen},
  moredelim=[s][\color{black}]{>}{<},
  moredelim=[s][\color{cyan}]{\ }{=},
  stringstyle=\color{maroon},
  identifierstyle=\color{ceruleanblue}
}
%----------------------------------------------------------------------------------------
% DIFF Coding Language Style
% modified from http://tex.stackexchange.com/questions/50176/highlighting-a-diff-file
%----------------------------------------------------------------------------------------
\lstdefinelanguage{diff}
{
  basicstyle=\ttfamily\footnotesize,
  columns=fullflexible,keepspaces,
  breaklines=true,                            % wrap text
  morecomment=[f][\color{ceruleanblue}]{@@},  % group identifier
  morecomment=[f][\color{red}]-,              % deleted lines
  morecomment=[f][\color{darkgreen}]+,        % added lines
  morecomment=[f][\color{deepmagenta}]{---},  % Diff header lines (must appear after +,-)
  morecomment=[f][\color{deepmagenta}]{+++},
}
%----------------------------------------------------------------------------------------
% Python Coding Language Style
%----------------------------------------------------------------------------------------
\lstdefinelanguage{python}
{
  basicstyle=\ttfamily\footnotesize,
  columns=fullflexible,keepspaces,
  keywordstyle=\color{ceruleanblue},
  commentstyle=\color{darkgreen},
  stringstyle=\color{orange},
  morekeywords={
    print, if, sys, len, from, import, as, open,close, def, main, for, else,
    write, read, range,
  },
  comment=[l]{\#}
}
%----------------------------------------------------------------------------------------
% Fontsize Notes in order from smallest to largest
%----------------------------------------------------------------------------------------
%    \tiny
%    \scriptsize
%    \footnotesize
%    \small
%    \normalsize
%    \large
%    \Large
%    \LARGE
%    \huge
%    \Huge

%----------------------------------------------------------------------------------------

\begin{document}
\maketitle
\thispagestyle{empty}
\newpage

        \begin{center}
        \textit{\textbf{Revision History}}
                \begin{table}[H]
                \label{table:revisions} % Add "[H]" to force placement of table
                        \begin{tabularx}{\textwidth}{|c|X|l|}
                        \hline
                        \rowcolor{blue}
                        \textbf{Revision} & \textbf{Description of Change} & \textbf{Date} \\
                        \hline
                        v1.1 & Initial creation for OpenCPI 1.1 & 3/2017 \\
                        \hline
                        v1.2 & Updated for OpenCPI Release 1.2 & 8/2017 \\
                        \hline
                        v1.4 & Updated for OpenCPI Release 1.4 & 9/2018 \\
                        \hline
                        v1.5 & Updated for OpenCPI Release 1.5 & 4/2019 \\
                        \hline
                        v1.6 & Updated for OpenCPI Release 1.6 & 1/2020 \\
                        \hline
                        \end{tabularx}
                \end{table}
        \end{center}
\newpage

\tableofcontents
\newpage

% How to add a new question / answer:
% \item[Question]~\\
% Answer
\section{General Questions}
% AV-1724
%% \begin{description}[style=nextline]
%% \item[Is the RPM suite a standalone install?]~\\
%% \label{faq:whatis}%
%% Yes, the RPMs distributed by OpenCPI Maintainers incorporates and extends the Free / Open Source Project ``OpenCPI.'' Any OpenCPI installation documents that still exist are for reference and legacy users. All other OpenCPI documentation still applies and should be referenced. Do \textbf{not} attempt to install OpenCPI from source at the same time as the RPM distribution.
%% \end{description}

\begin{description}[style=nextline]
\item[Where can I go for more help?]~\\
\label{faq:halp}%
All documentation is available at \href{https://opencpi.gitlab.io/}{\path{opencpi.gitlab.io}} and there is a public mailing list (with archive) at \href{http://lists.opencpi.org/}{\path{lists.opencpi.org}}.
\end{description}

\section{Install-Specific Questions}
\begin{description}[style=nextline]
% AV-1724
\item[Does it matter what version of CentOS is used?]~\\
Both CentOS~6 (tested using 6.10) and CentOS~7 are supported. Local hardware support (\textit{e.g.} PCIe-based platforms) is officially supported on both OS releases starting with Version 1.1.
CentOS~6, while not officially deprecated, does not get as much testing by the core team, so may have unnoticed issues.

CentOS~7 is now considered a ``rolling release'' so they say, for example, that at a point in
time\footnote{\url{https://wiki.centos.org/About/Product} and \url{https://wiki.centos.org/FAQ/General}'s ``How does CentOS versioning work?''} it is ``CentOS~7 (1810)'' but \textit{not} specifically ``7.6''.
OpenCPI is tested using ``7 (1804)'' which is informally 7.5, so if you are running a version older than 7.5, you need to upgrade the target OS or use the source distribution.
\end{description}

\section{General Usage Problems / Questions}
\begin{description}[style=nextline]
% AV-1724
% AV-4029
\item[Make error: ``*** isim not an available HDL platform.  Stop.'']~\\
Either the Core Project was never built, or it is not properly registered. This is explained in the \textit{Getting Started Guide}.

% Support email 2019-05-02
\item[\code{/opt/opencpi/cdk/centos7/bin/ocpigen: /opt/Xilinx/.../libstdc++.so.6: version `GLIBCXX\_3.4.15' not found (required by /opt/opencpi/cdk/centos7/bin/ocpigen)}]~\\
\label{xilinx-paths}%
If the OpenCPI tools, \textit{e.g.} \path{ocpigen}, are reporting problems with a \path{libstdc++} \textit{within a Xilinx tool path}, that means your environment has imported the Xilinx tool's configuration script, \textit{e.g.} \path{settings64.sh} (\sref{bug:1736}).
If you didn't manually import it, ensure you don't have a line that does elsewhere, \textit{e.g.} in \path{~/.bashrc}.

% 2016-10-19
\item[I am trying to run a demo application with ``ocpirun'' and artifacts are not being found.]~\\
The usual causes of this are:
\begin{itemize}
\setlength\itemsep{0pt}
\item Core Project was not built for the target platform
\begin{itemize}
\item Consult the \textit{Getting Started Guide}
\end{itemize}
\item \path{OCPI_LIBRARY_PATH} was not properly set
\begin{itemize}
\item View the artifacts being checked by adding ``\code{-l 8}'' on the \texttt{ocpirun} command line to increase the logging level
\end{itemize}
\end{itemize}

% AV-3149, 2018-09-12 (1.4)
\item[HDL Workers are failing Unit Tests that passed before 1.4.]~\\
The most likely cause is that``backpressure'' is now automatically asserted by default; see the \textit{Component Development Guide} for details.

% AV-4310 AV-4311 AV-4327
\item[My application's I and Q seem wrong after moving to 1.4.]~\\
See \href{sec:14_iqdata}{below} and the \textit{Release Notes}.

% AV-4310 AV-4311 AV-4327
\label{sec:14_iqdata}
\item[How do I handle \code{iqstream\_protocol}'s I and Q data ordering in HDL workers?]~\\
As noted in the \textit{HDL Development Guide}, when a Protocol contains a Struct Argument, the first Argument Member defined in the Protocol's XML is \textit{always} in the \textit{least significant} bits of the resulting Port. A good example of the importance of this is the scenario where a Port's default data width is overridden (in the OWD) to present all of a Struct's Members in parallel within a single clock cycle. For example, \code{iqstream\_protocol}'s default ordering is a 16-bit interleaved I/Q data: ``$I_{t=0}, Q_{t=0}, I_{t=1}, Q_{t=1}, I_{t=2}, ...$''. However, if the Worker configures its data Ports to be a width of 32 bits, then the I/Q data is presented as a parallel I/Q sample pair with ``I'' in the \textit{lower} 16 bits and ``Q'' in the \textit{upper} 16 bits, \textit{i.e.}:
\begin{center}
$I_{0}=InPort_{0}[15:0]; Q_{0}=InPort_{0}[31:16]\newline
I_{1}=InPort_{1}[15:0]; Q_{1}=InPort_{1}[31:16]\newline
$
\end{center}
A Worker to consult as an example is ``\path{iqstream_max_calculator.hdl}.''

\end{description}

\section{Xilinx-Specific Questions}
\begin{description}[style=nextline]
% AV-1724, 2016-10-15
\item[Are there any other setups I need to perform on the Xilinx Vivado or ISE side?]~\\
No, we abstract away a lot of the requirements if you simply install as described in the \textit{OpenCPI Installation Guide}.

% AV-1736
\label{bug:1736}
Additionally, sourcing the Xilinx setup scripts, \textit{e.g.} ``\path{source /opt/Xilinx/14.7/ISE_DS/settings64.sh}'' or ``\path{source /opt/Xilinx/Vivado/2017.1/settings64.sh}'', can cause other problems (\sref{xilinx-paths}) and \textbf{should not be performed}.

% 2016-10-07
\item[The ZedBoard comes with a license, but it is for the Vivado tools.]~\\
Xilinx's ``WebPack'' Vivado or ISE license is sufficient to do anything with the ZedBoard.

\textit{ISE Note:} As for purchasing, you can ``rollback'' a Vivado license by contacting Xilinx and they will issue you an ISE license with the same expiration with a gentleman's agreement that you won't use both at the same time.
\end{description}

\end{document}
