\iffalse
This file is protected by Copyright. Please refer to the COPYRIGHT file
distributed with this source distribution.

This file is part of OpenCPI <http://www.opencpi.org>

OpenCPI is free software: you can redistribute it and/or modify it under the
terms of the GNU Lesser General Public License as published by the Free Software
Foundation, either version 3 of the License, or (at your option) any later
version.

OpenCPI is distributed in the hope that it will be useful, but WITHOUT ANY
WARRANTY; without even the implied warranty of MERCHANTABILITY or FITNESS FOR A
PARTICULAR PURPOSE. See the GNU Lesser General Public License for more details.

You should have received a copy of the GNU Lesser General Public License along
with this program. If not, see <http://www.gnu.org/licenses/>.
\fi

%----------------------------------------------------------------------------------------
% Required document specific properties
%----------------------------------------------------------------------------------------
\def\docTitle{FPGA Vendor Tools Installation Guide}
\def\snippetpath{snippets}
%----------------------------------------------------------------------------------------
% Global latex header (this must be after document specific properties)
%----------------------------------------------------------------------------------------
\iffalse
This file is protected by Copyright. Please refer to the COPYRIGHT file
distributed with this source distribution.

This file is part of OpenCPI <http://www.opencpi.org>

OpenCPI is free software: you can redistribute it and/or modify it under the
terms of the GNU Lesser General Public License as published by the Free Software
Foundation, either version 3 of the License, or (at your option) any later
version.

OpenCPI is distributed in the hope that it will be useful, but WITHOUT ANY
WARRANTY; without even the implied warranty of MERCHANTABILITY or FITNESS FOR A
PARTICULAR PURPOSE. See the GNU Lesser General Public License for more details.

You should have received a copy of the GNU Lesser General Public License along
with this program. If not, see <http://www.gnu.org/licenses/>.
\fi

% Sets OpenCPI Version used throughout all the docs. This is updated by
% scripts/update-release.sh when a release is being made and must not
% be changed manually.
\def\ocpiversion{v2.2.0}

\documentclass{article}
\author{}  % Force author to be blank
\date{OpenCPI Release:\ \ \ocpiversion}  % Force date to be blank and override date with version
\title{OpenCPI\\\docTitle}  % docTitle must be defined before including this file
%----------------------------------------------------------------------------------------
% Paper size, orientation and margins
%----------------------------------------------------------------------------------------
\usepackage{geometry}
\geometry{
  letterpaper,  % paper type
  portrait,     % text direction
  left=.75in,   % left margin
  top=.75in,    % top margin
  right=.75in,  % right margin
  bottom=.75in  % bottom margin
}
%----------------------------------------------------------------------------------------
% Header/Footer
%----------------------------------------------------------------------------------------
\usepackage{fancyhdr} \pagestyle{fancy}  % required for fancy headers
\renewcommand{\headrulewidth}{0.5pt}
\renewcommand{\footrulewidth}{0.5pt}
\lhead{\small{\docTitle}}
\rhead{\small{OpenCPI}}
%----------------------------------------------------------------------------------------
% Various packages
%----------------------------------------------------------------------------------------
\usepackage{amsmath}
\usepackage[page,toc]{appendix}  % for appendix stuff
\usepackage{enumitem}
\usepackage{graphicx}   % for including pictures by file
\usepackage{hyperref}   % for linking urls and lists
\usepackage{listings}   % for coding language styles
\usepackage{pdflscape}  % for landscape view
\usepackage{pifont}     % for sideways table
\usepackage{ragged2e}   % for justify
\usepackage{rotating}   % for sideways table
\usepackage{scrextend}
\usepackage{setspace}
\usepackage{subfig}
\usepackage{textcomp}
\usepackage[dvipsnames,usenames]{xcolor}  % for color names see https://en.wikibooks.org/wiki/LaTeX/Colors
\usepackage{xstring}
\uchyph=0  % Never hyphenate acronyms like RCC
\renewcommand\_{\textunderscore\allowbreak}  % Allow words to break/newline on underscores
%----------------------------------------------------------------------------------------
% Table packages
%----------------------------------------------------------------------------------------
\usepackage[tableposition=top]{caption}
\usepackage{float}
\floatstyle{plaintop}
\usepackage{longtable}  % for long possibly multi-page tables
\usepackage{multicol}   % for more advanced table layout
\usepackage{multirow}   % for more advanced table layout
\usepackage{tabularx}   % c=center,l=left,r=right,X=fill
% These define tabularx columns "C" and "R" to match "X" but center/right aligned
\newcolumntype{C}{>{\centering\arraybackslash}X}
\newcolumntype{M}[1]{>{\centering\arraybackslash}m{#1}}
\newcolumntype{P}[1]{>{\centering\arraybackslash}p{#1}}
\newcolumntype{R}{>{\raggedleft\arraybackslash}X}
%----------------------------------------------------------------------------------------
% Block Diagram / FSM Drawings
%----------------------------------------------------------------------------------------
\usepackage{tikz}
\usetikzlibrary{arrows,decorations.markings,fit,positioning,shapes}
\usetikzlibrary{automata}  % used for the fsm
\usetikzlibrary{calc}      % for duplicating clients
\usepgfmodule{oo}          % to define a client box
%----------------------------------------------------------------------------------------
% Colors Used
%----------------------------------------------------------------------------------------
\usepackage{colortbl}
\definecolor{blue}{rgb}{.7,.8,.9}
\definecolor{ceruleanblue}{rgb}{0.16, 0.32, 0.75}
\definecolor{cyan}{rgb}{0.0,0.6,0.6}
\definecolor{darkgreen}{rgb}{0,0.6,0}
\definecolor{deepmagenta}{rgb}{0.8, 0.0, 0.8}
\definecolor{maroon}{rgb}{0.5,0,0}
%----------------------------------------------------------------------------------------
% Define where to hyphenate
%----------------------------------------------------------------------------------------
\hyphenation{Cent-OS}
\hyphenation{install-ation}
%----------------------------------------------------------------------------------------
% Define Commands & Rename Commands
%----------------------------------------------------------------------------------------
\newcommand{\code}[1]{\texttt{#1}}  % For inline code snippet or command line
\newcommand{\sref}[1]{Section~\ref{#1}}  % To quickly reference a section
\newcommand{\todo}[1]{\textcolor{red}{TODO: #1}\PackageWarning{TODO:}{#1}}  % To do notes
\renewcommand{\contentsname}{Table of Contents}
\renewcommand{\listfigurename}{List of Figures}
\renewcommand{\listtablename}{List of Tables}

% This gives a link to gitlab.io document. By default, it outputs the filename.
% You can optionally change the link, e.g.
% \githubio{FPGA\_Vendor\_Tools\_Installation\_Guide.pdf} vs.
% \githubio[\textit{FPGA Vendor Tools Installation Guide}]{FPGA\_Vendor\_Tools\_Installation\_Guide.pdf}
% or if you want the raw ugly URL to come out, \githubioURL{FPGA_Vendor_Tools_Installation_Guide.pdf}
\newcommand{\githubio}[2][]{% The default is for FIRST param!
\href{http://opencpi.gitlab.io/releases/\ocpiversion/docs/#2}{\ifthenelse{\equal{#1}{}}{\texttt{#2}}{#1}}}
\newcommand{\gitlabcom}[2][]{% The default is for FIRST param!
\href{http://gitlab.com/opencpi/#2}{\ifthenelse{\equal{#1}{}}{\texttt{#2}}{#1}}}
\newcommand{\githubioURL}[1]{\url{http://opencpi.gitlab.io/releases/\ocpiversion/docs/#1}}
% Lastly, if you want a SINGLE leading path stripped, e.g. assets/X.pdf => X.pdf:
\newcommand{\githubioFlat}[1]{%
\StrBehind{#1}{/}[\den]%
\href{http://opencpi.gitlab.io/releases/\ocpiversion/docs/#1}{\texttt{\den}}%
}
%----------------------------------------------------------------------------------------
% VHDL Coding Language Style
% modified from: http://latex-community.org/forum/viewtopic.php?f=44&t=22076
%----------------------------------------------------------------------------------------
\lstdefinelanguage{VHDL}
{
  basicstyle=\ttfamily\footnotesize,
  columns=fullflexible,keepspaces,  % https://tex.stackexchange.com/a/46695/87531
  keywordstyle=\color{ceruleanblue},
  commentstyle=\color{darkgreen},
  morekeywords={
    library, use, all, entity, is, port, in, out, end, architecture, of,
    begin, and, signal, when, if, else, process, end,
  },
  morecomment=[l]--
}
%----------------------------------------------------------------------------------------
% XML Coding Language Style
% modified from http://tex.stackexchange.com/questions/10255/xml-syntax-highlighting
%----------------------------------------------------------------------------------------
\lstdefinelanguage{XML}
{
  basicstyle=\ttfamily\footnotesize,
  columns=fullflexible,keepspaces,
  morestring=[s]{"}{"},
  morecomment=[s]{!--}{--},
  commentstyle=\color{darkgreen},
  moredelim=[s][\color{black}]{>}{<},
  moredelim=[s][\color{cyan}]{\ }{=},
  stringstyle=\color{maroon},
  identifierstyle=\color{ceruleanblue}
}
%----------------------------------------------------------------------------------------
% DIFF Coding Language Style
% modified from http://tex.stackexchange.com/questions/50176/highlighting-a-diff-file
%----------------------------------------------------------------------------------------
\lstdefinelanguage{diff}
{
  basicstyle=\ttfamily\footnotesize,
  columns=fullflexible,keepspaces,
  breaklines=true,                            % wrap text
  morecomment=[f][\color{ceruleanblue}]{@@},  % group identifier
  morecomment=[f][\color{red}]-,              % deleted lines
  morecomment=[f][\color{darkgreen}]+,        % added lines
  morecomment=[f][\color{deepmagenta}]{---},  % Diff header lines (must appear after +,-)
  morecomment=[f][\color{deepmagenta}]{+++},
}
%----------------------------------------------------------------------------------------
% Python Coding Language Style
%----------------------------------------------------------------------------------------
\lstdefinelanguage{python}
{
  basicstyle=\ttfamily\footnotesize,
  columns=fullflexible,keepspaces,
  keywordstyle=\color{ceruleanblue},
  commentstyle=\color{darkgreen},
  stringstyle=\color{orange},
  morekeywords={
    print, if, sys, len, from, import, as, open,close, def, main, for, else,
    write, read, range,
  },
  comment=[l]{\#}
}
%----------------------------------------------------------------------------------------
% Fontsize Notes in order from smallest to largest
%----------------------------------------------------------------------------------------
%    \tiny
%    \scriptsize
%    \footnotesize
%    \small
%    \normalsize
%    \large
%    \Large
%    \LARGE
%    \huge
%    \Huge

%----------------------------------------------------------------------------------------

\begin{document}
\maketitle
\thispagestyle{empty}
\newpage

	\begin{center}
	\textit{\textbf{Revision History}}
		\begin{table}[H]
		\label{table:revisions} % Add "[H]" to force placement of table
			\begin{tabularx}{\textwidth}{|c|X|l|}
			\hline
			\rowcolor{blue}
			\textbf{Revision} & \textbf{Description of Change} & \textbf{Date} \\
			\hline
			v1.1 & Initial Release & 3/2017 \\
			\hline
			v1.2 & Updated for Release 1.2 & 8/2017 \\
			\hline
			v1.4 & Updated for Release 1.4 & 9/2018 \\
			\hline
			v1.5 & Updated for Release 1.5 & 4/2019 \\
			\hline
			v1.6 & Updated for Release 1.6 and deprecated & 1/2020 \\
			\hline
			\end{tabularx}
		\end{table}
	\end{center}

\newpage

\tableofcontents

\newpage

\section{References}

	This document assumes a basic understanding of the Linux command line (or ``shell'') environment. A working knowledge of OpenCPI is required for understanding what vendor tools are necessary to perform various operations. However, no OpenCPI knowledge is required to perform the toolset installation and configuration herein. The reference(s) in Table \ref{table:references} can be used as an overview of OpenCPI and may prove useful.  In particular, the sections in the \textit{OpenCPI Installation Guide} referring to FPGA tools installations supercede this document.  \textbf{This document is deprecated but not removed since there is some content not yet integrated into that other document.}
\def\refcapbottom{}
\iffalse
This file is protected by Copyright. Please refer to the COPYRIGHT file
distributed with this source distribution.

This file is part of OpenCPI <http://www.opencpi.org>

OpenCPI is free software: you can redistribute it and/or modify it under the
terms of the GNU Lesser General Public License as published by the Free Software
Foundation, either version 3 of the License, or (at your option) any later
version.

OpenCPI is distributed in the hope that it will be useful, but WITHOUT ANY
WARRANTY; without even the implied warranty of MERCHANTABILITY or FITNESS FOR A
PARTICULAR PURPOSE. See the GNU Lesser General Public License for more details.

You should have received a copy of the GNU Lesser General Public License along
with this program. If not, see <http://www.gnu.org/licenses/>.
\fi

% This snippet creates the "References" table labeled "table:references"
% It creates three columns: Name, Publisher, Link and then inserts default documents
%
% To skip these defaults, define macros named
% refskipgs to skip "Getting Started"
% refskipig to skip "Installation Guide"
% refskipac to skip "Acronyms and Definitions"
% refskipocpiov to skip "OpenCPI Overview"
%
% See RPM_Installation_Guide.tex for examples
%
% After the defaults, it optionally inserts the "myreferences" macro that
% you defined elsewhere (you put hlines above all lines)
%
% If you want the \caption on the bottom, define "refcapbottom"
\begin{center}
\renewcommand*\footnoterule{} % Remove separator line from footnote
\renewcommand{\thempfootnote}{\arabic{mpfootnote}} % Use Arabic numbers (or can't reuse)
\begin{minipage}{0.9\textwidth}
  \begin{table}[H]
\ifx\refcapbottom\undefined
  \caption {References}
  \label{table:references}
\fi
  \begin{tabularx}{\textwidth}{|C|C|}
    \hline
    \rowcolor{blue}
    \textbf{Title} & \textbf{Link} \\
\ifx\refskipocpiov\undefined
    \hline
    OpenCPI User Guide & \githubio{OpenCPI\_User.pdf} \\
\fi
\ifx\refskipac\undefined
    \hline
    Acronyms and Definitions & \githubio{Acronyms\_and\_Definitions.pdf} \\
\fi
\ifx\refskipig\undefined
    \hline
    OpenCPI Installation Guide & \githubio{OpenCPI\_Installation.pdf} \\
\fi
\ifx\myreferences\undefined
\else
    \myreferences
\fi
    \hline
  \end{tabularx}
\ifx\refcapbottom\undefined
\else
  \caption {References}
  \label{table:references}
\fi
  \end{table}
\end{minipage}
\end{center}


\begin{flushleft}
\begin{landscape}
\section{Supported Vendor Tools and OpenCPI functionality}
\label{sec:doc_overview}
OpenCPI utilizes third party FPGA vendor tools to perform various operations, such as, building bitstreams or, for certain platforms, loading  bitstreams into FPGAs. Table~\ref{table:tool-support}
describes the OpenCPI functionality that is provided by each supported vendor tool with regards to building bitstreams (hardware or simulation), loading of bitstreams, or running a simulation. Since licensing of vendor tool plays a critical role in build for certain target devices and usage of a given tool, its relationship is also listed.\\

Note that Quartus Standard and Quartus Pro are \textit{different tools}. These two tools support different sets of devices and users should consult Intel's documentation for more information. Older versions of some FPGA tools have been supported by OpenCPI but are not actively regression tested, such as, Vivado 2015.4 and Quartus Standard Edition 15.1. \\
%Commented this out because there may be other Vivado functionality that requires a non-webpack license.
%Maximum functionality for OpenCPI's currently supported platforms is provided by a WebPACK licensed Xilinx Vivado installation, a non-WebPACK licensed Xilinx ISE installation and a licensed Quartus installation.

\begin{center}
	\renewcommand*\footnoterule{} % Remove separator line from footnote
	\renewcommand{\thempfootnote}{\arabic{mpfootnote}} % Use Arabic numbers (or can't reuse)
	\begin{table}[H]
		\newcolumntype{Z}{>{\raggedright\arraybackslash}X}
		\def\arraystretch{1.5}
		\begin{tabularx}{1.35\textwidth}{| p{2.6cm} | p{5cm} | p{1.8cm} | p{2.9cm} | p{3.6cm} | p{2.0cm} | Z |}
			\hline
			\rowcolor{blue}
			\multicolumn{1}{|l|}{\textbf{OpenCPI + \{Tool\}}} & \multicolumn{1}{|c|}{\textbf{Version/License}} & \textbf{Supported \newline simulators} & \textbf{Load bitstreams onto} & \textbf{Run applications on these platforms} & \textbf{Build bitstreams for} & \textbf{Build software for} \\
			\hline
			\multicolumn{2}{|l|}{\textbf{OpenCPI \textit{ONLY} (without vendor tools)}} & & Zynq-7000 & Zynq-7000 based$^1$, x86-only & & x86 \\
			\hline
			\multirow{3}{*}{\textbf{Xilinx Vivado}} & 2017.1 with WebPACK License & \code{xsim} & & & Zynq-7000$^2$ & \\ \cline{2-7}
			& 2013.4 (SDK only)$^4$ & & & & & Zynq-7000 ARM \\ \cline{2-7}
			& 2017.1 \textit{and} 2013.4 SDK with WebPACK License & \code{xsim} & & & Zynq-7000$^2$ &  Zynq-7000 ARM \\
			\hline
			\multicolumn{2}{|l|}{\textbf{Xilinx LabTools 14.7}} & & ML605 & x86/ML605 & & \\
			\hline
			\multirow{2}{*}{\textbf{Xilinx ISE 14.7}} & WebPACK License & \code{isim} & ML605 & x86/ML605 & Zynq-7000$^2$ & Zynq-7000 ARM \\ \cline{2-7}
			& Full License & \code{isim} & ML605 & x86/ML605 & Zynq-7000, ML605 & Zynq-7000 ARM \\
			\hline
			\multicolumn{2}{|l|}{\textbf{Intel Quartus Standard 17.1 with License}}  & & ALST4 & x86/ALST4 & ALST4 & \\
			\hline
			\multicolumn{2}{|l|}{\textbf{Intel Quartus Pro Edition 17.0.2 with License}}  & & & & arria10soc$^3$ & \\
			\hline
			\multicolumn{2}{|l|}{\textbf{Mentor Graphics ModelSim DE 10.6e with License}}  & modelsim &  & & modelsim & \\
			\hline
		\end{tabularx}\newline

		\footnotesize{$^1$``Zynq-7000 based'' platform includes both a Zynq-7000's FPGA and ARM PS. The usage of ``Zynq'' or ``Zynq-based'' does not imply Zynq UltraScale+ devices.}\\
		\footnotesize{$^2$Building bitstreams with a WebPACK license is limited to certain Zynq parts. Refer to the vendor's documentation for further information.}\\
		\footnotesize{$^3$While there are currently no OpenCPI Board Support Packages developed for Quartus Pro, HDL workers can be built targeting the \textit{arria10soc} device family.}\\
		\footnotesize{$^4$The relationship between the Vivado Design Edition and SDK is discussed in \ref{sec:viv_intro}.}\\
		\caption {Added-value of Vendor Tools to OpenCPI}
		\label{table:tool-support} % Add "[H]" to force placement of table
	\end{table}
\end{center}
\end{landscape}

\section{Xilinx Toolset Installation and Configuration}
\subsection{Xilinx Vivado Installation in CentOS~6/7}
\label{sec:viv_intro}
\begin{flushleft}
As described in Table~\ref{table:tool-support}, building for OpenCPI board support packages (BSPs) which are Xilinx FPGA-based requires various Xilinx FPGA tools to be installed. \\ \medskip
In the case of Zynq-7000 based OpenCPI BSPs, the required tools are Vivado 2017.1 \textit{and} Vivado 2013.4's SDK, where the 2013.4 SDK is necessary because OpenCPI's ``\code{xilinx13\_3}'' and ``\code{xilinx13\_4}'' software platforms require an SDK with matching glibc/glibc++ versions. An SDK meeting this requirement can be found explicitly in either ISE 14.7 or Vivado 2013.4 SDK. For more information on this requirement you can reference the README for the \code{xilinx13\_3} software platform. This is located in the core project (\textit{e.g.}: \path{<core-project>/rcc/platforms/xilinx13_3}).\\ \medskip
In the case of the ML605 development board (PCIe), only ISE v14.7 is required, because the host's gcc-compiler will be used.

\subsubsection{Xilinx Vivado 2017.1 Installation in CentOS~6/7}
\label{sec:viv}
\begin{enumerate}
\item A Xilinx account is required for this step. Download the Vivado 2017.1 installation files from Xilinx's download site:
\url{https://www.xilinx.com/support/download/index.html/content/xilinx/en/downloadNav/vivado-design-tools/2017-1.html}.
\begin{figure}[ht]
	\centerline{\includegraphics[scale=0.6]{figures/xilinx_vivado_2017_download}}
	\caption{Xilinx Vivado 2017.1 Download}
\end{figure}
\item If installing Xilinx tools in a permission-restricted directory, you may need to change the umask temporarily:\newline
\code{\% sudo su -}\newline
\code{\% umask 0002}
\item Extract the tarball:\newline
\code{\% tar -xf Xilinx\_Vivado\_SDK\_2017.1\_0415\_1.tar.gz}
\item Enter the resulting directory and run the installer:\newline
\code{\% cd Xilinx\_Vivado\_SDK\_2017.1\_0415\_1}\newline
\code{\% ./xsetup}\newline
\pagebreak
\item Step through the installation process. Refer to the images below when applicable.
\begin{figure}[H]
	\centerline{\includegraphics[scale=0.4]{figures/xilinx_vivado_2017_install}}
	\caption{Xilinx Vivado Installer}
\end{figure}
We do not direct you to acquire a license, but if you do not already have one, you will need to select ``Acquire or Manage a License Key'' in the image below.
\begin{figure}[H]
	\centerline{\includegraphics[scale=0.4]{figures/xilinx_vivado_2017_choose_installation}}
	\caption{Xilinx Vivado Installation Choice}
\end{figure}
\pagebreak
Take note of the installation directory chosen (e.g. \code{/opt/Xilinx}) as well as the Vivado version (e.g. \code{2017.1}) for later use.
\begin{figure}[H]
	\centerline{\includegraphics[scale=0.4]{figures/xilinx_vivado_2017_install_location}}
	\caption{Xilinx Vivado Install Location}
\end{figure}
\end{enumerate}
% TODO: Make this next section a snippet because it is copypasted a lot
\subsubsection{OpenCPI Considerations}
\begin{enumerate}
\item Note that sourcing the ``\verb+<Vivado-install-dir>/Vivado/<Vivado-version>/settings64.sh+'' script will interfere with OpenCPI's environment setup. Accordingly, it is \textit{highly} recommended to always source these scripts and execute any follow-on commands in a \textit{separate terminal}.
\item To use OpenCPI with any Xilinx Vivado installation, it is required to set the following environment variables before running OpenCPI commands. Note that each of the following \code{export} statements is only necessary under the following conditions:
\begin{itemize}
\item When using a non-default installation location (i.e. anything other than \path{/opt/Xilinx})
\item When Vivado \textit{and} ISE are both being used and are installed in different locations
\item Or when multiple versions of Vivado are installed and you wish to use a version other than the newest.
\end{itemize}

\subitem \code{\% export OCPI\_XILINX\_VIVADO\_DIR=<Vivado-install-dir>}
\subitem \code{\% export OCPI\_XILINX\_VIVADO\_VERSION=<Vivado-version>}

\end{enumerate}
If OpenCPI has been installed prior to the Vivado installation, and it is desired to make the aforementioned environment variables set automatically upon login for all users, the variables should be added in \code{/opt/opencpi/cdk/env.d/xilinx.sh}. Logging out and logging back into the user account will apply said variables.
\subsubsection{Xilinx Vivado 2013.4 SDK Only Installation in CentOS~6/7}
\label{sec:viv_sdk}
\begin{enumerate}
\item A Xilinx account is required for this step. Download the Vivado 2013.4 Standalone SDK installation files from Xilinx's download site:
\url{https://www.xilinx.com/support/download/index.html/content/xilinx/en/downloadNav/vivado-design-tools/archive.html}. Navigate to ``2013.4'' $\rightarrow$ ``Software Development Kit''.

\begin{figure}[H]
	\centerline{\includegraphics[scale=0.4]{figures/xilinx_vivado_sdk_download}}
	\caption{Xilinx Vivado 2013.4 SDK Download}
\end{figure}
\item If installing Xilinx tools in a permission-restricted directory, you may need to change the umask temporarily:\newline
\code{\% sudo su -}\newline
\code{\% umask 0002}
\item Extract the tarball:\newline
\code{\% tar -xf Xilinx\_SDK\_2013.4\_1210\_1.tar}
\item Enter the resulting directory and run the installer:\newline
\code{\% cd Xilinx\_SDK\_2013.4\_1210\_1}\newline
\code{\% ./xsetup}\newline
\pagebreak
\item Step through the installation process. Refer to the images below when applicable.
\begin{figure}[H]
	\centerline{\includegraphics[scale=0.4]{figures/xilinx_vivado_sdk_install}}
	\caption{Xilinx Vivado SDK Installer}
\end{figure}

\begin{figure}[H]
	\centerline{\includegraphics[scale=0.4]{figures/xilinx_vivado_sdk_choose_installation}}
	\caption{Xilinx Vivado SDK Installation Choice}
\end{figure}
\pagebreak
Take note of the installation directory chosen (e.g. \code{/opt/Xilinx}) as well as the Vivado version (e.g. \code{2013.4}) for later use.
\begin{figure}[H]
	\centerline{\includegraphics[scale=0.4]{figures/xilinx_vivado_sdk_install_location}}
	\caption{Xilinx Vivado SDK Install Location}
\end{figure}
\end{enumerate}
\end{flushleft}

\subsection{Xilinx ISE 14.7 Installation in CentOS~6/7}
\label{sec:ise}
\begin{flushleft}
\begin{enumerate}
\item A Xilinx account is required for this step. Download the ISE 14.7 installation files from Xilinx's download site:
\url{https://www.xilinx.com/support/download/index.html/content/xilinx/en/downloadNav/design-tools.html}.
\begin{figure}[ht]
	\centerline{\includegraphics[scale=0.4]{figures/xilinx_ise_download}}
	\caption{Xilinx ISE Download}
\end{figure}
\item If installing Xilinx tools in a permission-restricted directory, you may need to change the umask temporarily:\newline
\code{\% sudo su -}\newline
\code{\% umask 0002}
\item Extract the tarball:\newline
\code{\% tar -xf Xilinx\_ISE\_DS\_14.7\_1015\_1.tar}
\item Enter the resulting directory and run the installer:\newline
\code{\% cd Xilinx\_ISE\_DS\_14.7\_1015\_1}\newline
\code{\% ./xsetup}\newline
\pagebreak
\item Run through the installation process. Refer to the images below when applicable. Note that the checkbox for cable drivers is left unchecked. Cable driver installation, if necessary, should be handled after this installation is complete. See section \ref{sec:cable} for more information.
\begin{figure}[H]
	\centerline{\includegraphics[scale=0.4]{figures/xilinx_ise_install}}
	\caption{Xilinx ISE Installer}
\end{figure}

\begin{figure}[H]
	\centerline{\includegraphics[scale=0.4]{figures/xilinx_labtools_choose_installation}}
	\caption{Xilinx ISE Installation Choice}
\end{figure}
\pagebreak
Take note of the installation directory chosen (e.g. \code{/opt/Xilinx}) as well as the LabTools version (e.g. \code{14.7}) for later use.
\begin{figure}[H]
	\centerline{\includegraphics[scale=0.4]{figures/xilinx_ise_install_location}}
	\caption{Xilinx ISE Install Location}
\end{figure}
\end{enumerate}


\subsubsection{OpenCPI Considerations}
\label{sec:iseav}
\begin{enumerate}
\item Note that sourcing the ``\verb+<ISE-install-dir>/<version>/LabTools/settings64.sh+'' or ``\verb+<ISE-install-dir>/<version>/LabTools/settings32.sh+'' scripts will interfere with OpenCPI's environment setup. Accordingly, it is \textit{highly } recommended to always source these scripts and execute any follow-on commands in a \textit{separate terminal}.
\item To use OpenCPI with any Xilinx ISE or LabTools installation,  it is required to set the following environment variables before running OpenCPI commands. Note that each of the following \code{export} statements is only necessary under the following conditions:
\begin{itemize}
\item When using a non-default installation location (i.e. anything other than \path{/opt/Xilinx})
\item Non-default version (i.e. anything other than \path{14.7}) of the tools were used.
\end{itemize}

\subitem If only one of Xilinx LabTools or ISE is installed,
\subsubitem \code{\% export OCPI\_XILINX\_DIR=<ISE-or-LabTools-install-dir>}
\subsubitem \code{\% export OCPI\_XILINX\_VERSION=<ISE-or-LabTools-version>}
\subitem If Xilinx LabTools and ISE are the same version and installed in the same directory,
\subsubitem \code{\% export OCPI\_XILINX\_DIR=<ISE-and-LabTools-install-dir>}
\subsubitem \code{\% export OCPI\_XILINX\_VERSION=<ISE-and-LabTools-version>}
\subitem If Xilinx LabTools and ISE are the same version and are installed in different directories,
\subsubitem \code{\% export OCPI\_XILINX\_DIR=<ISE-install-dir>}
\subsubitem \code{\% export OCPI\_XILINX\_LAB\_TOOLS\_DIR=<LabTools-install-dir>}
\subsubitem \code{\% export OCPI\_XILINX\_VERSION=<ISE-and-LabTools-version>}
\subitem If Xilinx LabTools and ISE are different versions (LabTools will be ignored),
\subsubitem \code{\% export OCPI\_XILINX\_DIR=<ISE-install-dir>}
\subsubitem \code{\% export OCPI\_XILINX\_VERSION=<ISE-version>}
\end{enumerate}

If OpenCPI has been installed prior to the ISE installation, and it is desired to make the aforementioned environment variables set automatically upon login for all users, the variables should be added in \code{/opt/opencpi/cdk/env.d/xilinx.sh}. Logging out and logging back into the user account will apply said variables.
\end{flushleft}

\subsection{Xilinx LabTools 14.7 Installation in CentOS~6/7}
\label{sec:labtools}
\begin{flushleft}
\begin{enumerate}
\item A Xilinx account is required for this step. Download the LabTools 14.7 installation files from Xilinx's download site:
\url{https://www.xilinx.com/support/download/index.html/content/xilinx/en/downloadNav/design-tools.html}.
\begin{figure}[H]
	\centerline{\includegraphics[scale=0.4]{figures/xilinx_labtools_download}}
	\caption{Xilinx LabTools Download}
\end{figure}
\item If installing Xilinx tools in a permission-restricted directory, you may need to change the umask temporarily:\newline
\code{\% sudo su -}\newline
\code{\% umask 0002}
\item Extract the tarball:\newline
\code{\% tar -xf Xilinx\_LabTools\_14.7\_1015\_1.tar}
\item Enter the resulting directory and run the installer:\newline
\code{\% cd Xilinx\_LabTools\_14.7\_1015\_1}\newline
\code{\% ./xsetup}\newline
\pagebreak

\item Step through the installation process. Refer to the images below when applicable. Note that the checkbox for cable drivers is left unchecked. Cable driver installation, if necessary, should be handled after this installation is complete. See section \ref{sec:cable} for more information.
\begin{figure}[H]
	\centerline{\includegraphics[scale=0.4]{figures/xilinx_labtools_install}}
	\caption{Xilinx LabTools Installer}
\end{figure}
\begin{figure}[H]
	\centerline{\includegraphics[scale=0.4]{figures/xilinx_labtools_choose_installation}}
	\caption{Xilinx LabTools Installation Choice}
\end{figure}
\pagebreak
Take note of the installation directory chosen (e.g. \code{/opt/Xilinx}) as well as the LabTools version (e.g. \code{14.7}) for later use.
\begin{figure}[H]
	\centerline{\includegraphics[scale=0.4]{figures/xilinx_labtools_install_location}}
	\caption{Xilinx LabTools Install Location}
\end{figure}
\end{enumerate}

\subsubsection{OpenCPI Considerations}
\begin{enumerate}
\item Note that sourcing the ``\verb+<LabTools-install-dir>/<version>/LabTools/settings64.sh+'' or ``\verb+<LabTools-install-dir>/<version>/LabTools/settings32.sh+'' scripts will interfere with OpenCPI's environment setup. Accordingly, it is \textit{highly} recommended to always source these scripts and execute any follow-on commands in a \textit{separate terminal}.
\item To use OpenCPI with any Xilinx ISE or LabTools installation, it is required to set the environment variables according to Section \ref{sec:iseav} before running OpenCPI commands.
\end{enumerate}

\end{flushleft}
\pagebreak

\end{flushleft}

\subsection{Xilinx Toolset Licensing}
\label{xilinx}
A license, either WebPACK or non-WebPACK, is required for Xilinx Vivado and Xilinx ISE, however the Xilinx LabTools does not require a license.

\subsubsection{Generate and download a license file from Xilinx}
\begin{enumerate}
\item The following screenshots show is an example of Xilinx's license website with a ISE WebPACK license selected. Refer to \ref{sec:doc_overview} to determine which license is necessary. To generate a license, navigate to \url{http://www.xilinx.com/getlicense} and login (or create an account). Generate a license file:

\begin{figure}[H]
	\centerline{\includegraphics[scale=0.5]{./figures/xilinx_license_gen.jpg}}
	\caption{Generate Xilinx license file}
\end{figure}

\item Download the file and move it to the intended location:

\begin{figure}[H]
	\centerline{\includegraphics[scale=0.5]{./figures/xilinx_license_download.jpg}}
	\caption{Download Xilinx license file}
\end{figure}
\end{enumerate}

\subsubsection{Load license into Vivado}
\begin{enumerate}
\item In a terminal, run ``\verb+source <Vivado-install-dir>/Vivado/<version>/settings64.sh+''.
\item Open up the license manager and load the downloaded license. The license manager can be launched either from the Vivado GUI, or from the command line by running: \\\verb+sudo <Vivado-install-dir>/Vivado/<version>/bin/vlm+\newline

Here, you can either navigate to ``Load License'' and load a copy of the license file, or you can enter the license search paths via ``Manage License Search Paths''.

\begin{figure}[H]
	\centerline{\includegraphics[scale=0.5]{./figures/xilinx_vivado_license_load}}
	\caption{Load Xilinx Vivado license file}
\end{figure}
\end{enumerate}

\subsubsection{Load license into ISE}
\begin{enumerate}
\item In a terminal, run ``\verb+source <ISE-install-dir>/<version>/ISE_DS/settings64.sh+'' (or settings32.sh if the system has a 32-bit architecture).
\item Open up the license manager and load the downloaded license. The license manager can either be launched from the ISE GUI, or launched from the command line by running: \\\verb+sudo <ISE-or-LabTools-install-dir>/<version>/ISE_DS/common/bin/lin[64]/xlcm+

\begin{figure}[H]
	\centerline{\includegraphics[scale=0.5]{./figures/xilinx_license_load.jpg}}
	\caption{Load Xilinx ISE license file}
\end{figure}
\end{enumerate}

\subsubsection{Note on node-locked licenses in CentOS~7}
If using a Xilinx node-locked license under CentOS~7, see \href{https://access.redhat.com/documentation/en-US/Red_Hat_Enterprise_Linux/7/html/Networking_Guide/sec-Disabling_Consistent_Network_Device_Naming.html}{the Red Hat Networking Guide} to revert to the
\texttt{eth\textit{N}} naming convention.

\subsubsection{OpenCPI Considerations}
Note that sourcing the ``\verb+settings64.sh+'' or ``\verb+settings32.sh+'' scripts will interfere with OpenCPI's environment setup. Accordingly, it is \textit{highly} recommended to always source these scripts and execute any follow-on commands in a \textit{separate terminal}. \medskip\newline
To enable a license for use by OpenCPI, the OpenCPI environment variable which supports locating the Xilinx license listing (file or server) must be configured. Edit the \verb+/opt/opencpi/cdk/env.d/xilinx.sh+ to support either a license file or server: 
\begin{itemize}
\item license file: 
\\ \verb+export OCPI_XILINX_LICENSE_FILE=<PATH_TO_LIC>+
\item license server:
\\ \verb+export OCPI_XILINX_LICENSE_FILE=<port>@<server.ip.addr>+
\end{itemize}
If the \verb+OCPI_XILINX_LICENSE_FILE+ environment variable is not set, the license is assumed to be in one of the following locations:
\begin{itemize}
\item /opt/Xilinx/Xilinx-License.lic
\item /opt/Xilinx/Vivado/Xilinx-License.lic
\end{itemize}
Alternatively, if using a floating license server, it is possible to set to the license server and Xilinx's environment variable, which will allow use of a local license, e.g. a local WebPACK license, by default and the served floating license when WebPACK license is not sufficient.\footnote{See Xilinx ``\href{https://www.xilinx.com/support/answers/42507.html}{AR\# 42507: What are the search order and locations...}'' and ``\href{https://www.xilinx.com/support/answers/44024.html}{AR\# 44024: If a feature is licensed in multiple locations...}''} \medskip
\begin{itemize}
\item license server and local license:
\\ \verb+export OCPI_XILINX_LICENSE_FILE=<port>@<server.ip.addr>+ \\ \verb+export XILINXD_LICENSE_FILE=<PATH_TO_LOCAL_LIC>+
\end{itemize}


\subsection{Xilinx Cable Driver Installation in CentOS~6/7}
\label{sec:cable}
This section is a collection of notes or links that have been gathered for the installation or verification of Xilinx cable drivers for Vivado and ISE. However, it is not intended to an exhaustive list of instructions.
\subsubsection{Vivado}
\begin{flushleft}
The steps herein are a slightly modified subset of those outlined in \url{https://www.xilinx.com/support/answers/66440.html}.
\end{flushleft}
\begin{enumerate}
\item Run the following command : \code{ls -al /etc/udev/rules.d}
\item Check if the following two files are present : \code{52-digilent-usb.rules 52-xilinx-pcusb.rules}
\item If the files above are not present, run the installer (\textit{it is important to have the JTAG cable unplugged while you perform the installation}):\\
	\code{cd <YOUR\_XILINX\_INSTALL>/data/xicom/cable\_drivers/<lin64 or lin32>/install\_script/install\_drivers;} \\
	\code{./install\_drivers;}
\end{enumerate}

\subsubsection{ISE}
Verify that the udev rules have been created by the Xilinx cable driver installation scripts. This may require some modification to the xusbdfwu.rules file
\begin{enumerate}
\item Run the following command : \code{ls -al /etc/udev/rules.d}
\item Check if the following file is present : \code{xusbdfwu.rules}
\item If the file is present, go to step 5. If the files above are not present, open the \code{setup\_pcusb} script and change line 26 from \code{TP\_USE\_UDEV="0"} to \code{TP\_USE\_UDEV="1"}
\item Rerun the \code{setup\_pcusb} installation script
\item \code{xusbdfwu.rules} should now be present in \code{ls -al /etc/udev/rules.d}. Open the file and change (if necessary)\newline
\code{SYSFS} to \code{ATTRS}\newline
\code{BUS} to \code{SUBSYSTEM}\newline
\code{\$TEMPNODE} to \code{\$tempnode}
\item Reload the udev rules by typing \code{udevadm control --reload-rules}
\end{enumerate}
\subsubsection{Testing Cable Driver Installation}
\textbf{Vivado}
\begin{flushleft}
After installing the cable driver as previously discussed, the Xilinx JTAG pod's LED may still not illumniate (Amber or Green).  It has been observed that by attempting to establish a connection to the pod using the Vivado tools, only then will the pod be discovered and correctly illuminate it's LED. If after the cable driver has been load and the JTAG pod's LED is off (while connected to the host), perform the following steps to force pod discovery: \medskip\newline
\code{\$ cd /opt/Xilinx/Vivado/2017.1} \\
\code{\$ . ./settings64.sh} \\\medskip
Once the environment has been configured, launch the Vivado IDE and use the Hardware Manager to scan for JTAG pod. The expected result is for the pod to be recognized by the tools and its LED to illuminate Amber if its JTAG connector is not powered, or Green if the JTAG connect is powered.\newline
(While this has not been confirmed, it is believed that some host system environments prevent non-interactive driver accesses.)\medskip
\end{flushleft}
\textbf{ISE}
\begin{flushleft}
To verify successful cable driver installation, you can run the following:\medskip\newline
\code{\$ cd /opt/Xilinx/14.7/ISE\_DS} \\
\code{\$ . ./settings64.sh} \\
\code{\$ cd ~} \\
\code{\$ echo listusbcables | impact -batch} \\\medskip
If the cable driver is successfully installed, ``\code{Using libusb.}'' will be included in the text printed to the screen.
%
%This section assumes you have completed the cable driver installation and you have OpenCPI installed.
%
%To verify successful cable driver installation, you can run the following:\newline\newline \code{/opt/opencpi/cdk/scripts/probeJtag}\newline\newline This script uses \code{impact} and enumerates the JTAG device chain. On success, it will print information such as the USB port, device serial number, and target part:
% \begin{figure}[H]
%	\centerline{\includegraphics[scale=0.7]{./figures/probeJtag_success.jpg}}
%	\caption{Successful probeJtag}
%\end{figure}
%
%If you have an OpenCPI bitstream ready to be deployed onto your target platform, you can follow these steps to test bitstream loading:
%\begin{enumerate}
%\item Set OCPI\_PROJECT\_PATH to the project containing the platform of interest. For example, to load bitstreams to the ML605, you will need to run:\\
%\code{export OCPI\_PROJECT\_PATH=\$OCPI\_PROJECT\_PATH:<path-to-baseproject>}
%
%\item Run ``\code{ocpihdl search}'' to determine the device of interest (i.e. PCI:XXXX:XX:XX.X for PCI platforms).
%\item Load a bitstream via USB-JTAG using ocpihdl:\newline
% \code{ocpihdl load -d PCI:XXXX:XX:XX.X <path-to-bitstream>}
%\end{enumerate}
%
\end{flushleft}
\pagebreak

\newpage

\section{Intel Quartus Toolset Installation and Configuration}
\subsection{Intel Quartus Prime Standard Edition 17.1 Installation in CentOS~7}
\begin{flushleft}
\begin{enumerate}
	\item Download the Quartus Prime Standard Edition 17.1 installation files from Altera's download site: \url{https://www.intel.com/content/www/us/en/programmable/downloads/download-center.html}. Choose \verb+Standard Edition 17.1+ and either choose the ``Complete Download'', or the ``Multiple File Download'' (for this option, make sure to download the device packages of interest). An \verb+Intel Customer+ account will be required.
\item If installing Quartus tools in a permission-restricted directory, you may need to change the umask temporarily:\newline
\code{\% sudo su -}\newline
\code{\% umask 0002}
\item Extract the tarball:\newline
\code{\% tar xvf Quartus-17.1.0*.tar}
\item Run the installer:\newline
\code{\% ./setup.sh}\newline
\item Run through the installation process and choose your installation directory. Note that OpenCPI will search for Quartus Standard in \path{/opt/altera} or \path{~/intelFPGA} without any additional user settings.
\end{enumerate}

\subsubsection{OpenCPI Considerations}
It may required to set the following environment variables before running OpenCPI commands. Note that \texttt{<quartus-version>} should be replaced with the appropriate Quartus version (e.g. \code{17.1}), and \texttt{<quartus-install-dir>} should be replaced with the installation directory (\textit{e.g.} \path{~/intelFPGA}). Note also that each of the following \code{export} statements are only necessary when the non-default installation location (\textit{e.g.} anything other than \path{~/intelFPGA}, \path{/opt/intelFPGA}, \path{~/altera} or \path{/opt/Altera}), or non-default version (\textit{e.g.} anything other than the newest version) of the tools were used.\newline
\code{\% export OCPI\_ALTERA\_DIR=<quartus-install-dir>}\newline
\code{\% export OCPI\_ALTERA\_VERSION=<quartus-version>}\newline
\code{\% export OCPI\_ALTERA\_LICENSE\_FILE=<path\_to\_license\_file>}

These variables can be set automatically upon login for all users if added in \path{/opt/opencpi/cdk/env.d/altera.sh}. Logging out and logging back into the user account will apply said variables.

\end{flushleft}

\subsection{Intel Quartus Prime Pro Edition 17.0.2 Installation in CentOS~7}
\begin{flushleft}
	NOTE: Do not install Quartus Pro in the same directory as Quartus Standard because OpenCPI cannot differentiate between the two.\\
	NOTE: Quartus Pro and Quartus Standard are \textit{different tools}. The devices supported by each are different, and users should consult Intel documentation before choosing a tool edition.
\begin{enumerate}
	\item Download the Quartus Prime Pro Edition 17.0 installation files from Altera's download site: \url{https://www.intel.com/content/www/us/en/programmable/downloads/download-center.html}. Choose \verb+Pro Edition 17.0+ and either choose the ``Complete Download'', or the ``Multiple File Download'' (for this option, make sure to download the device packages of interest). An \verb+Intel Customer+ account will be required.
\item If installing Quartus tools in a permission-restricted directory, you may need to change the umask temporarily:\newline
\code{\% sudo su -}\newline
\code{\% umask 0002}
\item Extract the tarball:\newline
\code{\% tar xvf Quartus-pro-17.0.0*.tar}
\item Run the installer:\newline
\code{\% ./setup.sh}\newline
\item Run through the installation process and choose your installation directory. Note that OpenCPI will search for Quartus Pro in \path{~/intelFPGA_pro} or \path{/opt/intelFPGA_pro} without any additional user settings.
\item Download the 17.0.2 patch by navigating to the \verb+Updates+ tab and downloading ``Quartus Prime Software v17.0 Update 2''.
\item Run the installer:\newline
	\code{\% ./QuartusProSetup-17.0.2*.run}
\end{enumerate}

\subsubsection{OpenCPI Considerations}
It may be required to set the following environment variables before running OpenCPI commands. Note that \texttt{<quartus-version>} should be replaced with the appropriate Quartus version (e.g. \code{17.0} not \code{17.0.2}), and \texttt{<quartus-install-dir>} should be replaced with the installation directory (\textit{e.g.} \path{~/intelFPGA_pro}). Note also that each of the following \code{export} statements are only necessary when the non-default installation location (\textit{e.g.} anything other than \path{~/intelFPGA_pro}, \path{/opt/intelFPGA_pro}, \path{~/altera} or \path{/opt/Altera}), or non-default version (\textit{e.g.} anything other than the newest version) of the tools were used.\newline
\code{\% export OCPI\_ALTERA\_PRO\_DIR=<quartus-install-dir>}\newline
\code{\% export OCPI\_ALTERA\_PRO\_VERSION=<quartus-version>}\newline
\code{\% export OCPI\_ALTERA\_PRO\_LICENSE\_FILE=<path\_to\_license\_file>}

These variables can be set automatically upon login for all users if added in \path{/opt/opencpi/cdk/env.d/altera.sh}. Logging out and logging back into the user account will apply said variables.
\end{flushleft}

\begin{flushleft}
\subsection{Licensing Notes}
% AV-3316
If the user runs the Quartus software in its native GUI mode outside of OpenCPI, a license file configuration \textit{might} be stored in the variable \path{LICENSE_FILE} within \path{~user/.altera.quartus/quartus2.ini}; this setting overrides the \code{OCPI\_ALTERA\_LICENSE\_FILE} noted above and may cause confusion.

\end{flushleft}

\section{ModelSim Installation and Configuration}
\subsection{ModelSim DE 16.0e Installation in CentOS~7}
\begin{flushleft}
\begin{enumerate}
	\item Download the ModelSim installation files for version 10.6e.
\item If installing ModelSim tools in a permission-restricted directory, you may need to change the umask temporarily:\newline
\code{\% sudo su -}\newline
\code{\% umask 0002}
\item Run the installer:\newline
\code{\% ./install.linux64}
\item Run through the installation process and choose your installation directory. Note that OpenCPI has no default search paths for ModelSim installations.
\end{enumerate}


\subsubsection{OpenCPI Considerations}
Users will need to set the following environment variables to use ModelSim with OpenCPI. Note that \texttt{<modelsim-version>} should be replaced with the appropriate ModelSim version (e.g. \code{10.6}), and \texttt{<modelsim-install-dir>} should be replaced with the installation directory (\textit{e.g.} \path{~/modelsim_dlx}). The version variable need only be set if multiple ModelSim versions exist in this directory and the user wishes to use a version \textit{other than the most recent}.\\
\code{\% export OCPI\_MODELSIM\_DIR=<modelsim-install-dir>}\newline
\code{\% export OCPI\_MODELSIM\_VERSION=<modelsim-version>}\newline
\code{\% export OCPI\_MODELSIM\_LICENSE\_FILE=<path\_to\_license\_file>}

These variables can be set automatically upon login for all users if added in \path{/opt/opencpi/cdk/env.d/modelsim.sh}. Logging out and logging back into the user account will apply said variables.
\subsection{Compile Xilinx/Zynq simulation libraries for ModelSim}

	This section describes how to compile Xilinx simulation libraries of a device(s) for a particular 3rd party simulator, such as ModelSim.

	\begin{enumerate}
	 	\item Compile Xilinx libraries for ModelSim
		\item Modify \texttt{modelsim.ini} to include path of compiled Xilinx libraries
	\end{enumerate}

\subsubsection{Compile Vivado's simulation libraries}
	This section provides the steps necessary to compile Xilinx Vivado's simulation libraries of the Zynq device, for ModelSim. If using ModelSim 10.4c, note that Vivado 2017.1 does not support compilation of simulation libraries for ModelSim versions earlier than 10.5c. Therefore, if using a ModelSim 10.4c, you will need to use an earlier version of Vivado (\textit{e.g} 2015.4) to compile the simulation libraries. For this example, we use Vivado 2017.1 with ModelSim DE 10.6e.

\begin{flushleft}
	\begin{enumerate}
		\item Open a terminal window and switch the user to root:
			\subitem \code{> sudo su -}
		\item Configure the terminal for Xilinx Vivado by sourcing the setup script (for bash):
			\subitem \code{> source /opt/Xilinx/Vivado/<version>/settings64.sh}
		\item Launch Vivado:
			\subitem \code{> vivado}
		\item Select Tools $\rightarrow$ Compile Simulation Libraries...
		\item Select the following:
			\subitem Simulator: ModelSim Simulator
			\subitem Language: VHDL
			\subitem Library: All
			\subitem Family: Zynq-7000
			\subitem Compiled library location: \path{/opt/Xilinx/Vivado/<version>/vhdl/modelsim/<version>/lin64}
			\subitem Simulator executable path: \path{/opt/Modelsim/modelsim_dlx/linuxpe}
			\subitem Compile 32-bit libraries: Yes
		\item Click ``Compile''
		\item Note that 2017.1 Vivado will result in errors for ModelSim versions earlier than 10.5c. Here, we show the results for Vivado 2017.1 with ModelSim DE 10.6e, and Vivado 2015.4 with ModelSim DE 10.4c.
	\begin{figure}[H]
	\centering\captionsetup{type=figure}\includegraphics[scale=0.5]{figures/xilinx_vivado_2017_compsimlib_out}
		\captionof{figure}{Vivado 2017.1 Compilation Output with ModelSim DE 10.6e}
	\end{figure}
	\begin{figure}[H]
	\centering\captionsetup{type=figure}\includegraphics[scale=0.5]{figures/xilinx_vivado_2015_compsimlib_out}
		\captionof{figure}{Vivado 2015.4 Compilation Output with ModelSim DE 10.4c}
	\end{figure}
		\end{enumerate}
\end{flushleft}
\subsubsection{Compile ISE's simulation libraries}
	This section provides the steps necessary to compile Xilinx ISE's simulation libraries of the Zynq-7000 device, for ModelSim.

\begin{flushleft}
	\begin{enumerate}
	 	\item Open a terminal window and switch the user to root:
			\subitem \code{> sudo su -}
		\item Configure the terminal window for Xilinx ISE by sourcing the setup script (for bash):
			\subitem \code{> cd /opt/Xilinx/14.7/ISE\_DS/}
			\subitem \code{> source settings64.sh}
		\item Launch the Xilinx CompXLib GUI:
			\subitem \code{> cd /opt/Xilinx/14.7/ISE\_DS/ISE/bin/lin64}
			\subitem \code{> ./compxlib}

	\begin{figure}[H]
	\centering\captionsetup{type=figure}\includegraphics[scale=0.5]{figures/Xilinx_CompXLib_1_Select32bit}
		\captionof{figure}{Compilation Wizard - Select Simulator}
		\label{fig:wizard_page_1}
	\end{figure}

		\item Select ModelSim DE.
		\item Set Simulator Executable Location.
		\item Click ``Next''.

	\begin{figure}[H]
	\centering\captionsetup{type=figure}\includegraphics[scale=0.5]{figures/Xilinx_CompXLib_3_VHDLonly}
		\captionof{figure}{Compilation Wizard - HDLs to support simulator}
		\label{fig:wizard_page_3}
	\end{figure}

		\item Select ``VHDL''.
		\item Click ``Next''.

	\begin{figure}[H]
	\centering\captionsetup{type=figure}\includegraphics[scale=0.5]{figures/Xilinx_CompXLib_2_SelectZynq}
		\captionof{figure}{Compilation Wizard - Select Device Families}
		\label{fig:wizard_page_2}
	\end{figure}

		\item Uncheck ``All FPGA Device Families''.
		\item Uncheck ``All CPLD Device Families''.
		\item Check ``Zynq''.
		\item Click ``Next''.

	\begin{figure}[H]
	\centering\captionsetup{type=figure}\includegraphics[scale=0.5]{figures/Xilinx_CompXLib_4_BuildAll}
		\captionof{figure}{Compilation Wizard - Select Simulation Libraries}
		\label{fig:wizard_page_4}
	\end{figure}

		\item No change.
		\item Click ``Next''.

	\begin{figure}[H]
	\centering\captionsetup{type=figure}\includegraphics[scale=0.5]{figures/Xilinx_CompXLib_5_SelectDefaults}
		\captionof{figure}{Compilation Wizard - Select Output directory}
		\label{fig:wizard_page_5}
	\end{figure}

		\item Select defaults.
		\item Click ``Launch Compile Process''.
			\subitem Note: This step will take approximately 20 mins.

	\begin{figure}[H]
	\centering\captionsetup{type=figure}\includegraphics[scale=0.5]{figures/Xilinx_CompXLib_6_CompilationLog}
		\captionof{figure}{Compilation Wizard - Start Compilation}
		\label{fig:wizard_page_6}
	\end{figure}

		\item Click ``Next''.

	\begin{figure}[H]
	\centering\captionsetup{type=figure}\includegraphics[scale=0.5]{figures/Xilinx_CompXLib_7_CompilationSummary}
		\captionof{figure}{Compilation Wizard - Compilation Summary}
		\label{fig:wizard_page_7}
	\end{figure}


		\item Click ``Finish''.
	\end{enumerate}

\newpage

\end{flushleft}

\subsubsection{Modify ``modelsim.ini'' to include path to built library}
	This section details the steps to modify the ``\texttt{modelsim.ini}'' file.

	\begin{enumerate}
		\item Browse to the install directory of ModelSim
			\subitem \code{> cd /opt/Modelsim/modelsim\_dlx}
		\item Open the modelsim.ini file as the root user
			\subitem \code{> vi modelsim.ini}
		\item Locate the bottom of the ``\texttt{[Library]}'' section and add the following for Vivado:
			\subitem unifast = /opt/Xilinx/Vivado/2017.1/vhdl/modelsim/10.6e/lin64/unifast
			\subitem unisim = /opt/Xilinx/Vivado/2017.1/vhdl/modelsim/10.6e/lin64/unisim
		\item Or, add the following for ISE:
			\subitem xilinxcorelib = /opt/Xilinx/14.7/ISE\_DS/ISE/vhdl/mti\_de/10.4c/lin64/xilinxcorelib
			\subitem unisim = /opt/Xilinx/14.7/ISE\_DS/ISE/vhdl/mti\_de/10.4c/lin64/unisim

	\end{enumerate}

\end{flushleft}

\end{document}
